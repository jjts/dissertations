%%%%%%%%%%%%%%%%%%%%%%%%%%%%%%%%%%%%%%%%%%%%%%%%%%%%%%%%%%%%%%%%%%%%%%%%
%                                                                      %
%     File: Thesis_Glossary.tex                                        %
%     Tex Master: Thesis.tex                                           %
%                                                                      %
%     Author: Carlos A. Rodrigues                                           %
%     Last modified : 30 Oct 2012                                      %
%                                                                      %
%%%%%%%%%%%%%%%%%%%%%%%%%%%%%%%%%%%%%%%%%%%%%%%%%%%%%%%%%%%%%%%%%%%%%%%%
%
% The definitions can be placed anywhere in the document body
% and their order is sorted by <symbol> automatically when
% calling makeindex in the makefile
%
% The \glossary command has the following syntax:
%
% \glossary{entry}
%
% The \nomenclature command has the following syntax:
%
% \nomenclature[<prefix>]{<symbol>}{<description>}
%
% where <prefix> is used for fine tuning the sort order,
% <symbol> is the symbol to be described, and <description> is
% the actual description.

% ----------------------------------------------------------------------

\glossary{name={\textbf{OpenOCD}},description={Open On-Chip Debugger}}

\glossary{name={\textbf{JTAG}},description={joint Test Action Group}}

\glossary{name={\textbf{SCLK}},description={Serial Clock, linha de clock.}}

\glossary{name={\textbf{MOSI}},description={Master Output Slave Input, saida no mestre entrada no escravo .}}

\glossary{name={\textbf{MISO}},description={Master Input Slave Output, entrada no mestre saida no escravo .}}

\glossary{name={\textbf{SS}},description={Slave Select, Escolha de escravo.}}

\glossary{name={\textbf{SDA}},description={Serial Data Line, linha serie de dados .}}

\glossary{name={\textbf{SCL}},description={Serial Clock Line. linha serie de clock}}



