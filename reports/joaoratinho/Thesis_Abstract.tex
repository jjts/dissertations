%%%%%%%%%%%%%%%%%%%%%%%%%%%%%%%%%%%%%%%%%%%%%%%%%%%%%%%%%%%%%%%%%%%%%%%%
%                                                                      %
%     File: Thesis_Abstract.tex                                        %
%     Tex Master: Thesis.tex                                           %
%                                                                      %
%     Author: Andre C. Marta                                           %
%     Last modified :  2 Jul 2015                                      %
%                                                                      %
%%%%%%%%%%%%%%%%%%%%%%%%%%%%%%%%%%%%%%%%%%%%%%%%%%%%%%%%%%%%%%%%%%%%%%%%

\section*{Abstract}

% Add entry in the table of contents as section
\addcontentsline{toc}{section}{Abstract}

Versat is a Coarse-Grain Reconfigurable Array architecture (CGRA), which
implements self and partial reconfiguration by using a simple controller
unit. This report studies the current state of the art in HDL and CGRA
simulation, providing a basis to the development of a simulation environment for
Versat. The main objective of this environment is to provide a faster way to
develop and debug software without the use of prototyping hardware. Therefore,
the two types of HDL simulators, event-driven and cycle-accurate, their
advantages and disadvantages are studied, along with a performance comparison
between them. A study of high-level implementations for CGRA simulation is
also presented.

\vfill

\textbf{\Large Keywords:} Versat, coarse-grain reconfigurable arrays, HDL
simulation, CGRA simulation, high-level simulation

