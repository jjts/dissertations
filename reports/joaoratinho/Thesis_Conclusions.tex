%%%%%%%%%%%%%%%%%%%%%%%%%%%%%%%%%%%%%%%%%%%%%%%%%%%%%%%%%%%%%%%%%%%%%%%%
%                                                                      %
%     File: Thesis_Conclusions.tex                                     %
%     Tex Master: Thesis.tex                                           %
%                                                                      %
%     Author: Andre C. Marta                                           %
%     Last modified :  2 Jul 2015                                      %
%                                                                      %
%%%%%%%%%%%%%%%%%%%%%%%%%%%%%%%%%%%%%%%%%%%%%%%%%%%%%%%%%%%%%%%%%%%%%%%%

\chapter{Conclusions}
\label{chapter:conclusions}

In this report, a study of the state-of-the-art of the current HDL simulation
was made with the perspective of developing a new simulation environment for the
Versat CGRA architecture, which is the subject of the thesis dissertation.

The Versat architecture, a minimal CGRA with self-generated partial
reconfiguration, has been introduced. This architecture comprises a Data Engine
and a Configuration Module to implement the runtime reconfigurable datapaths
characteristic of CGRAs. Versat has a complete graph topology, which allows a
high level of configurability. Additionally, Versat features a Controller,
Program Memory, a Register File, a DMA, in a modular setup.

% ----------------------------------------------------------------------
\section{Achievements}
\label{section:achievements}

In the analysis of the state-of-the-art of HDL simulators, two main types have
been identified: event-driven and cycle-accurate simulators. While the
cycle-accurate ones have the advantage of providing faster simulations, they
have the disadvantage of not taking into account the different delays and
propagation times inside a circuit, which makes them less reliable.

The application of the two main types to CGRA simulation has been
analyzed.  It has been concluded that event-driven simulation should be used in
early stage development or simulation of individual modules but that for
complete system simulation it is preferable to use cycle-accurate simulators to
speed up verification.

Lastly, high-level simulation techniques that detach themselves from the above
traditional techniques have been discussed and two different approaches to this
topic have been presented. They bring the advantage of simulating the CGRA
behaviour from a higher level than traditional RTL simulators, evaluating just
the signals exchanged between the different functional units, instead of also
looking to the intermediate signals.

Through this work, not only was it possible to study and understand the
advantages and disadvantages of the different RTL simulators, but it was also
possible to assess their performance, by analysing the results in~\cite{verilator:benchmarks,chen:CGRA}. From these benchmarks it was seen that
Verilator, an open source cycle-accurate simulator, has a clear advantage over
the other commercially available HDL simulators.

The candidate spent a considerable time working with Versat in the scope of an
industrial internship at company IObundle, Lda, and participated in a project
with a real customer. In this project the candidate programmed Versat in
assembly participating in the acceleration of MP3 encoding. This experience is
invaluable for making correct design choices when developing the Versat
simulation environment.

% ----------------------------------------------------------------------
\section{Future Work}
\label{section:future}

The work to be developed in the thesis consists in creating a simulation
environment for the Versat architecture. This simulation environment must be
faster than the one that is currently used (based on NCsim), allowing for fast
software development and debugging before running it on an FPGA for more
exhaustive testing. This way, the environment that is going to be developed will
be based in Verilator, given its performance advantage over the other
simulators. The work that will be performed is shown in Table~\ref{tab:planning},
together with the scheduling for each task.

\begin{table}[!htbp]
	\centering
	\caption{Planning of the work that will be performed in the thesis}
	\label{tab:planning}
	\begin{tabular}{|p{8cm}|l|}
		\hline
		\textbf{Work Planning}                                                                            & \textbf{Scheduling} \\
		\hline
		Study Verilator documentation                                                                     & 18/02 - 22/02       \\
		\hline
		Perform simulations with Verilator using simple circuits                                          & 23/02 - 04/03       \\
		\hline
		Study the changes needed to the simulation environment in order to simulate Versat with Verilator & 05/03 - 09/03       \\
		\hline
		Apply the changes and perform the first simulations                                               & 10/03 - 10/04       \\
		\hline
		Create a testbench for the simulation environment                                                 & 11/04 - 29/04       \\
		\hline
		Test \& Debug the simulation environment                                                          & 30/04 - 20/05       \\
		\hline
		Write the Thesis report                                                                           & 21/05 - 01/07       \\
		\hline
	\end{tabular}
\end{table}

A high-level simulator that works at the functional unit level instead of the
RTL is going to be developed by another student. It will use a high-level
simulator specially developed for the Versat architecture, similarly to~\cite{chen:CGRA}, extracting the data from the memories and executing the
desired high-level operations (no bit-level operations) by reading the
configuration bits. This will allow for an even better performance but also has
a disadvantage: architecture changes in Versat would also require changes in the
simulator architecture. Note that Verilator is excellent at solving this problem
%JTS: faltou dizer atrás que o verilator compila um simulador para cada circuito a simular.
as RTL changes can simply be recompiled. This kind of high-level approach does
not evaluate any kind of timing in the circuit, but this is also not needed once
the circuit is silicone proven and its frequency of operation is known.

