%%%%%%%%%%%%%%%%%%%%%%%%%%%%%%%%%%%%%%%%%%%%%%%%%%%%%%%%%%%%%%%%%%%%%%%%
%                                                                      %
%     File: Thesis_Introduction.tex                                    %
%     Tex Master: Thesis.tex                                           %
%                                                                      %
%     Author: Andre C. Marta                                           %
%     Last modified :  2 Jul 2015                                      %
%                                                                      %
%%%%%%%%%%%%%%%%%%%%%%%%%%%%%%%%%%%%%%%%%%%%%%%%%%%%%%%%%%%%%%%%%%%%%%%%

\chapter{Introduction}
\label{chapter:introduction}


%%%%%%%%%%%%%%%%%%%%%%%%%%%%%%%%%%%%%%%%%%%%%%%%%%%%%%%%%%%%%%%%%%%%%%%%
\section{Motivation}
\label{section:motivation}

During the last few years reconfigurable architectures have evolved greatly, as
a result of the research that has been carried out in this topic. This has
created a vast market for this type of architectures due to the shorter time to
market and lower non-recurring engineering costs, when compared with other
solutions.

Some solutions use standard fine-grained reconfigurable architectures like
commercial FPGAs. However, they are often too large and power hungry to be used
as embedded cores. In this context, the Coarse Grain Reconfigurable Array (CGRA)
is a suitable alternative. Essentially, a CGRA consists of a collection of
functional units and memories, interconnected by programmable switches for
forming hardware datapaths that accelerate computations.

In CGRAs, like with any other digital circuit, simulation is fundamental to
verify if the circuit is working properly in the different stages of its
development. However, neither FPGAs nor CGRAs have good high-level simulation
tools. This means that they have to be simulated using simulators already
available in the market or, in alternative, using simulators developed just for
the architecture in consideration.


%%%%%%%%%%%%%%%%%%%%%%%%%%%%%%%%%%%%%%%%%%%%%%%%%%%%%%%%%%%%%%%%%%%%%%%%
\section{Topic Overview}
\label{section:overview}

In digital electronics, frequent simulation is required during the different
phases of project development. There are multiple tools, proprietary or open
source, that can be used. Each of these tools present different characteristics,
that can make them more or less adequate, depending not only on the circuit that
is going to be simulated, but also on the speed and accuracy of the desired
simulation.

In a CGRA, the reconfigurable interconnects between the functional units allow
building different datapaths during the run-time. This means that event-driven
simulators using HDL input are usually applied for generic simulation of
applications mapped on CGRAs, resulting in long simulation times, since the
simulator needs to consider all the events generated by the changing signals
inside the architecture.

Consequently, finding a valid alternative to simulate CGRAs (in this particular
case, the Versat architecture) could save an important amount of time and money
during the development of applications for this type of architectures.



%%%%%%%%%%%%%%%%%%%%%%%%%%%%%%%%%%%%%%%%%%%%%%%%%%%%%%%%%%%%%%%%%%%%%%%%
\section{Objectives}
\label{section:objectives}

The main objective of this work is to study the current state of the art in high
level simulation of CGRAs. This implies studying the current HDL Simulator
Technology, including the different types of simulators, and also the Versat
architecture, a CGRA that is suitable for low-cost and low-power devices. This
way, is possible to understand what types of simulators are more suitable for
the Versat architecture.

Finally, this work is intended to serve as a basis to the thesis dissertation,
where a simulation environment for the Versat reconfigurable processor will be
developed.


%%%%%%%%%%%%%%%%%%%%%%%%%%%%%%%%%%%%%%%%%%%%%%%%%%%%%%%%%%%%%%%%%%%%%%%%
\section{Author's Work}
\label{section:work}

The candidate developed this work during an internship at IObundle, Lda, the company that
developed the Versat architecture (and is sponsoring the candidate's thesis). This allowed the candidate
to gain a deeper knowledge of the Versat architecture, mainly by participating
in a project with an international client where the Versat architecture was used
to accelerate the front end of an MP3 encoder (as described in section~\ref{section:application}).

%%%%%%%%%%%%%%%%%%%%%%%%%%%%%%%%%%%%%%%%%%%%%%%%%%%%%%%%%%%%%%%%%%%%%%%%
\section{Report Outline}
\label{section:outline}

This report is composed by 3 more chapters. In the second chapter the Versat
architecture is presented. In the third chapter the current HDL simulation
state-of-the-art is analysed. Finally, the fourth chapter discusses the best
strategy for the problem addressed, with two different approaches to CGRA
simulation being analysed.

