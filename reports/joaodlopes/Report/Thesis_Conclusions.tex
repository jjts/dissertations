%%%%%%%%%%%%%%%%%%%%%%%%%%%%%%%%%%%%%%%%%%%%%%%%%%%%%%%%%%%%%%%%%%%%%%%%
%                                                                      %
%     File: Thesis_Conclusions.tex                                     %
%     Tex Master: Thesis.tex                                           %
%                                                                      %
%     Author: João D. Lopes                                            %
%     Last modified :  2 June 2016                                     %
%                                                                      %
%%%%%%%%%%%%%%%%%%%%%%%%%%%%%%%%%%%%%%%%%%%%%%%%%%%%%%%%%%%%%%%%%%%%%%%%

\chapter{Conclusions}
\label{chapter:conclusions}

In this report we have introduced a new CGRA architecture named Versat
and we have compared it with some existing CGRAs architectures. The
main difference is that Versat can take care of the reconfiguration
process itself as well as the data movement operations to and from an
external memory. For that, runtime partial reconfigurations triggered
by an internal controller are used. This allows Versat to accelerate
complex kernels such as FFTs without using too much hardware.

Versat is a minimal CGRA with 4 embedded memories, 11 FUs and a
basic 16-instruction controller. Compared with other CGRAs with larger
arrays, Versat requires more configurations per kernel and a more
sophisticated reconfiguration mechanism. Thus, the Versat controller
can generate configurations and uses partial reconfiguration whenever
possible. The controller is also in charge of data transfers and basic
algorithmic flows.


%Insert your chapter material here...


% ----------------------------------------------------------------------
\section{Achievements}
\label{section:achievements}

In this first version of Versat, the results show that the speedup
improves with the kernel complexity. The kernel complexity depends on
the number of datapaths that can be run sequentially using the data
already in the Versat memories. In fact, our results show that single
datapath kernels achieve speedups in the order of 2x while multiple
datapath kernels can achieve speedups an order of magnitude higher.

Unlike other CGRAs, which are designed to accelerate one program loop,
Versat is designed to accelerate a sequence of chained program loops,
where the results produced in one loop are consumed by the next loop. It
has been explained that most of the times the next configuration can
be generated while the DMA or the current configuration is running on
the data engine. Because the Versat controller can generate
configurations, these do not need to be stored in external memory and
then moved into Versat. In general, the code to generate
configurations is much smaller than the configurations themselves.

Versat has been implemented in FPGA and in ASIC technology.  In terms
of silicon area, Versat is comparable to a basic low range CPU.
Results on a VLSI implementation show that Versat is competitive in
terms of silicon area, frequency of operation and power
consumption. Performance results on running a 1024-point FFT show that
a system combining a state-of-the-art embedded processor and the
Versat core can be 17x faster and more energy efficient than the
embedded processor alone. Overall performance depends of course on how
often such kernels are used in a real system. However, knowledge of
multimedia algorithms can reveal that between 40\% to 80\% of the
execution time in a regular CPU can use an accelerator such as the one
described here. This means that overall speedups of 2x to 5x can be
expected in real systems.

%The major achievements of the present work...


% ----------------------------------------------------------------------
\section{Future Work}
\label{section:future}

The work on Versat can be continued in many different fronts. First
of all, more examples as compelling as the FFT example need to be
implemented to better asses Versat's capabilities. In fact, complete
applications using several such kernels should be developed.

A compiler is already under development and needs to be constantly evolved. Software
libraries for host processors to use Versat are also needed. For
example, to run the FFT kernel, the host program would just call a
function {\tt fft(int*x,int *X, int n)}, where $x$ is a pointer to the
input vector, $X$ is a pointer to the result vector and $n$ is the
size of the vectors.

Finally, more work on the architecture is needed. One would like to
support fast changes to Versat's data engine and that such changes are
automatically reflected on the assembler and compiler tools. This
would allow targeting new classes of applications in a short period of
time. Another area is to incorporate floating-point units in the data
engine. Other than that, many small hardware optimizations can be
effected to improve the area and frequency of operation of certain
blocks.
