%%%%%%%%%%%%%%%%%%%%%%%%%%%%%%%%%%%%%%%%%%%%%%%%%%%%%%%%%%%%%%%%%%%%%%%%
%                                                                      %
%     File: Thesis_Results.tex                                         %
%     Tex Master: Thesis.tex                                           %
%                                                                      %
%     Author: João D. Lopes                                            %
%     Last modified :  3 June 2016                                     %
%                                                                      %
%%%%%%%%%%%%%%%%%%%%%%%%%%%%%%%%%%%%%%%%%%%%%%%%%%%%%%%%%%%%%%%%%%%%%%%%

\chapter{Results}
\label{chapter:results}

%% rever
This chapter presents results on implementing Versat on FPGA, as well
as on ASIC technology. Performance results comparing Versat to an FPGA
embedded processor (Microblaze, from Xilinx) and to an ARM Cortex A9
system are also presented.


% ----------------------------------------------------------------------
\section{FPGA implementation results}
\label{subsection:FPGAresults}

Versat's FPGA implementation results are given in
Table~\ref{tabFPGAr}. These results show that in terms of size, Versat
can fit in the smallest FPGAs: in a Xilinx XC5VLX50T device of the
Virtex V family, Versat occupies half of the logic resources; in an
Altera Cyclone IV EP4CE22F17C6N device, Versat takes about 80\% of the
logic resources. The Xilinx implementation makes better use of the RAM
resources for implementing the configuration memory than the Altera
device. This explains why the Xilinx implementation consumes more RAM
memory but less logic (LookUp Tables or LUTs) and the Altera device
consumes more Logic Elements (LEs) and less RAM. This is also
reflected in the frequency of operation, although the fact that we are
comparing two different architectures also plays an important role in
this difference, since the Virtex V architecture is known to achieve
higher frequencies of operation compared to the Cyclone IV
architecture.  Nevertheless, it is a fact that the current Versat
implementation is not very well optimized in terms of maximum
frequency of operation and more work can be done on this once it
becomes a priority.


\begin{table}[!htb]
  \renewcommand{\arraystretch}{1.2} % more space between rows
  \caption{FPGA implementation results}
  \label{tabFPGAr}
  \centering
  \begin{tabular}{lccccc}
    \toprule
    Architecture & Logic & Regs & RAM(KB) & Mults & Fmax(MHz)\\
    \midrule
    Cyclone IV &  19366 LEs & 4673 &  43.88 & 32 (9 bits)  & 64 \\
    Virtex V   & 12510 LUTs & 4396 & 132.75 & 16 (18 bits) & 102\\
    \bottomrule
  \end{tabular}
\end{table}

%Insert your section material and possibly a few figures...

%Make sure all figures presented are referenced in the text!

% ----------------------------------------------------------------------
\section{ASIC implementation results}
\label{subsection:ASICresults}

Versat has been designed using a UMC 130nm
process. Table~\ref{tabASICr} compares Versat with a state-of the-art
embedded processor and two other CGRA implementations. The Versat
frequency and power results have been obtained using the Cadence IC
design tools, and the node activity rate extracted from simulating an
FFT kernel.

\begin{table}[!htb]
  \renewcommand{\arraystretch}{1.2} % more space between rows
  \caption{ASIC implementation results}
  \label{tabASICr}
  \centering
  \begin{tabular}{lccccc}
    \toprule
    Core & Node(nm) & Area(mm\textsuperscript{2}) & RAM(KB) &  Freq.(MHz) & Power(mW)\\
    \midrule
    ARM Cortex A9~\cite{wang} &  40 & 4.6 & 65.54 & 800 &  500 \\
    Morphosys~\cite{Lee00}    & 350 & 168 &  6.14 & 100 & 7000 \\
    ADRES~\cite{Mei05}        &  90 &   4 & 65.54 & 300 &   91 \\
    Versat                    & 130 & 4.2 & 46.34 & 170 &   99 \\
    \bottomrule
  \end{tabular}
\end{table}

Because the different designs use different technology nodes, to
compare the results in Table~\ref{tabASICr}, we need to use a scaling
method. A standard scaling method is to assume that the area scales
with the square of the feature size and that the power density remains
constant at constant frequency. Doing that we conclude that Versat is
the smallest and least power hungry of the CGRAs. If Versat were
implemented in the 40nm technology, it would occupy about 0.4
mm\textsuperscript{2}, and consume about 44mW running at a frequency
of 800MHz. That is, Versat is 10x smaller and consumes about 11x less
power compared with the ARM processor.

The ADRES architecture is about twice the size of Versat. Morphosys is
the biggest one, occupying half the size of the ARM processor. These
differences can be explained by the different capabilities of these
cores. While Versat has a 16-instruction controller and 11 FUs
(excluding the memory units), ADRES has a VLIW processor and a 4x4 FU
array, and Morphosys has a RISC processor and an 8x8 FU array.


% ----------------------------------------------------------------------
\section{Execution results}
\label{subsection:executionResults}

The execution results of running a set of example kernels on Versat
and on a embedded processor are divided in two parts: 1) Versat is
compared with Microblaze (Xilinx embedded processor); 2) Versat is
compared with an ARM Cortex A9 system embedded in a Zynq FPGA. In both
cases a hardware timer has been used to measure the time in elapsed
clock cycles. All kernels used for tests operate on vector sizes of 1024.
The {\tt lpf}, {\tt lpf2} and {\tt fft} kernels use Q1.31 fixed-point format.

\subsection{Comparing with the Microblaze processor}

The execution results using Microblaze for comparison are given in
Table~\ref{tabExecR}. The Microblaze and Versat programs are initially
placed in their local on-chip memories. Microblaze is configured with
a 32KB one-way data cache. In Table~\ref{tabExecR}, column {\it MB1}
gives the clock cycle counts for MicroBlaze when the data is placed in
external memory, column {\it MB2} gives the clock cycle counts for
MicroBlaze when the data is placed in the cache, column {\it Versat1}
gives the cycle counts for Versat when the data is placed in external
memory and column {\it Versat2} gives the Versat cycles when the data
is placed in the DE memories. Two speedup results are presented:
Speedup1 is the speedup including data transfer time and Speedup2 is
the speedup excluding data transfer time. Comparing these two speedups
one can have an idea of the overhead of moving data.

\begin{table}[!htb]
  \renewcommand{\arraystretch}{1.2} % more space between rows
  \caption{Execution results using Microblaze for comparison}
  \label{tabExecR}
  \centering
  \begin{tabular}{lccccccc}
    \toprule
    Kernel & MB1 & MB2 & Versat1 & Versat2 & Speedup1 & Speedup2\\
    \midrule
    {\tt vec\_add} &  11356 &   6147 &  4517 &  1090 &   2.51 &  5.64\\
    {\tt lpf1}     &  15311 &  10756 &  7487 &  5205 &   2.05 &  2.07\\
    {\tt lpf2}     &  20644 &  16388 & 10567 &  8310 &   1.95 &  1.97\\
    {\tt cdp}      &  25110 &  12292 &  6673 &  2185 &   3.76 &  5.63\\
    {\tt fft}      & 260226 & 238749 & 16705 & 12115 &  15.58 & 19.71\\
    \bottomrule
  \end{tabular}
\end{table}

Kernel {\tt vec\_add} is a vector addition, {\tt lpf1} and {\tt lpf2}
are $1^{st}$ and $2^{nd}$ order IIR filters, {\tt cdp} is a complex
dot (inner) product and {\tt fft} is a Fast Fourier Transform. The
first 4 kernels use a single Versat configuration. The {\tt fft}
kernel uses multiple Versat configurations.

In terms of performance results, kernel {\tt vec\_add} is a fully
pipelined kernel and produces one vector element result per cycle. It
can be accelerated 5.64x in Versat but if the data is transferred into
Versat just for running this kernel then the acceleration is only
2.51x. The processing time is only 1090 cycles and the remaining 3427
cycles account for data transfer and control.  This means that if the
data is already in the Microblaze data cache it may not be worth it to
run this kernel on Versat. On the other hand, if this datapath is part
of a larger kernel where the data is already in the Versat memories,
it becomes advantageous to use Versat.

Kernels {\tt lpf1} and {\tt lpf2} have similar acceleration of about
2x with or without the data in cache. This is because the acceleration
in moving the data is similar to the processing acceleration. Thus it
always pays off to have these kernels executed in Versat. The modest
yet effective speedup is due to the feedback loops needed to implement
the filters (loop carried dependencies); they produce new vector
elements every 5 and 8 cycles, respectively.

Kernel {\tt cdp} is more complex with 4 multipliers in parallel
followed by two adders. Despite the deeper pipeline, the processing
speedup is not better than for the {\tt vec\_add}
kernel. Unfortunately, the adders can not do an accumulation per cycle
and need an external feedback loop which causes a new vector element
result is accumulated every other cycle. Adding an accumulator mode to
the ALUs is straightforward and will be considered in the future in
order to double the processing speedup.

The {\tt fft} kernel is the most complex kernel and goes through 43
Versat configurations generated on the fly by the Versat
controller. The processing time is 12115 cycles and the remaining 4590
cycles is for data transfer and control. Note that most of the control
is done while either the DMA or the data engine is running, and the
controller runs alone for only 566 cycles. The processing time is
almost 20x smaller compared to Microblaze. Thus, a slower acceleration
in data movement brings the overall speedup down to about 16x.


\subsection{Comparing with ARM}

A prototype has been built using a Xilinx Zynq 7010 FPGA, which
features a dual-core embedded ARM Cortex A9 system. Versat is
connected as a peripheral of the ARM cores using its AXI4 slave
interface. The ARM core and Versat are connected to an on-chip memory
controller using their AXI master interfaces. The memory controller is
connected to an off-chip DDR module.

The execution results using an ARM Cortex A9 core for comparison are
given in Table~\ref{tabExecR2}. In column {\it ARM}, the clock cycle
counts is given, including the time to move the data between external
DDR and the data cache. The total cycle counts for Versat are given in
the {\it Versat} column, including the time to move the data between
the Versat memories and the DDR. Column {\it Speedup} column gives the
measured speedup and the {\it Energy Ratio} column compares the energy
spent by an ARM system to the energy spent by ARM+Versat system. The
speedup and energy ratio have been obtained assuming the ARM is
running at 800 MHz and Versat is running at 600MHz in the 40nm
technology. The energy ratio is the ratio between the energy spent by
the ARM processor alone and the energy spent by an ARM+Versat combined
system using the power figures in Table~\ref{tabASICr}.

\begin{table}[!htb]
  \renewcommand{\arraystretch}{1.2} % more space between rows
  \caption{Execution results using ARM for comparison}
  \label{tabExecR2}
  \centering
  \begin{tabular}{lcccc}
    \toprule
    Kernel & ARM & Versat & Speedup & Energy Ratio\\
    \midrule
    {\tt vec\_add} &  14726 &  4517 &  2.45 &  2.29\\
    {\tt lpf1}     &  18890 &  7487 &  1.89 &  1.77\\
    {\tt lpf2}     &  24488 & 10567 &  1.74 &  1.62\\
    {\tt cdp}      &  25024 &  6673 &  2.81 &  2.63\\
    {\tt fft}      & 394334 & 16705 & 17.70 & 16.55\\
    \bottomrule
  \end{tabular}
\end{table}

Surprisingly, despite all its advanced features, the ARM system does
not do much better compared to the Microblaze results in
Table~\ref{tabExecR}. In fact for the {\tt fft} kernel Microblaze
slightly outperforms the ARM system. If this result can be
generalized, it suggests that simpler processor architectures combined
with accelerators like Versat would yield even better silicon area and
power results. Note that, if implemented in silicon with roughly the
same amount of embedded RAM, a Microblaze core would be similar in
size to a Versat core if not smaller.

The ARM processor has much more hardware than Microblaze. ARM can
fetch two instructions in paralell and dispatch four instructions in
parallel. It has two pipelined multipliers, while Microblaze has only
one. The reason why it underperforms the MicroBlaze in the {\tt fft}
kernel, can be explained by a number of factors: one factor may be the
memory hierarchy, which in the ARM case is composed by two cache
levels instead of one level in the Microblaze case; Microblaze runs
the instructions from the local memory (with 1 clock cycle latency),
while the ARM uses its on-chip memory which is in parallel with the
level 2 cache and still has to go through level 1 cache; finally,
there may be compiler differences that result in two different machine
codes.


\subsection{Comparing with other CGRAs}

Comparing Versat with Morphosys is possible since it is reported
in~\cite{Kamalizad03} that the processing time for a 1024-point FFT is
2613 cycles. Compared with the 12115 cycles taken by Versat this means
that Morphosys was 4.6x faster. This is not surprising since Morphosys
has 64 FUs compared to 11 FUs in Versat. However, our point is whether
an increased area and power consumption is justified when the CGRA is
integrated in a real system. Note that, if scaled to the same
technology, Morphosys would be 5x the size of Versat. Unfortunately,
comparisons with the ADRES architecture have not been possible, since
we have not found any cycle counts published, despite ADRES being one
of the most published CGRA architectures.


