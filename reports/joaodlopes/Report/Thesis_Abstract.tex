%%%%%%%%%%%%%%%%%%%%%%%%%%%%%%%%%%%%%%%%%%%%%%%%%%%%%%%%%%%%%%%%%%%%%%%%
%                                                                      %
%     File: Thesis_Abstract.tex                                        %
%     Tex Master: Thesis.tex                                           %
%                                                                      %
%     Author: João D. Lopes                                            %
%     Last modified :  18 May 2016                                     %
%                                                                      %
%%%%%%%%%%%%%%%%%%%%%%%%%%%%%%%%%%%%%%%%%%%%%%%%%%%%%%%%%%%%%%%%%%%%%%%%

\section*{Abstract}

% Add entry in the table of contents as section
\addcontentsline{toc}{section}{Abstract}

This report introduces Versat, a reconfigurable hardware accelerator
to be used in an embedded system in order to optimize performance and
power. Versat is a Coarse-Grain Reconfigurable Array architecture
(CGRA), which implements self and partial reconfiguration by using a
simple controller unit. Compared to other CGRAs, Versat has a smaller
number of functional units with a fully connected graph topology for
maximum flexibility. The idea is to operate in a region of the design
space, where less computations can be mapped onto the array but the
array itself is more frequently reconfigured. An original feature of
Versat is the ability to map sequences of nested program loops instead
of just one program loop. Reconfiguration happens when moving from one
nested loop to the next. Experimental results are presented.

\vfill

\textbf{\Large Keywords:} reconfigurable computing, coarse-grain reconfigurable arrays, embedded systems

