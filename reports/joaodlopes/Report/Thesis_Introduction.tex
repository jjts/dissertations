%%%%%%%%%%%%%%%%%%%%%%%%%%%%%%%%%%%%%%%%%%%%%%%%%%%%%%%%%%%%%%%%%%%%%%%%
%                                                                      %
%     File: Thesis_Introduction.tex                                    %
%     Tex Master: Thesis.tex                                           %
%                                                                      %
%     Author: João D. Lopes                                            %
%     Last modified :  2 June 2016                                     %
%                                                                      %
%%%%%%%%%%%%%%%%%%%%%%%%%%%%%%%%%%%%%%%%%%%%%%%%%%%%%%%%%%%%%%%%%%%%%%%%

\chapter{Introduction}
\label{chapter:introduction}

\section{Motivation}
\label{section:motivation}

With the expansion in the number of features, modern embedded systems
are becoming more and more power hungry. Therefore it is crucial to
minimize power consumption. The main reason for this low power demand
is the short battery life of most devices used daily.

Another problem is the device price. The smaller the circuit the
cheaper it will be, and higher profit margins can be obtained from
it. Therefore, if one can deliver the same functionality with a
smaller and more power efficient device, one will have a more
competitive product.

These problems have been tackled with smaller and smaller transistors
which can work at higher frequencies without changing the
architecture. However, with the eminent demise of Moore's law and the
advent of the Internet of Things (IoT), this solution is not
viable. Reconfigurable hardware, which is known to be extremely
efficient in terms of power consumption and silicon area utilization,
has been used in mid-range to high-end applications. Fine grain
reconfigurable fabrics such as Field-Programmable Gate Arrays (FPGAs)
can be an alternative but, compared to Coarse-Grain Reconfigurable Arrays
(CGRAs), embedded FPGA cores consume significantly more silicon area and
power.

A CGRA is a collection of programmable functional units and embedded
memories connected by programmable interconnects. This structure is
what is called the reconfigurable array. When given a set of
configuration bits, the reconfigurable array forms a hardware datapath
able to execute a certain task orders of magnitude faster than a
conventional Central Processing Unit (CPU). CGRAs are used as hardware
co-processors to accelerate algorithms that are time/power consuming in
regular CPUs.

Normally, the reconfigurable array is used only to accelerate program
loops and the non-loop code is run on an attached processor which has
a conventional architecture. For this reason, CGRAs normally include a
conventional processor. For example, the Morphosys
architecture~\cite{Lee00} integrates a small Reduced Instruction Set
Computer (RISC) and the ADRES architecture~\cite{Mei05} integrates a
Very Large Instruction Word (VLIW) processor.


\section{Problems addressed}
\label{section:problemsAddressed}

The main problems that we have identified with existing CGRAs are
their large size, limited reconfiguration control and difficulty
in programming. Therefore, we propose some architectural improvements
to address these problems which follow three basic ideas.

The first idea is to make the CGRA smaller and to use fully connected
graph topology. Normally, graphs with constrained connectivity are
employed in CGRAs to avoid decreasing the frequency of
operation. However, low power devices will rarely choose to operate at
a high frequency, so using a fully connected graph becomes a
possibility. In terms of silicon area, fewer compute nodes can be used
if all nodes are connected to one another but more routing resources
are needed. Fortunately, these routing resources do not increase power
consumption as our reconfiguration frequency is low. In fact, our
reconfiguration process is more frequent compared to static arrays
such as~\cite{Hartenstein99}, but much less frequent compared to
dynamic arrays such as~\cite{Mei05}. Although we cannot build large
datapaths with many functional units, because we use fewer, we can
build {\it any} datapath with the functional units we do have.
Finally, programming is drastically simplified as there is no need for
{\em place \& route} algorithms in the compilation flow such as in
FPGAs and other fabrics.

The second idea is to make the configuration register addressable to
the word level. The configuration is divided in spaces, which
correspond to each functional unit, and the configuration spaces are
further divided in configuration fields which are made individually
addressable. Partial reconfiguration is useful to keep reconfiguration
to a minimum, exploiting the similarity between successive
configurations. Reducing reconfiguration time has a dramatic influence
on improving the performance, which can compensate for a lower
frequency of operation.

The third idea is to integrate a specialized and programmable
controller in the CGRA to manage reconfigurations and data transfers
to/from external memory. The controller is in charge of the main
program flow of the accelerator, sequencing the configurations and
using partial reconfiguration whenever possible. The controller can
spawn data stream compute threads in the CGRA and data movement
threads using a Direct Memory Access (DMA) unit. While these threads are
running, the controller can prepare the next configurations.


\begin{comment}

Normally, the reconfigurable array is used only to accelerate program
loops and the non-loop code is run on an attached processor which has
a conventional architecture. For this reason, CGRAs normally include a
conventional processor. For example, the Morphosys
architecture~\cite{Lee00} integrates a small Reduced Instruction Set
Computer (RISC) and the ADRES architecture~\cite{Mei05} integrates a
Very Large Instruction Word (VLIW) processor.

\end{comment}

%Relevance of the subject...


%%%%%%%%%%%%%%%%%%%%%%%%%%%%%%%%%%%%%%%%%%%%%%%%%%%%%%%%%%%%%%%%%%%%%%%%
\section{Topic Overview}
\label{section:overview}

In this work we presente Versat, a new reconfigurable hardware
accelerator which is suitable for low-cost low-power devices. Its
architecture uses a relatively small number of functional units and a
simple controller. A smaller array limits the size of the data
expressions that can be mapped to the CGRA but large expressions can
be broken into smaller expressions which can be executed sequentially
in the CGRA. Therefore, Versat requires mechanisms for handling large
numbers of configurations and frequent reconfigurations efficiently.

Versat is to be used as a co-processor featuring an Application Programming
Interface (API) containing a set of useful kernels. Applications developers
can use a commercial embedded processor with a rich ecosystem and drop in a
Versat core for performance and power optimization. Versat programmers can
create a set of useful kernels that application programmers will want to
use. In this way, the software and programming tools of the CGRA are
clearly separated from those of the application processor. This makes
Versat suitable for supporting the Open Computing Language (OpenCL)
standard or others.

%Provide an overview of the topic to be studied...


%%%%%%%%%%%%%%%%%%%%%%%%%%%%%%%%%%%%%%%%%%%%%%%%%%%%%%%%%%%%%%%%%%%%%%%%
\section{Objectives}
\label{section:objectives}

The main objective of this new architecture is to get acceleration
with low power consumption and small silicon area.

In fact, Versat can replace a number of dedicated hardware
accelerators in a System on Chip (SoC), making it smaller, more power
efficient and safer to design (the development risk of designing
dedicated hardware accelerators is eliminated).

Digital signal processing applications are targeted: biometrics,
speech recognition, artificial vision, security, etc. The overall goal
of the project is to create an Intellectual Property (IP) core and a
library of useful procedures. A clean procedural interface to a host
becomes possible. With such an interface the host can have tasks
executed in the CGRA by simply calling procedures and passing
arguments.

%Explicitly state the objectives set to be achieved with this thesis...

%%%%%%%%%%%%%%%%%%%%%%%%%%%%%%%%%%%%%%%%%%%%%%%%%%%%%%%%%%%%%%%%%%%%%%%%
\section{Author's Work}
\label{section:authorWork}

%falar no trabalho feito por outros; trabalho feito pelo autor
The work presented here is the result of the work of a few
people. When I started working on this project, in the summer of 2014,
there was a preliminary, non-functional version of the system.  I
contributed the multi-threading idea, where the state machines of the
address generators work independently, I pipelined parts of the
implementation to improve the frequency of operation, I wrote the boot
Read Only Memory (ROM) software and its hand shaking protocol with a
host system. I also implemented the DMA design, master and slave
Advanced Extensible Interface (AXI) interfaces, contributed to the
Application-Specific Integrated Circuit (ASIC) implementation, and,
finally, I wrote all assembly kernels used for tests and implemented a
regression test system where it is easy to add or remove new tests. As
a result of this work, the team has already a paper accepted for
publication~\cite{deSousa16}.

%%%%%%%%%%%%%%%%%%%%%%%%%%%%%%%%%%%%%%%%%%%%%%%%%%%%%%%%%%%%%%%%%%%%%%%%
\section{Report Outline}
\label{section:outline}

This report is composed by a few more chapters. In the second chapter,
the advantages and disadvantages of other architectures are
discussed. In the third chapter, Versat's architecture is fully
described. In the fourth chapter, some experimental results are
presented. In the fifth and final chapter, our achievements are
pointed out and some future work is outlined.

%Briefly explain the contents of the different chapters...

