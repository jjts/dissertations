%%%%%%%%%%%%%%%%%%%%%%%%%%%%%%%%%%%%%%%%%%%%%%%%%%%%%%%%%%%%%%%%%%%%%%%%
%                                                                      %
%     File: Thesis_Introduction.tex                                    %
%     Tex Master: Thesis.tex                                           %
%                                                                      %
%     Author: Andre C. Marta                                           %
%     Last modified :  2 Jul 2015                                      %
%                                                                      %
%%%%%%%%%%%%%%%%%%%%%%%%%%%%%%%%%%%%%%%%%%%%%%%%%%%%%%%%%%%%%%%%%%%%%%%%

\chapter{Introduction}
\label{chapter:introduction}

%%%%%%%%%%%%%%%%%%%%%%%%%%%%%%%%%%%%%%%%%%%%%%%%%%%%%%%%%%%%%%%%%%%%%%%%
\section{Topic Overview}
\label{section:overview}

In 1977, with the growing popularity of audio interfaces, the Audio Engineering
Society (AES) founded the AES Digital Audio Standards Committee, creating some of
the most used audio standards to this day. One of the most popular standards, the
AES3 digital audio interface, is stil used in current audio equipment, like
microphones and speakers with XLR connectors.

Since then, there has been a significant rise in AES standards, most of them
demanding the conversion of an audio signal's sample rate. One of the classical
examples is the the conversion from CD quality with a sampling rate $F_s = 44.1 kHz$
to DVD quality with $F_s = 48 kHz$. By consequence, there is a great demand for
sample rate converters, both in the form of software algorithms and hardware
designs.

To answer this need, some integrated circuit manufacturers developed multiple
sample rate converters with varied specifications. The AD1896~\cite{ad1896}, for
instance, is an asynchronous sample rate converter made by Analog Devices, in
2003. It supports a stereo (2 channel) audio signal and converts its sampling
rate from 7.75:1 to 1:8 ratios. Another example is the SRC4194~\cite{src4194},
manufactured by Texas Instruments since 2004, supporting up to 4 channel inputs,
and a wider sampling rate ratio, ranging from 16:1 to 1:16 ratios. The first
Intelectual Property (IP) was developed by Coreworks and called the CWda52~\cite{cwda52},
a multi-channel sample rate converter, able to convert sampling
rates using ratios from 8:1 to 1:8.

%%%%%%%%%%%%%%%%%%%%%%%%%%%%%%%%%%%%%%%%%%%%%%%%%%%%%%%%%%%%%%%%%%%%%%%%
\section{Motivation}
\label{section:motivation}

While there are currently hardware implementations of asynchronous sample rate
converters, most of them are integrated in a circuit (IC), while implementations
as intelectual property (IP) are few and have limited specifications compared to
their IC counterpart. This makes it difficult for some companies that
manufacture devices like television sets to integrate a cheaper sample rate
converter which fulfills the specifications of the respective chipsets.

The overall growth of the market for embedded audio systems which require the
sample rate conversion function motivates the development of better sample rate
converter IP cores, as a cheaper and more versatile alternative to discrete
solutions.

%%%%%%%%%%%%%%%%%%%%%%%%%%%%%%%%%%%%%%%%%%%%%%%%%%%%%%%%%%%%%%%%%%%%%%%%
\section{Objectives}
\label{section:objectives}

The main objective of this work is to design an asynchronous sample rate
converter (ASRC) IP core using the Verilog hardware description language. This
converter should support any input or output signal, sampled from
\SI{8}{\kilo\hertz} to \SI{192}{\kilo\hertz}, with up to 24-bit samples. To
ensure high audio quality, the sample rate converter should ensure a total
harmonic distortion plus noise ratio (THD+N) of -130 dB or less. It should
also support multi-channel signals, with a channel limit that depends
on the computation time restraints.

The ASRC should be tested in simulation and field programmable array logic
(FPGA) boards, using a PC to send and receive audio signals via Ethernet to the
FPGA board; the audio signals originate from .wav files, and the result of
sample rate conversion is stored back in another .wav file.

At the initial stages of this work some of the most used sample rates will be
tested while the input sample rate is kept constant. Later, dynamically varying
input sample rates will be tested. The work requires the study of digital signal
processing techniques, with emphasis on digital signal filtering and
interpolation techniques.

%%%%%%%%%%%%%%%%%%%%%%%%%%%%%%%%%%%%%%%%%%%%%%%%%%%%%%%%%%%%%%%%%%%%%%%%
\section{Author's Work}
\label{section:work}

Part of the work reported here was done in a summer internship at IObundle. At
the beginning of the internship, a non-functional Verilog prototype was already
available, as well as an Octave model detailing the structure of the ASRC. This
made it possible not only to gain knowledge of the dataflow of the sample rate
conversion algorithm, but also to have a starting point for developing of the
system. The work was focused on the reduction of the output signal's THD+N and
optimization of the measurement of input and output sample rates for overall
synchronization.

%%%%%%%%%%%%%%%%%%%%%%%%%%%%%%%%%%%%%%%%%%%%%%%%%%%%%%%%%%%%%%%%%%%%%%%%
\section{Report Outline}
\label{section:outline}

This report is composed of 4 more chapters. In the second chapter, the sample
rate conversion algorithm will be generically explained and some sample rate
converter designs will be presented, along with their advantages and
disadvantages. In the third chapter, a hardware implementation is proposed,
containing the architecture of the core along with its interfaces and
submodules. In the fourth chapter, the process of simulation and implementation
of the core is described and some preliminary results are presented. In the
fifth and final chapter, some conclusions of the research needed for this thesis
are drawn, and an outline of the work that still needs to be done is defined.
