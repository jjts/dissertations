%%%%%%%%%%%%%%%%%%%%%%%%%%%%%%%%%%%%%%%%%%%%%%%%%%%%%%%%%%%%%%%%%%%%%%%%
%                                                                      %
%     File: Thesis_Abstract.tex                                        %
%     Tex Master: Thesis.tex                                           %
%                                                                      %
%     Author: Andre C. Marta                                           %
%     Last modified :  2 Jul 2015                                      %
%                                                                      %
%%%%%%%%%%%%%%%%%%%%%%%%%%%%%%%%%%%%%%%%%%%%%%%%%%%%%%%%%%%%%%%%%%%%%%%%

\section*{Abstract}

% Add entry in the table of contents as section
\addcontentsline{toc}{section}{Abstract}

Currently, in digital audio processing, sampling rate convertions
are common. While there solutions for this problem is the form of
software and integrated circuit implementations, there is no viable
solution in the form of intellectual property, existing only one
(limited) solution.

A sample rate converter can be structured as: (1) an interpolation
module, (2) a reconstruction and anti-aliasing filter
(fractional delay filter), and (3) a decimation module.
This structure can be optimized by joining these modules,
avoiding computation of decimated samples.
The two main concerns of this design are the design of the
fractional delay filter, and the synchronization mechanisms
used to guarantee that the filter is adjusted with changes
on the conversion ratio.

The focus of this report is on the study of asynchronous sampling rate
converter designs, as well as the digital signal processing
techniques used in the current state-of-the-art. This
study leads to a proposed design of a prototype multi-channel
asynchronous sampling rate converter.

\vfill

\textbf{\Large Keywords:Asynchronous Sample Rate Converter, Digital Signal Processing, FPGA, Fractional Delay Filter} 
