%%%%%%%%%%%%%%%%%%%%%%%%%%%%%%%%%%%%%%%%%%%%%%%%%%%%%%%%%%%%%%%%%%%%%%%%
%                                                                      %
%     File: Thesis_Conclusions.tex                                     %
%     Tex Master: Thesis.tex                                           %
%                                                                      %
%     Author: Andre C. Marta                                           %
%     Last modified :  2 Jul 2015                                      %
%                                                                      %
%%%%%%%%%%%%%%%%%%%%%%%%%%%%%%%%%%%%%%%%%%%%%%%%%%%%%%%%%%%%%%%%%%%%%%%%

\chapter{Conclusions}
\label{chapter:conclusions}

In this report, a study of the most used audio Asynchronous Sample Rate
Conversion (ASRC) algorithm and a preliminary hardware implementation is
presented. The circuit is described in Verilog and will be prototyped in
FPGA. The objective is to design a multi-channel sample rate converter
Intellectual Property (IP) module which can be integrated in System on Chip. An
ASRC is a rather complex circuit and only major semiconductor players such as
Cyrrus Logic, Analog Devices and Texas Instruments have Integrated Circuit (IC)
solutions available in the market. To the best of our knowledge the only
existing ASRC IPs in the market, the CWda5x family, is made available by
Coreworks, SA. According to the CWda5x documentation, there is much room for
improvements, which are studied in detail in this thesis.

ASRC algorithms require high numeric precision to ensure the quality of the
audio signal, given the high sensitivity of the human auditory system. There are
two major sources of imprecision: (1) the precision of the time measurements
needed to determine the sampling instant; (2) the precision of the filter
coefficients, obtained from an ideal low pass filter impulse response (sinc
function), since they vary with the exact instant when the output sample is
computed.

Once the output sample issue time is determined, the computation of the output
sample may proceed synchronously. As a consequence, synchronous sample rate
converter IPs can easily be derived from the present work, since this type of
converter is a sub part of the a full ASRC solution. Synchronous sample rate
converters find many applications in digital audio to operate on stored signals
for which the sample rate is known \textit{a priori}.

The high precision demands of ASRC algorithms tends to undesirably increase the
hardware area occupation and computation time. The solution which will be further
explored in the following master's dissertation will need to compromise between
signal fidelity and resources usage.

% ----------------------------------------------------------------------
\section{Achievements}
\label{section:achievements}

A research about sample rate converters has been conducted, emphasizing the
distinction between synchronous and asynchronous converters. In fact synchronous
sample rate converters are a sub part of asynchronous sample rate converters in
the sense that they do not need to dynamically determine the sample rate of the
input signal, nor do they need to perform dynamic computation as they could
perform the calculation using predetermined filter coefficients. Note however,
that the number of fixed coefficient sets may be rather large if the repetition
period (in samples) of the sampling instant relative to the phase of the input
signal is large.

Asynchronous sample rate converters involve a more complex algorithm, using more
hardware resources to dynamically determine the output sample instants. However,
this type of converters is very useful as they can work on input signals for
which the sample rate is unknown. They are also able to compensate deviations on
the input or output clocks while maintaining the same hardware.

To fully understand the principle on which the algorithm is based, some basic
digital signal processing principles are reviewed, with emphasis on digital
signal filtering, as well as the implications of sampling and aliasing. This
study made it possible to derive the specifications of the digital resampling
filter in terms of its type, cutoff frequency and phase response.

To implement the synchronous converter in hardware, some filter structures were
analyzed, with emphasis on their advantages and disadvantages. While the Farrow
structure is the hardest to implement, it is also the most versatile, and the
one which may lead to a smaller total harmonic distortion plus noise ratio,
while maintaining a low area and computational cost.

The asynchronous nature of the algorithm dictates the need to work with multiple
clock domains, which is as complex as it can get in terms of digital hardware
design. In the chosen implementation 3 different clock domains are used: the
input, output and processing clock domains. Different types of hardware
synchronizers using registers and dual-port memories are employed.


% ----------------------------------------------------------------------
\section{Future Work}
\label{section:future}

The work to be developed in the thesis consists in creating a sample rate
converter design which is able to dynamically convert sample rates which may vary
over time. The core should occupy as less hardware resources as possible, and
have an audio fidelity on par with the current integrated circuit
implementations. Furthermore, it shall be possible to configure the core using
the AXI-Lite interface, as well as sending and receiving audio samples using a
parallel/I2S interface. With this implementation it should be possible to
implement the core in FPGAs and SoCs. The work plan is shown in
Table~\ref{tab:planning}, together with the scheduling for each task.

\begin{table}[!htbp]
	\centering
	\caption{Planning of the work that will be performed in the thesis}
	\label{tab:planning}
	\begin{tabular}{|p{8cm}|l|}
		\hline
		\textbf{Task}                                                                            & \textbf{Schedule} \\
		\hline
		Improve sample rate converter prototype, fulfilling specifications for main sample rates          & 17/02 - 15/03       \\
		\hline
		Implement ratio estimator into sample rate converter                                              & 16/03 - 12/04       \\
		\hline
		Insert pipelines and control units into core to increase its system clock frequency               & 13/04 - 19/04       \\
		\hline
		Optimize core to decrease FPGA resources usage                                                    & 20/04 - 03/05       \\
		\hline
		Design and implement interfaces                                                               & 04/05 - 10/05       \\
		\hline
		Test and debug core in FPGA                                                                       & 11/05 - 31/05       \\
		\hline
		Write the Thesis report                                                                           & 1/06 - 19/07       \\
		\hline
	\end{tabular}
\end{table}

The first step of this work is the improvement of the prototype, as it already
contains the fractional delay filter implementation, as well as most of the
modules defined in Chapter~\ref{chapter:circ_design}.  This step is considered
finished when the prototype is able to output a signal with a total harmonic
distortion and noise ratio equal or lower than $\SI{-130}{\dB}$ for all
combinations involving the commonly used sample rates outlined in
Table~\ref{tab:common_samp_rates}.


\begin{table}[!htbp]
	\centering
	\caption{Commonly used sample rates in audio applications and their particular uses}
	\label{tab:common_samp_rates}
	\begin{tabular}{|l|p{8cm}|}
		\hline
		\textbf{Sample Rate} & \textbf{Use} \\
		\hline
		$\SI{8.000}{\kilo\hertz}$ & Telephones/walkie-talkies.\\
		\hline
                $\SI{11.025}{\kilo\hertz}$ & Lower-quality PCM, MPEG audio and analizers of subwoofer bandpasses.\\
		\hline
                $\SI{16.000}{\kilo\hertz}$ & VoIP and VVoIP applications.\\
		\hline
                $\SI{22.050}{\kilo\hertz}$ & Lower-quality PCM, MPEG audio and low frequency energy analizers.\\
		\hline
                $\SI{32.000}{\kilo\hertz}$ & miniDV camcorders, and some video tapes. Also used for high-quality wireless microphones.\\
		\hline
                $\SI{44.100}{\kilo\hertz}$ & Audio CDs, most used with MPEG-1 audio (VCD, MP3) covers the audible bandwidth (up to $\SI{20}{\kilo\hertz}$.\\
		\hline
                $\SI{48.000}{\kilo\hertz}$ & Professional digital video equipment and consumer video formats, like digital TV, DVD and films.\\
		\hline
                $\SI{88.200}{\kilo\hertz}$ & Some professional recording equipment that targets CD, such as mixers or equalizers.\\
		\hline
                $\SI{96.000}{\kilo\hertz}$ & High definition DVD and blu-ray audio tracks.\\
		\hline
                $\SI{176.400}{\kilo\hertz}$ & High definition CD recorders and other applications targeting CD.\\
		\hline
                $\SI{192.000}{\kilo\hertz}$ & Professional video equipment targeting high definition DVD and blu-ray audio tracks.\\
		\hline
	\end{tabular}
\end{table}

With a prototype capable of fulfilling the output fidelity specification for a
synchronous ASRC, the next step consists in adding the ratio estimator, making
it possible to convert sample rates dynamically. This means supporting sample rate
changes over time, while maintaining the output fidelity. This step
should also include the optimization of this module to ensure that it stabilizes
as fast as possible. After this step, the prototype can be considered complete,
although poorly optimized.

With an inefficient prototype, the next two steps are focused on optimization.
Firstly, pipelines will be added, as well as other timing closure oriented
optimizations, in order to enlarge the range of the system clock (processing)
frequency as much as possible for maximum usage flexibility and capability to
support multiple channels.

The second optimization step consists in optimizing the FPGA resource usage,
which indirectly leads to a compact ASIC implementation. Firstly, the widths of
the intermediate buses will be reduced as much as possible. This will create a
trade-off with the output's harmonic distortion and noise ratio, since narrower
buses will raise the quantization errors. The $\SI{-130}{\dB}$ will be used as
a reference to limit the reduction in the width of the buses. Afterwards, some
further optimizations will be done, through resource sharing and Verilog code
optimization. These optimizations should be done without impacting the
algorithm's execution time, as computing less samples per output sample clock
cycle leads to supporting less audio channels.

After all optimizations are done, the core needs to be implemented and tested in
a FPGA. To properly conduct the tests, and also to facilitate the integration in
any system, standard interfaces will be designed and implemented, using the
AXI-Lite interface for control and the $I^2S$ interfaces for the audio. With the
standard interface in place, the core will be integrated in a System-on-Chip
containing a RISC-V processor.

With the sample rate converter fully implemented and integrated, the final step
is to conduct system-level tests using both test signals and audio files send
from a PC. After system test, the sample rate converter IP core will be deemed
complete, and its documentation including the thesis to be written will be
finalized.
