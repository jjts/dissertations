%%%%%%%%%%%%%%%%%%%%%%%%%%%%%%%%%%%%%%%%%%%%%%%%%%%%%%%%%%%%%%%%%%%%%%%%
%                                                                      %
%     File: Thesis_Abstract.tex                                        %
%     Tex Master: Thesis.tex                                           %
%                                                                      %
%     Author: Andre C. Marta                                           %
%     Last modified :  2 Jul 2015                                      %
%                                                                      %
%%%%%%%%%%%%%%%%%%%%%%%%%%%%%%%%%%%%%%%%%%%%%%%%%%%%%%%%%%%%%%%%%%%%%%%%

\section*{Abstract}

% Add entry in the table of contents as section
\addcontentsline{toc}{section}{Abstract}

CPUs are extremely complex systems which take several years to develop and need
extraordinary amounts of capital investment. For a long time, only large
companies could afford creating their own CPUs, which they could either
integrate in other more complex systems they build or license to other companies
who use them to develop their own systems. However, thanks to a large open
source community, it is also becoming possible for smaller companies to build
their own systems using free and high quality hardware and software components
that are typically available in repositories hosted in web-based platforms like
GitHub and Bitbucket. The largest initiative so far to develop an open source
processor and respective ecosystem is arguably the RISC-V Instruction Set
Architecture (ISA), whose ambition is to become the standard ISA for all
computing devices, from microcontrollers to supercomputers. This paper presents
a development environment to create open source Systems on Chip (SoCs) that use
the PicoRV32 RISC-V core architecture, while providing a base SoC called
\socname to exemplify the development process and to serve as a starting point
for future development of SoCs of the same kind.

\vfill

\textbf{\Large Keywords:} RISC-V, Development Environment, Open Source, SoC

