%%%%%%%%%%%%%%%%%%%%%%%%%%%%%%%%%%%%%%%%%%%%%%%%%%%%%%%%%%%%%%%%%%%%%%%%
%                                                                      %
%     File: Thesis_Results.tex                                         %
%     Tex Master: Thesis.tex                                           %
%                                                                      %
%     Author: Andre C. Marta                                           %
%     Last modified :  2 Jul 2015                                      %
%                                                                      %
%%%%%%%%%%%%%%%%%%%%%%%%%%%%%%%%%%%%%%%%%%%%%%%%%%%%%%%%%%%%%%%%%%%%%%%%

\chapter{Results}
\label{chapter:results}

\section{Toolchain installation}

The toolchain installation was simple because the PicoRV32' GitHub
repository~\cite{bib:picorv32} provides a Makefile script for that
effect. Different make commands are available so that it is possible to install
just the RV32I, RV32IC, RV32IM and RV32IMC ISAs. Some adjustments to the Linux
shell commands were made in order for them to work in CentOS 7 (the operating
system running in the host machine), such as using \textit{yum} instead of
\textit{apt-get}, which is used in some Linux distributions such as Debian and
Ubuntu but not in CentOS, which uses \textit{yum} for software package
management\footnote{CentOS uses \textit{yum} because it is derived from the Red
  Hat Enterprise Linux distribution.}.

The provided testbenches were ran using simple make test commands (check the
README file in ~\cite{bib:picorv32} for details) and the results were
correct. As such, the RV32IMC toolchain was successfully installed in the host
computer running the CentOS 7 distribution of Linux.

\section{Synthesis of a simple PicoRV32 CPU + memory system}
%successful FPGA deployment of simple CPU + memory system (quartus script)
%tabela recursos
%tempos
%
%...
The simple CPU + memory system in folder scripts/quartus of the PicoRV32's
GitHub repository was successfully synthesized in an Intel's Cyclone V GT
FPGA device. The clock period used was 10 ns, which means the clock frequency was 100
MHz. The README file states that the maximum clock frequency is between 250 MHz
and 400 MHz on 7-Series Xilinx FPGAs.

The PicoRV32 core used was a pure RV32I core. The FPGA's resource usage summary
stated that this PicoRV32 core used:
\begin{itemize}
	\item 1103 combinational ALUTs;
	\item 648 dedicated logic registers;
	\item 2048 block memory bits.
\end{itemize}

The README file in the PicoRV32' GitHub repository states that a
regular-sized\footnote{See the bottom of the README file in~\cite{bib:picorv32}
  for the definition of small, regular and large-sized PicoRV32 core.} PicoRV32
core (which is the type of core used in the system synthesized) for a Xilinx
7-series Kintex device uses:

\begin{itemize}
	\item 917 slice LUTs;
	\item 583 slice registers;
	\item 48 LUTs as memory.
\end{itemize}

Although the comparison between Intel and Xilinx resources is not
straightforward, these results show that the number of LUTs and registers is
close enough, which validates the information provided in the repository. PicoRV32 is,
indeed, a small RISC-V CPU core that can be used in a SoC.
