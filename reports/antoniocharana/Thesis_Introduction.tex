%%%%%%%%%%%%%%%%%%%%%%%%%%%%%%%%%%%%%%%%%%%%%%%%%%%%%%%%%%%%%%%%%%%%%%%%
%                                                                      %
%     File: Thesis_Introduction.tex                                    %
%     Tex Master: Thesis.tex                                           %
%                                                                      %
%     Author: Andre C. Marta                                           %
%     Last modified :  2 Jul 2015                                      %
%                                                                      %
%%%%%%%%%%%%%%%%%%%%%%%%%%%%%%%%%%%%%%%%%%%%%%%%%%%%%%%%%%%%%%%%%%%%%%%%

\chapter{Introduction}
\label{chapter:introduction}


%%%%%%%%%%%%%%%%%%%%%%%%%%%%%%%%%%%%%%%%%%%%%%%%%%%%%%%%%%%%%%%%%%%%%%%%
\section{Motivation}
\label{section:motivation}

With the advent of portable devices such as smartphones, tablets and others,
there has been a high demand for low power CPUs, in order to meet stringent
battery life specifications. As the world market of portables devices grows and
the Internet of Things (IoT) emerges, hardware developers need to focus more and more
on efficient low power processor cores to be integrated in SoCs that have very
narrow specifications for power, area and performance.

Having just one or two providers of CPU cores, such as ARM, cannot satisfy the
enormous appetite of many smaller businesses and consultancies who may and
effectively can provide services in this domain. Therefore, open source
processor cores are the way to go since the CPU is one of the most complex parts
of the system but is also a commodity which, by itself, adds little value to the
final product. In this context, after previous attempts such as the OpenRISC
experience, the RISC-V architecture finally emerged and promises to become for
hardware what the Linux operating system is for software. The Internet of Things
is a promising market for the future and is one of the main targets of many
innovative startups that use open source hardware and software components for
development. SoCs are typically embedded in larger systems, so companies
nowadays tend to specialize their development in very particular systems and/or
applications, which are then licensed to other companies as intellectual
property (IP).


%%%%%%%%%%%%%%%%%%%%%%%%%%%%%%%%%%%%%%%%%%%%%%%%%%%%%%%%%%%%%%%%%%%%%%%%
\section{Problem}
\label{section:problem}

Having access to a complete enough set of hardware and software building blocks
to build useful systems is either too expensive or time consuming, which is a
huge barrier for startups to initiate their business. In the prequel of the IoT
era, the semiconductor IP market must change in order to allow smaller startups
to have some share of it. The entry of such new companies in this market is
imperative because the size of the future IoT market will simply be too big for
only a few large companies in the world to satisfy the brutal global demand on
extremely dedicated and complex IoT systems.

However, this also means that the IoT market will be extremely competitive. For
a company to survive in it, it will need to be able to produce high quality
hardware and software systems in a short window of time and budget, which may
seem a difficult task for small startups.


%%%%%%%%%%%%%%%%%%%%%%%%%%%%%%%%%%%%%%%%%%%%%%%%%%%%%%%%%%%%%%%%%%%%%%%%
\section{Solution}
\label{section:solution}

Therefore, for the IoT revolution to finally trigger off, it is indispensable
that an Instruction Set Architecture (ISA) free of intellectual property and
licensing fees is made available to companies and their engineers, so that they
can fully unleash their creativity in designing innovative systems without
having to worry about such legal aspects of the semiconductor IP market and
consuming the scarce funds they have available~\cite{bib:isafree}. Such an ISA
is provided by the RISC-V initiative.

Hence, the creation of a development environment for SoCs using RISC-V cores is
an asset in a such a company, as it saves time wasted in developing these
extremely dedicated systems which would otherwise take too much time to develop
if an open source CPU was not made available in the first place. This way, the
majority of the company's resources can be focused on developing higher level
and innovative systems for the particular application they commit to work on,
without wasting time and money creating a processor core from scratch or paying
companies like ARM to license them their CPUs.

Previous attempts to create SoCs along with their respective development
environments have been made by the same research team that currently is hosting 
the author and that has spun off company IObundle, Lda, also a stakeholder in this 
project. The first approach was made in~\cite{bib:blackbird}, using an OpenRISC open 
source processor to build an SoC called Blackbird and a programming environment with
the purpose of building reconfigurable systems based on it. However, this
OpenRISC approach was eventually put aside due to the lack of some essential
parts of the system, which made the team realize that it was not worth the
effort.

Recently, a new attempt was made in~\cite{bib:warpbird}, which introduced Warpbird,
whose ambition was to become one of the first untethered SoCs built with open
source components. Realizing the limitations of the OpenRISC initiative, the
team opted to use the open source RISC-V toolchain. Concretely, they used the
Chisel3 Hardware Description Language (HDL)~\cite{bib:chisel3}, allowing
a higher-level description of complex systems, and the Rocket Chip
generator~\cite{bib:rocketchip}, which converts the Chisel3 description of a
system to Verilog code, which in turn is converted to a C++ or SystemC
equivalent cycle-accurate behavioral model using Verilator~\cite{bib:verilator},
useful for simulation purposes. The use of the Rocket (Chip) framework reduced
the effort of building a RISC-V based SoC significantly. However, the project
budget ran out before its conclusion because the fundamental components of the
SoC, after being synthesized and implemented down to bitstream, faced some
unexpected problems in the FPGA emulation phase. There were also some
peripherals that were still in development which ultimately remained unfinished.

The work to be done and described in this document can be considered a
continuation of what was done in~\cite{bib:blackbird} and~\cite{bib:warpbird}. A
new solution using a different RISC-V core architecture will be considered, as
well as an Ethernet interface instead of a JTAG module in order to speed up
program data transfers from the host computer to the base SoC, which will be
named \socname for future reference in this document.


%%%%%%%%%%%%%%%%%%%%%%%%%%%%%%%%%%%%%%%%%%%%%%%%%%%%%%%%%%%%%%%%%%%%%%%%
\section{Objectives}
\label{section:objectives}

This project's main objective is to create an environment which allows a company
to fully develop SoCs using RISC-V processor cores and other open source
components in a reasonable time frame, such that it does not have to license
these IP from third parties, and is more able to develop their systems in time.

In order to achieve this goal, the first step to take is to build the \socname
System on Chip using the size-optimized RISC-V CPU known as
PicoRV32~\cite{bib:picorv32}. Afterwards, it will be necessary to understand how
to add, remove and/or modify hardware and software components and develop an
environment which allows this process to be made with the maximum ease possible,
so to assure that developers can work fast. Then, a test environment is to be
built so that the components added to the SoC can be properly tested with RTL
simulation and FPGA emulation. Finally, both these environments should be
integrated in a single development environment that comprises the entire flow of
the SoC's building process.


%%%%%%%%%%%%%%%%%%%%%%%%%%%%%%%%%%%%%%%%%%%%%%%%%%%%%%%%%%%%%%%%%%%%%%%%
\section{Author's work}
\label{section:authorswork}


The work to be done during the master thesis will be the continuation of the
works presented in~\cite{bib:blackbird,bib:warpbird}, which were carried out at
IObundle, Lda, a Portuguese company based in Lisbon and spun off by the INESC-ID
research institute, and that designs semiconductor IP. The author started a
professional internship at IObundle in October 2018, where he has done research
on the RISC-V ISA and open source RISC-V processor cores that will be useful to
build the \socname System on Chip and the development environment during
the dissertation project. The author aided the Versat~\cite{bib:versat} team at
IObundle to attach a UART core to the Versat architecture, managed to install
the toolchain necessary to use the PicoRV32 RISC-V processor architecture
~\cite{bib:picorv32} and synthesized a simple CPU + memory system using a
PicoRV32 core to have an early estimate of its size, based on the FPGA resources
usage report.



%%%%%%%%%%%%%%%%%%%%%%%%%%%%%%%%%%%%%%%%%%%%%%%%%%%%%%%%%%%%%%%%%%%%%%%%
\section{Outline}
\label{section:outline}

This document has four more chapters besides this introductory one. A brief
description of the topics addressed in each of these chapters follows.

\begin{itemize}
\item Chapter~\ref{chapter:socdesign} consists on a research regarding processor
  architectures and SoCs. A comparison between Complex Instruction Set Computer
  (CISC) and Reduced Instruction Set Computer (RISC) architectures is presented,
  highlighting the main differences between the two within the scope of SoC
  development and showing why RISC architectures are preferred for the goals of
  this project. A description of the RISC-V ISA and its main extensions is
  provided, while explaining which ones should be used for SoC development
  and why. Next, some state-of-the-art methods for building SoCs are referred
  and the \socname System on Chip is described. Its components' roles in the SoC
  are explained.

\item Chapter~\ref{chapter:environment} will cover the development environment
  to be built. First, the verification environment is presented and its novelties
  in respect to~\cite{bib:warpbird} are highlighted. Next, the several parts of the 
  SoC development flow are identified and described. Throughout this description, 
  it is referred how the toolchain is installed on the host machine, which of its 
  parts are used in each development phase and how they are used. Next, the SoC test 
  and verification environments developed in the previous iterations of the project
  at hand~\cite{bib:blackbird,bib:warpbird} are analyzed. A base concept of the
  development environment to be built is presented, based on \socname's
  structure and interfaces.

\item Chapter~\ref{chapter:results} demonstrates some preliminary results of the
  synthesis of a simple CPU + memory system in PicoRV32's GitHub
  repository~\cite{bib:picorv32}.

\item Chapter~\ref{chapter:conclusions} contains some conclusions of the
  research described throughout the remaining chapters of this document. A
  summary of the achievements made with this research is made and some future
  work is outlined. The planning of the tasks to be carried out in the dissertation
  project is also presented.
\end{itemize}

