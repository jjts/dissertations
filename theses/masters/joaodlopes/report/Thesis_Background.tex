%%%%%%%%%%%%%%%%%%%%%%%%%%%%%%%%%%%%%%%%%%%%%%%%%%%%%%%%%%%%%%%%%%%%%%%%
%                                                                      %
%     File: Thesis_Background.tex                                      %
%     Tex Master: Thesis.tex                                           %
%                                                                      %
%     Author: João D. Lopes                                            %
%     Last modified :  1 October 2017                                  %
%                                                                      %
%%%%%%%%%%%%%%%%%%%%%%%%%%%%%%%%%%%%%%%%%%%%%%%%%%%%%%%%%%%%%%%%%%%%%%%%

\chapter{Background}
\label{chapter:background}

% CGRAs on industry (add CGRA references)

CGRAs have gained increasing attention in the last two decades both in
academia and industry~\cite{Lee00,Mei05,Weinhardt03,Quax04,deSousa12}.
They are programmable hardware mesh structures that operate on
word-length variables and are efficient in terms of operations per
unit of silicon area. CGRAs nodes are generally ALUs with high-level
operations such as addition, subtraction, multiplication, shifting,
etc.~\cite{Tripp07}. The CGRA is a suitable architecture for low power
systems. In this chapter, the relevant CGRA literature is
reviewed. The chapter is organized in the following topics:
interconnect topology, reconfiguration, address generation,
heterogeneity versus homogeneity and compiler.


%%%%%%%%%%%%%%%%%%%%%%%%%%%%%%%%%%%%%%%%%%%%%%%%%%%%%%%%%%%%%%%%%%%%%%%%
\section{Interconnect topology}

An important aspect of CGRAs is the interconnect
topology~\cite{Park09}. Fully connected graph topologies have been
avoided as they scale poorly in terms of area, wire delays and power
consumption. Since large arrays have been preferred over smaller
arrays, it has been important to keep a lean interconnection
structure. However, in this work a relatively smaller array is used
which permits the use a fully connected topology, as the area,
frequency and power results demonstrate.




%%%%%%%%%%%%%%%%%%%%%%%%%%%%%%%%%%%%%%%%%%%%%%%%%%%%%%%%%%%%%%%%%%%%%%%%
\section{Reconfiguration}
\label{section:reconfiguration}


%dynamic vs static
A critical aspect for achieving CGRA programmability is the dynamic
reconfiguration of the array. Static reconfiguration, where the array
is configured once to run a complete kernel, has poor
flexibility~\cite{Hartenstein01}. Some arrays are dynamically
reconfigurable but they only iterate over a fixed sequence of
configurations~\cite{Lee00,Mei05,Quax04}. These configurations must be
moved to the CGRA from an external memory and stored inside the CGRA,
which costs memory bandwidth and internal storage space. Following a
fixed sequence of configurations means that it is impossible to
conditionally choose the next configuration. Furthermore, in many
cases, it is the host system that manages the
reconfiguration~\cite{Lee00,Mei05,Hartenstein99}. This is innefficient
as the host could be doing more useful tasks. It also makes
programming and integrating CGRAs in an embedded system difficult.

% full vs partial
In order to make the reconfiguration process efficient, full
reconfiguration of the array should be avoided. In this work we
exploit the similarity of different CGRA configurations by using {\em
  partial reconfiguration}. If only a few configuration bits differ
between two configurations, then only those bits are changed. Most
CGRAs are only fully reconfigurable~\cite{Lee00,Mei05,Hartenstein99}
and do not support partial reconfiguration. The disadvantage of
performing full reconfiguration is the amount of configuration data
that must be kept and/or fetched from external memory. Previous CGRA
architectures with support for partial reconfiguration include
RaPiD~\cite{Ebeling96}, PACT~\cite{Weinhardt03} and
RPU~\cite{Liu15}. RaPiD supports dynamic (cycle by cycle) partial
reconfiguration for a subset of the configuration bitstream, which
implies that the loop body may take several cycles to execute. The
partial reconfiguration process in PACT is reportedly slow and users
are recommended to avoid it and resort to full reconfiguration
whenever possible. In RPU, a kind of partial reconfiguration called
Hierarchical Configuration Context is proposed to mitigate these
problems.  Data to make performance comparisons with these approaches
have not been found but, compared to~\cite{Ebeling96}, the partial
reconfiguration in this work happens between program loops instead of
every clock cycle, and the whole loop body executes in only one
cycle. Compared to~\cite{Weinhardt03}, this partial reconfiguration is
fast and used more frequently.

Moreover, in this approach the configurations are {\em
  sef-generated}. This way, the host does not have to manage the
reconfiguration process and is free to perform more useful tasks.


%%%%%%%%%%%%%%%%%%%%%%%%%%%%%%%%%%%%%%%%%%%%%%%%%%%%%%%%%%%%%%%%%%%%%%%%
\section{Address Generation}
\label{section:addressGenerator}

Address generation is a well known topic. CPU architectures use
Address Generation Units (AGUs) to compute memory addresses, in order
to improve
performance~\cite{van2006address,wijeratne20079,cormie2002arm11}.
This computation is associated with load/store instructions.

The main contribution in~\cite{deSousa12} was the invention of an
address generation scheme able to support groups of nested loops
expressed in a single CGRA configuration. The idea, was to reduce the
reconfiguration time and was inspired by the use of cascaded counters
for address generation~\cite{Carta06}. This represented a major
improvement compared to other works that focus exclusively in
accelerating the inner loops of compute kernels~\cite{deSutter10}.
However, as this thesis shows, more can be done in terms of address
generation for reducing the reconfiguration overhead.


%%%%%%%%%%%%%%%%%%%%%%%%%%%%%%%%%%%%%%%%%%%%%%%%%%%%%%%%%%%%%%%%%%%%%%%%
\section{Heterogeneity versus Homogeneity}
\label{section:heterogeneityVsHomogeneity}

The question of CGRA heterogeneity versus homogeneity in terms of the
functional units is an important one. Some CGRAs are
homogeneous~\cite{Ebeling96}, i.e., they only have one type of
functional unit, whereas others are heterogeneous and support a
diversity of functional units~\cite{Heysters03}. A careful analysis
done in~\cite{Park12} has favored heterogeneous CGRAs, arguing that
the performance degradation when going from a homogeneous to a
heterogeneous architecture is greatly compensated by the better
silicon area utilization and power efficiency of heterogeneous
solutions. Hence, heterogeneous CGRAs have been adopted in this
project.


%%%%%%%%%%%%%%%%%%%%%%%%%%%%%%%%%%%%%%%%%%%%%%%%%%%%%%%%%%%%%%%%%%%%%%%%
\section{Compiler}
\label{section:compiler}

Compiler support is probably the most difficult aspect of CGRAs. CGRA
compilers make use of standard compilation techniques, especially the
well-known modulo scheduling approach used in VLIW
machines~\cite{Rau94}, and of more exotic techniques like the place
and route algorithms used in FPGA compilation~\cite{Betz99}.

One attempt to circumvent the compiler difficulties is to formulate
CGRAs as vector processors~\cite{Severance13,Severance14,Naylor14}. In
those approaches, instructions are the equivalent of small
configurations, and their authors claim several orders of magnitude
speedup in certain applications. However, the user has to work at a
very low level to make use of vector instructions. 

A new compiler for Versat has been
developed~\cite{Santiago2016}. Before, the use of standard compilers
such as {\em gcc} or {\em llvm} for Versat has been
investigated. However, it has been concluded that these compilers are
good at producing sequences of instructions but not sequences of
hardware datapaths. For this reason, it has been decided that a
specific compiler was needed. The developed compiler is simple as it
focuses on the tasks done in Versat's reconfigurable array while
performing little or no optimizations on the sequence of instructions
run by Versat's controller. There was no need to use complex {\em
  place \& route} algorithms thanks to Versat's full mesh
topology. The syntax of the Versat programming language is a subset of
the C/C++ language with a semantics that enables the description of
hardware datapaths. The compiler is not described in this thesis whose
main thrust is the description of the architecture and its
Very-Large-Scale Integration (VLSI) implementation.

