%%%%%%%%%%%%%%%%%%%%%%%%%%%%%%%%%%%%%%%%%%%%%%%%%%%%%%%%%%%%%%%%%%%%%%%%
%                                                                      %
%     File: Thesis_Glossary.tex                                        %
%     Tex Master: Thesis.tex                                           %
%                                                                      %
%     Author: João D. Lopes                                            %
%     Last modified : 4 June 2016                                      %
%                                                                      %
%%%%%%%%%%%%%%%%%%%%%%%%%%%%%%%%%%%%%%%%%%%%%%%%%%%%%%%%%%%%%%%%%%%%%%%%
%
% The definitions can be placed anywhere in the document body
% and their order is sorted by <symbol> automatically when
% calling makeindex in the makefile
%
% The \glossary command has the following syntax:
%
% \glossary{entry}
%
% The \nomenclature command has the following syntax:
%
% \nomenclature[<prefix>]{<symbol>}{<description>}
%
% where <prefix> is used for fine tuning the sort order,
% <symbol> is the symbol to be described, and <description> is
% the actual description.

% ----------------------------------------------------------------------

% \glossary{name={\textbf{MOSI}},description={Master Output Slave Input, saida no mestre entrada no escravo .}}

%\glossary{name={\textbf{MDO}},description={Multi-Disciplinar Optimization is an engineering technique that uses optimization methods to solve design problems incorporating two or more disciplines.}}

%\glossary{name={\textbf{CFD}},description={Computational Fluid Dynamics is a branch of fluid mechanics that uses numerical methods and algorithms to solve problems that involve fluid flows.}}

%\glossary{name={\textbf{CSM}},description={Computational Structural Mechanics is a branch of structure mechanics that uses numerical methods and algorithms to perform the analysis of structures and its components.}}

%\DeclareAcronym{<ID>}{
%  short = <short> ,
%  long  = <long> ,
%  class = <class>
%}


%\printacronyms[include-classes=abbrev,name=Abbreviations]

%\printacronyms[include-classes=nomencl,name=Nomenclature]

%\glossary{name={\textbf{CGRA}},description={Coarse-Grain Reconfigurable Array is a hardware accelerator that is used to improve data computation in an embedded system.}}

%\glossary{name={\textbf{IoT}},description={Internet of Things is the network of physical objects (devices, vehicles, etc) embedded with electronics, software, sensors, and network connectivity that enables these objects to collect and exchange data.}}

%\glossary{name={\textbf{FPGA}},description={Field-Programmable Gate Array is an integrated circuit designed to be configured by a customer or a designer after manufacturing.}}

%\glossary{name={\textbf{CPU}},description={Central Processing Unit is the electronic circuitry within a computer that carries out the instructions of a computer program by performing the basic arithmetic, logical, control and input/output (I/O) operations specified by the instructions.}}

%\glossary{name={\textbf{RISC}},description={Reduced Instruction Set Computing is a CPU design strategy based on the insight that a simplified instruction set provides higher performance when combined with a microprocessor architecture capable of executing those instructions using fewer microprocessor cycles per instruction.}}

%\glossary{name={\textbf{VLIW}},description={Very Long Instruction Word refers to processor architectures designed to exploit instruction level parallelism (ILP).}}

%\glossary{name={\textbf{API}},description={Application Programming Interface is a set of routines, protocols, and tools for building software and applications.}}

%\glossary{name={\textbf{OpenCL}},description={Open Computing Language is a framework for writing programs that execute across heterogeneous platforms consisting of central processing units (CPUs), graphics processing units, digital signal processors, field-programmable gate arrays and other processors or hardware accelerators.}}

%\glossary{name={\textbf{SoC}},description={A System on Chip is an integrated circuit that integrates all components of a computer or other electronic system into a single chip.}

%\glossary{name={\textbf{IP}},description={Intellectual Property refers to creations of the intellect for which a monopoly is assigned to designated owners by law.}}

%\glossary{name={\textbf{AXI4}},description={Advanced Extensible Interface (version 4) is an interface designed by ARM, which derives from the Advanced Microcontroller Bus Architecture (AMBA).}}

%\glossary{name={\textbf{ASIC}},description={Application-Specific Integrated Circuit is an integrated circuit customized for a particular use, rather than intended for general-purpose use.}}

%\glossary{name={\textbf{CAD}},description={Computer-Aided Drafting is the use of computer systems to aid in the creation, modification, analysis, or optimization of a design.}}

%\glossary{name={\textbf{GPU}},description={Graphics Processing Unit, also occasionally called Visual Processing Unit (VPU), is a specialized electronic circuit designed to rapidly manipulate and alter memory to accelerate the creation of images in a frame buffer intended for output to a display.}}

%\glossary{name={\textbf{VLSI}},description={Very-Large-Scale Integration is the process of creating an integrated circuit by combining thousands of transistors into a single chip.}}

\begin{acronym}

\acro{WSN}{Wireless Sensor Network}
\acro{GPU}{Graphics Processor Unit}
\acro{FPGA}{Filed Programmable Gate Array}
\acro{CGRA}{Coarse Grained Reconfigurable Array}
\acro{CPU}{Central Processing Unit}
\acro{RISC}{Reduced Instruction Set Computer}
\acro{VLIW}{Very Large Instruction Word}
\acro{IC}{Integrated Circuit}
\acro{API}{Application Programming Interface}
\acro{SoC}{System on Chip}
\acro{IP}{Intellectual Property}
\acro{FFT}{Fast Fourier Transform}
\acro{GPP}{General Prurpose Processor}
\acro{ASIC}{Application-Specific Integrated Circuit}
\acro{VLSI}{Very-Large-Scale Integration}

\acro{AXI}{Advanced Extensible Interface}
\acro{SPI}{Serial Peripheral Interface}
\acro{DE}{Data Engine}
\acro{FU}{Functional Unit}
\acro{TLP}{Thread-Level Parallelism}
\acro{DLP}{Data-Level Parallelism}
\acro{ILP}{Instruction-Level Parallelism}
\acro{AGU}{Address Generation Unit}
\acro{ALU}{Arithmetic and Logic Unit}
\acro{CM}{Configuration Module}
\acro{DMA}{Direct Memory Access}
\acro{CRF}{Control Register File}



\end{acronym}

%\DeclareAcronym{CGRA}{
%  short = CGRA ,
%  long  = Coarse Grained Reconfigurable Array
%}
