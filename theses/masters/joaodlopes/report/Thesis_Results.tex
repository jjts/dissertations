%%%%%%%%%%%%%%%%%%%%%%%%%%%%%%%%%%%%%%%%%%%%%%%%%%%%%%%%%%%%%%%%%%%%%%%%
%                                                                      %
%     File: Thesis_Results.tex                                         %
%     Tex Master: Thesis.tex                                           %
%                                                                      %
%     Author: João D. Lopes                                            %
%     Last modified :  9 September 2017                                %
%                                                                      %
%%%%%%%%%%%%%%%%%%%%%%%%%%%%%%%%%%%%%%%%%%%%%%%%%%%%%%%%%%%%%%%%%%%%%%%%

\chapter{Results}
\label{chapter:results}

In this chapter, experimental results for the Versat architecture are
presented and discussed. First, the implementation results for Versat
in FPGA, as well as in ASIC technology are presented. Then,
performance and energy consumption results, both for simple and more
complex computational kernels, are given. To end the chapter, Versat
is compared to other CGRAs.

% ----------------------------------------------------------------------
\section{FPGA implementation results}
\label{section:FPGAresults}

The FPGA implementation results for Versat are given in
Table~\ref{tabFPGAr}. These results show that in terms of size, Versat
is able to be implemented in the smallest FPGAs: in a Xilinx XC5VLX50T
device of the Virtex V family (the one that is present on the Xilinx
ML505 board), Versat occupies half of the logic resources; in an
Altera Cyclone IV EP4CE22F17C6N device (present on Altera DE0 Nano
board), Versat takes about $80\%$ of the logic resources. The Xilinx
implementation makes better use of the RAM resources for implementing
the configuration memory than the Altera device. This explains why the
Xilinx implementation consumes more RAM memory but less logic (LookUp
Tables or LUTs), while the Altera device consumes more Logic Elements
(LEs) and less RAM. This is also reflected in the frequency of
operation, although the fact that two different architectures are
being compared is also important; the Virtex V architecture can
generally achieve higher frequencies of operation compared to the
Cyclone IV architecture. Nevertheless, it is a fact that the current
Versat implementation is not very well optimized in terms of the
maximum frequency of operation, but more work can be done on this once
it becomes a priority.

\begin{table}[!htb]
  \renewcommand{\arraystretch}{1.2} % more space between rows
  \caption{FPGA implementation results.}
  \label{tabFPGAr}
  \centering
  \begin{tabular}{lccccc}
    \toprule
    Architecture &      Logic & Regs & RAM(kB) &        Mults & Fmax(MHz)\\
    \midrule
    Cyclone IV   & 19366  LEs & 4673 &   43.88 & 32  (9 bits) &  64\\
    Virtex V     & 12510 LUTs & 4396 &  132.75 & 16 (18 bits) & 102\\
    \bottomrule
  \end{tabular}
\end{table}

% ----------------------------------------------------------------------
\section{ASIC implementation results}
\label{section:ASICresults}

Versat has been designed using a UMC $130nm$
process. Table~\ref{tabASICr} compares Versat with a state-of-the-art
embedded processor (ARM Cortex A9) and two other CGRA implementations
(Morphosys~\cite{Lee00} and ADRES~\cite{Mei05}). The cores are
compared in terms of the technology node (Node), silicon area (Area),
embedded memory (RAM), frequency of operation (Frequency) and power
consumption (Power). The Versat frequency and power results have been
obtained using the Cadence IC design tools, and the node activity rate
extracted from simulating an FFT kernel.

\begin{table}[!htb]
  \renewcommand{\arraystretch}{1.2} % more space between rows
  \caption{ASIC implementation results.}
  \label{tabASICr}
  \centering
  \begin{tabular}{lccccc}
    \toprule
    Core & Node(nm) & Area(mm\textsuperscript{2}) & RAM(kB) &  Freq.(MHz) & Power(mW)\\
    \midrule
    ARM Cortex A9~\cite{wang} &  40 & 4.6 & 65.54 & 800 &  500 \\
    Morphosys~\cite{Lee00}    & 350 & 168 &  6.14 & 100 & 7000 \\
    ADRES~\cite{Mei05}        &  90 &   4 & 65.54 & 300 &   91 \\
    Versat                    & 130 & 5.2 & 46.34 & 170 &  132 \\
    \bottomrule
  \end{tabular}
\end{table}

Because the different designs use different technology nodes, it is
difficult to compare the results in Table~\ref{tabASICr}. In order to
facilitate the comparisons, the results are scaled to the $40nm$
technology node and presented in Table~\ref{tabASICrs}. The scaling is
performed as explained in~\cite{borkar99}.

\begin{table}[!htb]
  \renewcommand{\arraystretch}{1.2} % more space between rows
  \caption{ASIC implementation results scaled to $40nm$.}
  \label{tabASICrs}
  \centering
  \begin{tabular}{lcccc}
    \toprule
    Core & Area(mm\textsuperscript{2}) & RAM(kB) &  Freq.(MHz) & Power(mW)\\
    \midrule
    ARM Cortex A9~\cite{wang} &  4.6 & 65.54 & 800 & 500 \\
    Morphosys~\cite{Lee00}    & 2.19 &  6.14 & 875 & 800 \\
    ADRES~\cite{Mei05}        & 0.79 & 65.54 & 675 &  40 \\
    Versat                    & 0.49 & 46.34 & 553 &  41 \\
    \bottomrule
  \end{tabular}
\end{table}

The results show that Versat is the smallest core, and compared with
the ARM processor it is $9\times$ smaller. The ADRES architecture is
about twice the size of Versat and Morphosys is much larger, about
half the size of the ARM processor. These differences can be explained
by the different capabilities of these cores. While Versat has a
16-instruction controller and 11 FUs (excluding the memory units),
ADRES has a VLIW processor and a 4x4 FU array, and Morphosys has a
RISC processor and an 8x8 FU array.

In terms of embedded memory, Versat uses somewhat less memory than the
ARM Cortex A9 or ADRES cores, but its memory size ($46kB$) can be
considered typical for an embedded processor. Morphosys uses a lot
less memory, as it is designed to focus more on processing power and
less on storage capabilities.

The ARM core operation frequency can be considered low as this is a
power optimized version. Other versions of this core operating at
higher frequencies exist, but the area footprint is larger and the
power consumption is higher. Those versions are optimized for
performance rather than power. Among the three CGRAs, Versat is the
least optimized in terms of the working frequency. In fact, not too
much effort has been put into achieving timing closure for a higher
frequency. Notwithstanding, after analyzing the critical paths, it
became clear that there is plenty of room for optimization, so its
frequency can be considered comparable to the other CGRAs.

As far as power is concerned, Morphosys consumes more than the ARM
core. Again, this is the result of focusing in performance with a
large array of FUs. The ADRES architecture seems well optimized for
power, in spite of its cycle by cycle and progressive reconfiguration
scheme. However, the acceleration that can be achieved with ADRES is
not clearly documented in its publications~\cite{Mei05}. Versat
consumes about the same power as the ADRES core, but there is also
room for improvement, given the little effort spent so far in
low-level power optimization.

% ----------------------------------------------------------------------
\section{Execution results}
\label{section:executionResults}

In this section, Versat is compared with a state-of-the-art embedded
processor in terms of performance and energy comsumption by running a
set of example kernels on Versat and on the embedded processor. The
results are divided in two parts: simple and complex kernels. In the
simple kernels part, all tests operate on vector sizes of 1024, while
in complex kernels part, larger data sets are used. All data is in
32-bit fixed-point format. A hardware timer has been used to measure
the time in elapsed clock cycles. To end this section, results that
show why self-generated configurations are better than pre-compiled
stored configurations are presented.

In order to assess the performance of the Versat architecture, the
Zybo Zynq-7000 ARM/FPGA SoC development board, featuring a Xilinx Zynq
7010 FPGA and a dual-core embedded ARM Cortex A9 system, has been
used. Versat is connected as a peripheral to the ARM system using its
AXI4 slave interface. The ARM system comprises a memory controller for
accessing an external DDR3 module. Versat can also access this memory
controller by connecting its AXI4 master interface to an appropriate
AXI4 slave interface on the ARM system. The Zybo development board has
been used only to measure the number of clock cycles for executing
each kernel. The speedup was estimated using the following equation:

\begin{equation}
Speedup = \frac{t_{ARM}\times f_{Versat}}{t_{Versat}\times f_{ARM}},
\label{eq:speedup}
\end{equation}
where $t_{ARM}$ and $t_{Versat}$ are the execution cycle count for the
ARM and Versat, repectively, and $f_{ARM}$ and $f_{Versat}$ are their
clock frequency in the $40nm$ process, according to
table~\ref{tabASICrs}. The energy ratio was estimated by multiplying
their execution time by their respective power consumption figure also
given in table~\ref{tabASICrs}:

\begin{equation}
Energy~Ratio = \frac{P_{ARM}}{P_{Versat}} \times Speedup.
\label{eq:energyRatio}
\end{equation}

% ----------------------------------------------------------------------
\subsection{Performance and energy consumption results for simple kernels}
\label{subsection:ResultsSimpleKernels}

Results for the set of simple kernels are summarized in
Table~\ref{tabExecR}. These kernels use a single Versat configuration
(no reconfiguration or reuse of data already in the accelerator), in
order to get base values for performance and energy consumption. In
the next subsection, it will be shown that with massive reconfigurations
and data reuse, performance and energy use improve significantly.

The results compare the performance of the Versat core with the
performance of the ARM Cortex A9 core. The kernels are the following:
{\tt vadd} is a vector addition, {\tt iir1} and {\tt iir2} are $1^{st}$
and $2^{nd}$ order IIR filters and {\tt cdp} is a dot (inner) product of
two complex vectors. All kernels operate on Q1.31 fixed-point data
with vector sizes of 1024.

For both systems, the program has been placed in on-chip memory and
the data in the external DDR3 memory device. The {\em ARM} column
denotes the execution cycle count for the ARM core. The {\em Versat}
column gives the total cycle count for the Versat core, including data
transfer, processing, control and reconfiguration. The {\em Control}
column gives the unhidden control and reconfiguration cycles, that is,
the number of these cycles that do not occur in parallel with the
execution of the DE or DMA. The number of FUs used (column {\em
  \#FUs}) and the code size in bytes (column {\em Code Size}) are also
given for each kernel. The speedup and energy ratio have been obtained
assuming the ARM and the Versat cores are running at the frequencies
and power figures given in Table~\ref{tabASICrs} for the $40nm$
technology node (equations~\ref{eq:speedup}
and~\ref{eq:energyRatio}). The speedup (column {\em Speedup}) is the
ARM/Versat ratio of execution times. (The execution time is given by
the cycle count divided by the frequency of operation.) The energy
ratio (column {\em Energy Ratio}) is the energy spent by the ARM
processor divided by the energy spent by the Versat core. The consumed
energy is given by the execution time multiplied by the respective
power consumption figure.

\begin{table}[!htb]
  \renewcommand{\arraystretch}{1.2} % more space between rows
  \caption{Simple kernel results.}
  \label{tabExecR}
  \centering
  \begin{tabular}{lccccccc}
    \toprule
    Kernel & ARM & Versat & Control & \#FUs & Code Size & Speedup & Energy Ratio\\
    \midrule
    {\tt vadd} & 14726 &  4517 & 36 &  4 & 152 & 2.25 & 27.44\\
    {\tt iir1} & 18890 &  7487 & 26 &  5 & 220 & 1.74 & 21.22\\
    {\tt iir2} & 24488 & 10567 & 26 & 11 & 332 & 1.60 & 19.51\\
    {\tt cdp}  & 25024 &  6673 & 26 & 14 & 408 & 2.59 & 31.59\\
    \bottomrule
  \end{tabular}
\end{table}

From these results the main conclusion is that while the achievable
speedups are modest, the energy gains are very significant for these
single configuration kernels. This makes Versat a very attractive
accelerator for high performance battery operated devices. The only
requisite is that the vectors are long enough to justify the transfer
of data in and/or out of Versat. Although not shown by the above
results, the data transfer time dominates. For example, the {\tt vadd}
kernel processing time is only 1090 cycles and the remaining 3427
cycles account for data transfer and control.

The number of FUs used is low for the {\tt vadd} and {\tt iir1}
kernels. The {\tt vadd} kernel could perform multiple additions in
parallel but it is not necessary as a single ALU is enough to hide the
data transfer time if streaming the vectors. If the data were already
in the DE then multiple ALUs in parallel would accelerate the
processing time. The {\tt iir2} and {\tt cdp} kernels use more FUs
as they require more computations per vector element.

The {\em Control} number of cycles is low for all examples, which
shows that the configuration of the DE can be accomplished almost
completely while the DMA is running. Configuration can also be done
while the DE is running, as will be shown in the next subsection where
runtime reconfiguration is considered. In any case, the configuration
overhead is low.

Given the simplicity of the examples, their code size, in the order of
hundreds of bytes, is small. Many of these simple kernels can be
placed in the $8kB$ program memory and invoked when necessary.

% ----------------------------------------------------------------------
\subsection{Performance and energy consumption results for complex kernels}
\label{subsection:ResultsComplexKernels}

In this subsection, three more complex kernels, that demand self and
partial reconfiguration, are presented. The number of reconfigurations
is high and the kernels operate for a long time on the data fetched
from the external memory and/or produced by themselves. In these
examples the data transfer time is less significant. The examples are
the following:
\begin{itemize}
\item 1D-Convolution (conv-1D): very popular in applications such as
  Convolutional Neural Networks;
\item Fast Fourier Transform (FFT): very common in digital signal
  processing (see section~\ref{section:FFT});
\item K-Means Clustering algorithm (K-Means): widely used in Big Data
  applications (see section~\ref{section:kmeans}).
\end{itemize}

These examples are {\em parameterizable} and the parameters are passed
by the host processor using the CRF. The algorithm that runs on Versat
processes the parameters and generates configurations
accordingly. This would be hard to achieve with statically compiled
configurations and demonstrates the strength of self-generated
configurations. Partial reconfigurations are equally important
since they reduce considerably the reconfiguration time.

Results for 3 particular instances of the conv-1D, FFT and K-Means
algorithms are detailed in Table~\ref{tabExecR2}. The conv-1D result
was obtained by running the kernel on a dataset of $10^6$ points
applying 1-D convolution on a 256-point sliding window. The FFT result
pertains a 16384-point window size with a $50\%$ overlap over a
dataset of $10^6$ complex points. The K-Means result has been obtained
for one iteration over a dataset of $1.36 \times 10^6$ points of 30
dimensions and 34 centroids.

\begin{table}[!htb]
  \renewcommand{\arraystretch}{1.2} % more space between rows
  \caption{Complex kernel results.}
  \label{tabExecR2}
  \centering
  \begin{tabular}{lcccccc}
    \toprule
    Kernel & ARM & Versat & \#FUs & Code Size & Speedup & Energy Ratio\\
    \midrule
    {\tt conv-1D} & 2.26G & 104.51M & 13   &  668 & 14.95 & 182.32\\
    {\tt FFT}     & 1.28G &  34.50M & 12   & 3492 & 25.65 & 312.80\\
    {\tt K-Means} & 9.02G &   1.64G & 12.5 & 2764 &  3.8  &  46.28\\
    \bottomrule
  \end{tabular}
\end{table}

These results show that the speedup and energy efficiency improve with
the complexity of the kernels when compared to the simpler kernels in
the previous subsection. This has to do with the number of operations
done in parallel in the DE, but also with the number of DE
configurations that can be executed without fetching or saving new
data in the external memory. The K-Means algorithm fetches a new
datapoint chunk and applies 2 DE configurations: one for performing
datapoint classification and the other to update the centroid
positions. As for the FFT and conv-1D, after fetching a datapoint
chunk, several configurations are applied corresponding to the several
FFT stages, in case of the FFT and a window computation, in case of
conv-1D. The datapaths for these kernels also expose a higher ILP and
DLP in its computations. Hence, the speedup and energy efficiency of
the FFT and conv-1D are much higher when compared to the K-Means
algorithm. All these algorithms illustrate the power of using the
Versat CGRA compared to the ARM processor.

\subsubsection{Conv-1D results}

In the conv-1D kernel, the only parameterizable parameter is the
window size. Versat/ARM speedup results are shown in
figure~\ref{fig:convsu}.  The window size is varied from 32 to 1024
and is increased by powers of 2. All datapoints, from the first data
chunk, are multiplied by the sliding window coefficients and
accumulated. Then Versat reconfigures itself to advance to the next
window and the process repeats until the window is slid over all
datapoints.

\begin{figure}[!htb]
  \centering \includegraphics[width=0.55\textwidth]{drawings/speedupConv.eps}
  \caption{Conv-1D: speedup vs. window size.}
  \label{fig:convsu}
\end{figure}

As the sliding window size increases, the ARM core can better hide the
time needed to store the computed values while the Versat core keeps
its execution time linear on the window size. This causes a slight
speedup drop, from $16.3$ to $15.2$, as shown in the figure. Due to
the small size of the data chunks, the size of the internal memories
is adequate for both cores.

\subsubsection{FFT results}

The FFT kernel can be parameterized with the following parameters: the
number of datapoints, window size (must be a power of 2) and window
overlap size. The algorithm computes the FFT successively, for the
points in the sliding window, advancing the window for a number of
points given by the window size minus the overlap size. In
figure~\ref{fig:fftsu}, the Versat/ARM speedup is shown as a function of
the window size for 1 million datapoints and a half window ($50\%$)
overlap.

\begin{figure}[!htb]
  \centering \includegraphics[width=0.55\textwidth]{drawings/speedup3.eps}
  \caption{FFT: speedup vs. window size.}
  \label{fig:fftsu}
\end{figure}

Initially, as the sliding window size reaches the capacity of the
Versat memories, the speedup drops, while the ARM core uses its data
cache and pre-fetch mechanism to sustain its performance. However, as
the window size further increases, the ARM core reaches its own
internal memory limitations for streaming data, and the speedup
increases steadily again after that.

\subsubsection{K-Means Clustering results}

The parameters that can be passed to the K-Means Clustering kernel are
the following: the number of datapoints, number of dimensions and
number of centroids. In figure~\ref{fig:timeVSdpts}, it is shown how
the time for one iteration varies with the number of datapoints for a
fixed number of dimensions and centroids. The results are given for
both cores using logarithmic scales.

\begin{figure}[!htb]
  \centering \includegraphics[width=0.55\textwidth]{drawings/points.eps}
  \caption{K-Means: iteration time vs. number of datapoints, for 30 dimensions and 34 centroids.}
  \label{fig:timeVSdpts}
\end{figure}

These results show that for both systems the execution time scales
linearly with the number of datapoints. The Versat/ARM average speedup
is $3.8$, taking into account the number of clock cycles and the
working frequencies of the cores (Table~\ref{tabASICrs}). Despite the
modest speedup achieved in the K-Means Clustering algorithm, a
considerable amount of energy can be saved.

Each point requires the DE to be reconfigured twice to apply the
Assignment and Update steps. Since this algorithm is applied to
millions of points then Versat is reconfigured millions of times at
runtime. There is no reconfiguration overhead, as all reconfigurations
are done while the DE or DMA is running.

\subsection{Generated versus stored configurations}

To show that using self-generated configurations has advantage
compared to storing all configurations in the external memory,
consider the K-Means Clustering kernel for instance.

This algorithm is especially useful when applied to large data
sets. In the implementation above it has been chosen to apply it to
millions of datapoints. Since for this algorithm Versat
self-reconfigures for each point, then millions of configurations are
needed. Since one configuration uses 672 bits then 672Mbits are needed
only for configurations, much more than needed to store the program
itself which generates the configurations.

For algorithms that require a number of configurations tied to the
dataset size, pre-compiling and storing all configurations becomes
simply not practical. Even for the FFT kernel that only requires 43
configurations, it can be proved that the self-generated version is
marginally more efficient and has smaller memory footprint than the
stored configurations version. Therefore, self-generated
configurations is a good technique for CGRAs.


% ----------------------------------------------------------------------
\section{Comparing with other CGRAs}
\label{subsection:OtherCGRAs}

It is difficult to compare the Versat performance with the other
CGRAs. Published results always omit important information, and it is
hard to ascertain the conditions under which they have been
obtained. A possible solution would be to replicate the other
approaches in order to make fair comparisons. However, CGRAs are
complex cores and implementing them from their descriptions in
research papers, besides representing a formidable effort, is not
guaranteed to yield trustworthy results, as some important details
could be missed.

Notwithstanding, Versat can be compared with Morphosys for the FFT
kernel, since it is reported in~\cite{Kamalizad03} that the processing
time for a 1024-point FFT is 2613 cycles. Compared with the 12115
cycles taken by Versat, Morphosys is $4.6\times$ faster on this kernel
at the same operating frequency. This is not surprising, since
Morphosys has 64 FUs compared to only 11 FUs in Versat; Morphosys is
$4.5\times$ the size of Versat according to
Table~\ref{tabASICrs}. Using the frequency and power figures, it can
be derived that Morphosys uses $2.66\times$ more energy than Versat
for the FFT kernel.

Despite ADRES being one of the most cited CGRA architectures,
comparisons with this architecture could not be made, since the
execution times for the examples used in its publications have not
been given in absolute terms.

