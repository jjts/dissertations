%%%%%%%%%%%%%%%%%%%%%%%%%%%%%%%%%%%%%%%%%%%%%%%%%%%%%%%%%%%%%%%%%%%%%%%%
%                                                                      %
%     File: Thesis_Abstract.tex                                        %
%     Tex Master: Thesis.tex                                           %
%                                                                      %
%     Author: João D. Lopes                                            %
%     Last modified :  18 May 2016                                     %
%                                                                      %
%%%%%%%%%%%%%%%%%%%%%%%%%%%%%%%%%%%%%%%%%%%%%%%%%%%%%%%%%%%%%%%%%%%%%%%%

\section*{Abstract}

% Add entry in the table of contents as section
\addcontentsline{toc}{section}{Abstract}

This thesis describes Versat, a reconfigurable hardware accelerator
for embedded systems, for which this work substantially contributed.
Versat is a small and low power Coarse Grained Reconfigurable Array
architecture (CGRA), which implements self and partial reconfiguration
by using a simple controller unit. This approach targets ultra-low
energy applications such as those found in Wireless Sensor Networks
(WSNs), where using a GPU or FPGA accelerator is out of the
question. Compared to other CGRAs, Versat has a smaller number of
functional units interconnected in a full graph topology for maximum
flexibility. The lower number of functional units is compensated by
the ease of configuration and runtime reconfiguration. Unlike other
CGRAs, Versat can map sequences of nested program loops instead of a
single program loop at a time; self and partial reconfiguration
happens between the nested loops. Versat can be programmed in assembly
language and using a C++ dialect. Assembly programmability is a novel
and useful feature for program optimization and working around
compiler or architectural issues. This work contributed to the design
of the whole system, with special emphasis on verification, debug,
timing closure, multi-threading at the datapath level, boot loader
coding, DMA, AXI interfaces, benchmark coding and regression testing.
Results on a set of benchmarks show that Versat can accelerate single
configuration kernels by 4x with one order of magnitude energy
consumption improvement compared to a state-of-the-art processor. For
multiple configuration kernels the improvements are even better: 20x
acceleration with up to 2 orders of magnitude better energy
efficiency.

\vfill

\textbf{\Large Keywords:} Reconfigurable Computing, Coarse Grained
Reconfigurable Arrays, Embedded Systems, Low Power Systems

