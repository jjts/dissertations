%%%%%%%%%%%%%%%%%%%%%%%%%%%%%%%%%%%%%%%%%%%%%%%%%%%%%%%%%%%%%%%%%%%%%%%%
%                                                                      %
%     File: Thesis_Programming.tex                                     %
%     Tex Master: Thesis.tex                                           %
%                                                                      %
%     Author: João D. Lopes                                            %
%     Last modified :  20 September 2017                               %
%                                                                      %
%%%%%%%%%%%%%%%%%%%%%%%%%%%%%%%%%%%%%%%%%%%%%%%%%%%%%%%%%%%%%%%%%%%%%%%%

\chapter{Programming}
\label{chapter:programming}

Versat can be programmed using a C/C++ subset, and the code can be
compiled using the Versat compiler~\cite{Santiago2016}. Versat can
also be programmed in assembly language, given its easy to apprehend
structure. To the best of our knowledge, Versat is the only CGRA that
can be programmed in assembly. In this chapter, Versat programming is
illustrated using its C/C++ dialect, which is easier for explaining
the basic concepts. Also it will be explaind how to use Versat drivers
from the host system point of view (where the host system is an
embedded processor, for instance an ARM processor).

%The programming tools will not be explained as they are not the main focus of this report.

% ----------------------------------------------------------------------
\section{Basic programming}
\label{section:basicProgramming}

In order to outline the basics of Versat programming, a typical
example program is given in this section. The program adds two memory
interleaved vectors: one vector resides in the even addresses while
the other vector resides in the odd addresses. The program is shown in
figure~\ref{fig_vec_add} and comments are added for clarity.

\begin{figure}[!htb]
\centering \includegraphics[width=0.8\textwidth]{drawings/code.pdf}
\caption{Vector addition code.}
\label{fig_vec_add}
\end{figure}

The program starts in the {\tt main()} function, as is usual in C/C++
programs, and proceeds immediately to loading Versat with data using
the DMA. The DMA is configured ({\tt dma.config()}) with the external
and internal addresses, transfer size and direction. The {\tt
  dma.run()} function sets the DMA running, but does not wait for its
completion.

With the DMA running, the program proceeds to configure the DE. The
configuration register file is cleared in order to start a new
configuration ({\tt de.clearConfig()}). Certain language constructs
are interpreted as DE configurations and the compiler automatically
generates instructions that write these configurations to the CM. This
is the case of the ensuing {\tt for} loop. The fact that memory ports
are invoked in its body triggers this interpretation. Note that object
or variable declarations are not yet supported by the compiler. Data
must be referenced using the names of the memories or registers where
they are stored. When variables are supported, the issue of
recognizing DE configurations will need further research. For now, it
can be conjectured that the presence of data arrays in a loop is
sufficient for inferring a DE configuration. In the given example, the
expression in the loop body configures the DE to read the two vectors
from the two ports of memory 0, add their elements and place the
result in memory 1 using its port A.

Next, the program checks whether the DMA has finished loading the
data. The {\tt dma.wait()} function blocks the Controller until the
DMA is idle again. Note that the DMA has been running concurrently
with the Controller, which has been configuring the DE.

Once the DE is loaded and configured, it is run by means of the {\tt
  de.run()} function call. While it runs, the DMA is configured in
advance to transfer the result of the vector addition to the external
memory. Then the program waits for DE completion ({\tt de.wait()}) and
runs the DMA ({\tt dma.run()}). It is necessary to wait for DMA
completion ({\tt dma.wait()}) before exiting the program, to guarantee
that the result is stored in the external memory before control is
passed to a host processor.

% ----------------------------------------------------------------------
\section{Self and partial reconfiguration}
\label{section:selfPartialReconfiguration}

In this section, an example is presented to illustrate {\em
  self-generated partial reconfiguration} in Versat
(figure~\ref{fig_preconf}). The example shows a {\tt do-while}
loop. These kind of loops are always executed by the Controller,
because the DE has no means to stop conditionally.

\begin{figure}[!htb]
\centering \includegraphics[width=0.8\textwidth]{drawings/code2.pdf}
\caption{Self and partial reconfiguration code.}
\label{fig_preconf}
\end{figure}

A 2-level nested loop follows, where the body contains only
expressions involving memory ports. This nested loop is therefore
interpreted as a DE configuration and the compiler generates
instructions that write this configuration to the CM. Note that the
memory address expressions (between square brackets) use register R1,
which is updated inside the {\tt do-while} loop: R1 depends on R7
which is also updated in the loop.

In fact, R1 is the start address for the data in memories 0 and 1,
which corresponds to the Start parameter of the AGUs in the memory
ports used (see Table~\ref{tab:MemParameter}). Thus, only the Start
parameters in these AGUs need to be reconfigured, which means the DE
is {\em partially reconfigured} in each iteration of the {\tt
  do-while} loop.

Furthermore, the {\tt do-while} loop is {\em generating} one DE
configuration per iteration. Since this is done by the Controller,
without any intervention of the host processor, it may be said that
{\em Versat is self-reconfigurable}.

The {\tt de.wait()} function call, after the nested loop, waits for
the previous DE configuration to finish running. The following {\tt
  de.run()} function call runs the DE again for the current
configuration. While the DE is running, registers R7 and R12 are
updated and the {\tt do-while} loop goes on to the next iteration,
provided the loop control register (R12) is non-zero. In the next
iteration R1 is updated, the DE is partially reconfigured for the next
run, and the process repeats all over again.

% ----------------------------------------------------------------------
\section{Versat Driver}
\label{section:drivers}

Versat is designed to work as a co-processor for host systems. The
host system needs a very simple driver, which consists of only a few
functions for managing Versat. A program that uses the Versat driver
as is shown in the figure~\ref{fig_drivers}.

\begin{figure}[!htb]
\centering \includegraphics[width=0.8\textwidth]{drawings/drivers.pdf}
\caption{C code using Versat drivers.}
\label{fig_drivers}
\end{figure}

The reduced number of functions present in the driver make Versat very
easy to use by host processors. This is mainly because Versat is very
independent, it can reconfigure itself, access data from the external
memory and run simple control algorithms. However, Versat needs to be
programmed separately.

To use the driver, first of all, the header file ({\tt versat.h}) must
be included in the source file of the host program willing to use
Versat as an accelerator.

Second, the {\tt versat\_load()} function is called to load Versat
kernel(s) into Versat's program memory. Parameter {\tt progExt\_addr}
is the external memory address where the program resides and parameter
{\tt progInt\_addr} is the internal (to Versat) memory address where
it should be transferred; parameter {\tt progSize} is the kernel size
in 32-bit words.

Next, the {\tt versat\_wait()} function is called for waiting until
Versat has finished the last command. In this case, it guarantees that
the kernel code is already in the program memory, so that Versat is ready
to execute the next command.

Then, the {\tt versat\_run()} function is called to start execution. A
FFT kernel is used as the example. The first parameter, {\tt FFT},
identifies the kernel to be run. It is actually the address of the
kernel in Versat's program memory. Parameters {\tt x\_ptr}, {\tt
  w\_ptr} and {\tt X\_ptr} indicate the location of the input and
output vectors in the external memory. The FFT is computed over a
complex dataset of {\tt Npts} points, using windows of size {\tt
  WindowSize}, where two consecutive windows overlap by {\tt Overlap}
points.

While Versat is running, some host code could be put before calling
the {\tt versat\_wait()} function again, as this code will execute in
parallel with Versat if it does to take more time than the Versat
kernel. When the {\tt versat\_wait()} function exits the FFT
computation is done and one may add some post-processing code before
exiting the host program.


% ----------------------------------------------------------------------
\section{Comparison with other architectures}
\label{section:progComparisonWOtherArchitectures}

Versat is the only known CGRA that can be programmed in assembly
language. It supports configurations pre-compiled, i.e., is capable of
using configurations that are stored in the external memory, by
transferring them into its own configuration memory like other
architectures that can only be full reconfigured by transferring
configurations~\cite{Lee00,Mei05,Hartenstein99}. As it is able to
perform {\em self and partial reconfiguration}, i.e., generate
configurations on the fly and use highly sophisticated reconfiguration
mechanisms to partially modify them, since they use to be similar to
each other. Unlike some other architectures that use inefficient
reconfiguration mechanisms and some of them are very recommended not
be used~\cite{Ebeling96,Weinhardt03,Liu15}.

In conclusion, Versat is a very capable architecture of perform {\em
  self and partial reconfiguration} on the fly. It works as a
co-processor used to accelerate data computation, attached to a host
system who calls it by using some procedures stored in the Versat's
library ({\tt versat.h}). It is quite independent from the host system
for generating configurations and transfer the data to/from the
external memory, freeing the host for other useful computations.
