%%%%%%%%%%%%%%%%%%%%%%%%%%%%%%%%%%%%%%%%%%%%%%%%%%%%%%%%%%%%%%%%%%%%%%%%
%                                                                      %
%     File: Thesis_Resumo.tex                                          %
%     Tex Master: Thesis.tex                                           %
%                                                                      %
%     Author: João D. Lopes                                            %
%     Last modified :  15 October 2017                                 %
%                                                                      %
%%%%%%%%%%%%%%%%%%%%%%%%%%%%%%%%%%%%%%%%%%%%%%%%%%%%%%%%%%%%%%%%%%%%%%%%

\section*{Resumo}

% Add entry in the table of contents as section
\addcontentsline{toc}{section}{Resumo}

Esta tese descreve o Versat, um acelerador de hardware reconfigurável
para sistemas embebidos. O Versat é uma arquitetura de matriz
reconfigurável de grão grosso e de baixa potência, que implementa
reconfiguração auto-gerada e parcial usando um controlador. Esta
abordagem visa aplicações de muito baixo consumo energético, como as
encontradas em redes de sensores sem fios, onde o uso de GPUs ou FPGAs
está fora de questão. Comparado com outras arquitecturas, o Versat
possui um baixo número de unidades funcionais interligadas em
topologia de grafo completo. O número de unidades funcionais é
compensado pela facilidade de configuração e reconfiguração. Ao
contrário de outras arquitecturas, o Versat pode mapear sequências de
malhas aninhadas em vez de uma única malha; a reconfiguração
auto-gerada e parcial ocorre entre malhas aninhadas. O Versat pode ser
programado em assembly ou em C++. A programação em assembly é uma
característica útil para optimização de programas e contorno de erros
de compilação/arquitetura. Este trabalho contribuiu para o projecto de
todo o sistema, com especial ênfase nas fases de verificação,
depuração, restrições temporais, concorrência ao nível de circuito de
dados, acesso directo à memória, interfaces AXI, programas para
caracterização e testes de regressão. Resultados de caracterização
mostram que o Versat pode acelerar algoritmos de configuração única
até 4x com um consumo energético uma ordem de grandeza inferior
comparado com um processador de referência. No caso de algoritmos com
múltiplas configurações, os benefícios são ainda maiores: aceleração
de 20x com até 2 ordens de grandeza de mais eficiência energética.

\vfill

\textbf{\Large Palavras-chave:} Computação reconfigur{\'a}vel, Matrizes Reconfigur{\'a}veis de Gr{\~a}o Grosso, Sistemas Embebidos, Sistemas de Baixa Pot{\^e}ncia


