%%%%%%%%%%%%%%%%%%%%%%%%%%%%%%%%%%%%%%%%%%%%%%%%%%%%%%%%%%%%%%%%%%%%%%%%
%                                                                      %
%     File: Thesis_Introduction.tex                                    %
%     Tex Master: Thesis.tex                                           %
%                                                                      %
%     Author: João D. Lopes                                            %
%     Last modified :  30 September 2017                               %
%                                                                      %
%%%%%%%%%%%%%%%%%%%%%%%%%%%%%%%%%%%%%%%%%%%%%%%%%%%%%%%%%%%%%%%%%%%%%%%%

\chapter{Introduction}
\label{chapter:introduction}

\section{Motivation}
\label{section:motivation}

Cloud Computing is a fashionable but well known topic, which has been
studied since at least 1971~\cite{white1971network}. However, it can
be argued that it is better to have distributed computing, storage and
networking resources than to concentrate everything in a single
node~\cite{armbrust2010view}.

Edge computing is the logic response to cloud computing; the data is
processed on the edge of the network, near its
source~\cite{gaber2014pocket}. This considerably reduces the bandwidth
needed to transfer the data between the nodes and the data
center. Edge Computing can benefit many applications and devices. In
particular it is extremely useful in Wireless Sensor Networks
(WSNs)~\cite{beckman2016waggle}.

A crucial problem of deploying computational resources on WSN nodes is
power consumption. WSN nodes are normally battery operated and very
price sensitive. Therefore, if one can deliver the same functionality
with a smaller and more power efficient device, a more competitive
product will be released.

Since WSN nodes are extremely constrained in terms of energy
efficiency and price, they often employ inexpensive and low
performance embedded processors. To attend the computational demands
of certain applications, it is common to place dedicated hardware
accelerators in the chip, used by these applications. However, the
fact that these hardware blocks are not programmable increases the
risk of design errors and prevents future updates. Ideally a
programmable accelerator should be employed such as a Graphics
Processing Unit (GPU) or a Field Programmable Gate Array
(FPGA). However, these accelerators are much larger than the embedded
processor cannot viably be used.

A more suitable accelerator is the Coarse Grained Reconfigurable Array
(CGRA), as it can be made small and energy efficient. A CGRA is a
collection of programmable Functional Units (FUs) and embedded memories
connected by programmable interconnects. This structure is called the
Reconfigurable Array. When programmed with a stream of configuration
bits, the reconfigurable array forms hardware datapaths able to
execute certain tasks orders of magnitude faster than a conventional
Central Processing Unit (CPU). CGRAs are used as hardware
co-processors to accelerate algorithms that are time/power consuming
in regular CPUs.

Normally, the reconfigurable array is used only to accelerate program
loops and the non-loop code is run on a CPU. For this reason, CGRAs
normally include a conventional processor. For example, the Morphosys
architecture~\cite{Lee00} integrates a small Reduced Instruction Set
Computer (RISC) and the ADRES architecture~\cite{Mei05} integrates a
Very Large Instruction Word (VLIW) processor.

Despite being an obvious solution, in practice CGRAs have not been
widely adopted in the industry. The main problems that were identified
with existing the CGRAs and that motivated this work are the
following: they are not made small enough, they have limited
reconfiguration control and are difficult to program. Therefore, we
propose some architectural improvements to address these problems
which follow three basic ideas.

\section{Solutions}

The first idea is to use a low number of FUs which form a fully
connected graph topology~\cite{vecparBook16}. This simultaneously
tackles the size and programmability problems (programmability
increases with connectivity). Normally, graphs with constrained
connectivity are employed in CGRAs to allow spatial compactness and a
high frequency of operation. However, with a low number of FUs the
full mesh topology becomes affordable. Integrated Circuit (IC)
implementation results indicate that only $4.04\%$ of the core area is
occupied by the full mesh interconnect. Plus, a low power device will
rarely choose to operate at a very high frequency, so the moderate
frequencies achieved by the IC implementation should cover a vast
range of applications.  The reconfiguration process is {\em
  quasi-static}, i.e., it is more frequently reconfigured compared to
static arrays such as~\cite{Hartenstein99}, but much less frequently
compared to dynamic arrays such as~\cite{Mei05}. Although large
datapaths cannot be build with few FUs, it is necessary to break large
datapaths into several smaller ones. On the other hand, {\em any}
datapath that does not require more FUs than existing can be build due
to the full connectivity of the nodes. Finally, programming is
drastically simplified as there is no need for {\em place \& route}
algorithms in compilation flow such as in FPGAs and other
fabrics. Another interesting feature is the fact that a full mesh is
such an easy to apprehend structure that assembly programming is
enabled, which is invaluable for optimizations and compiler or
hardware bug workarounds.

The second idea is to make the configuration register addressable at
the word level and therefore enable fine partial
reconfiguration~\cite{vecparBook16}. The CGRA configuration is divided
in spaces which correspond FUs. The configuration spaces are further
divided in configuration fields which are individually
addressable. Partial reconfiguration is useful to reduce the
reconfiguration time to a minimum, sometimes exploiting the similarity
between successive configurations. Reducing reconfiguration time has a
dramatic influence on improving the performance, which can compensate
a lower frequency of operation.

The third idea is to integrate a small programmable controller in the
CGRA to manage reconfigurations and data transfers to/from external
memory~\cite{jcer16}. The controller is in charge of the main program
flow of the accelerator, sequencing the configurations and using
partial reconfiguration whenever possible. The controller can spawn
data stream compute threads in the CGRA and data movement threads
using a Direct Memory Access (DMA) unit. While these threads are
running, the controller can prepare the next configurations.


%%%%%%%%%%%%%%%%%%%%%%%%%%%%%%%%%%%%%%%%%%%%%%%%%%%%%%%%%%%%%%%%%%%%%%%%
\section{Topic Overview}
\label{section:overview}

In this work we present Versat, a new reconfigurable hardware
accelerator which is suitable for low cost and low power devices. Its
architecture uses a small number of functional units and a simple
controller. A smaller array limits the size of the data expressions
that can be mapped to the CGRA, forcing large expressions to be broken
into smaller ones which can be executed sequentially in the
CGRA. Therefore, Versat requires mechanisms for handling large numbers
of configurations and frequent reconfiguration.

Versat is meant to be used as a co-processor used by means of an
Application Programming Interface (API) containing a set of useful
kernels. Full application developers can use a commercial embedded
processor to benefit from its rich ecosystem and drop in a Versat core
for performance and power optimization. Versat programmers can create
a set of kernels for application programmers use. In this way, the
software and programming tools of the CGRA are clearly separated from
those of the application processor.


%%%%%%%%%%%%%%%%%%%%%%%%%%%%%%%%%%%%%%%%%%%%%%%%%%%%%%%%%%%%%%%%%%%%%%%%
\section{Objectives}
\label{section:objectives}

The main objective driving the development of the Versat architecture
is to obtain useful acceleration at low power consumption and small
silicon area. In fact, Versat can replace a number of dedicated
hardware accelerators in a System on Chip (SoC), consisting in a
smaller, more power efficient and safer to design system, as a
programmable solution eliminates the development risk of designing
dedicated hardware accelerators.

Digital signal processing applications are targeted for biometrics,
speech recognition, artificial vision, security, etc. The overall goal
of the project is to create an Intellectual Property (IP) core and a
library of useful procedures. A clean procedural interface to a host
processor or driver is also an objective. With such an interface hosts
can have tasks executed in the CGRA by simply calling procedures and
passing arguments.

To illustrate the applicability of the Versat core, two popular
algorithms for embedded applications have been implemented: the Fast
Fourier Transform (FFT)~\cite{jcer17} and the K-Means Clustering
algorithm~\cite{fpl17}. The FFT is a central digital signal processing
algorithm, used in many applications. The importance of the K-Means
Clustering algorithm has been increasing with the demand for efficient
classification of large datasets of multiple dimensions and clusters.

These two algorithms have been accelerated using Graphics Processor
Units (GPUs)~\cite{moreland2003fft,doggettfft,Che07,farivar2008} and
using Field Programmable Gate Arrays
(FPGAs)~\cite{chao2005design,sun2008design,derafshi2010high,Liu2005,hussain2011,kutty2013}. Although
these implementations have been shown to improve performance and
energy consumption over General Purpose Processors (GPPs)
implementations, they can hardly be used in ultra low energy scenarios
such as WSNs. A few works have demonstrated that it is more efficient
in terms of time and energy to run the FFT and K-Means Clustering
algorithms in a distributed
fashion~\cite{canli2006power,Sasikumar2012,xu2012power}.

%%%%%%%%%%%%%%%%%%%%%%%%%%%%%%%%%%%%%%%%%%%%%%%%%%%%%%%%%%%%%%%%%%%%%%%%
\section{Author's Work}
\label{section:authorWork}

The work here presented is the result of a few people's work. When I
started working on this project, in the summer of 2014, there was a
preliminary, non-functional version of the system. I contributed with
many ideas: independent Address Generation Units (AGUs), which enabled
multi-threading in the Versat reconfigurable array; improvement of the
operation frequency by applying pipelining techniques; the boot loader
Read Only Memory (ROM) software and its handshaking protocol with a
host system. I also implemented the DMA, master and slave Advanced
Extensible Interface (AXI) interfaces, contributed to the
Application-Specific Integrated Circuit (ASIC) implementation, I wrote
all assembly kernels used for tests and implemented a regression test
system where it is easy to add or remove tests. As a result of this
work, the following papers have been
published:~\cite{deSousa16,vecparBook16,jcer16,rec17,fpl17,jcer17}.

%%%%%%%%%%%%%%%%%%%%%%%%%%%%%%%%%%%%%%%%%%%%%%%%%%%%%%%%%%%%%%%%%%%%%%%%
\section{Thesis Outline}
\label{section:outline}

This thesis is composed of 6 more chapters. In the second chapter, the
advantages and disadvantages of other architectures are discussed. In
the third chapter, the Versat architecture is fully described. In the
fourth chapter, it is described how to program Versat. In the fifth
chapter, two algorithms common in real applications are fully
studied. In the sixth chapter, experimental results are presented. In
the seventh and final chapter, our achievements are pointed out and
directions for future work are outlined.

