%%%%%%%%%%%%%%%%%%%%%%%%%%%%%%%%%%%%%%%%%%%%%%%%%%%%%%%%%%%%%%%%%%%%%%%%
%                                                                      %
%     File: Thesis_Resumo.tex                                          %
%     Tex Master: Thesis.tex                                           %
%                                                                      %
%     Author: Andre C. Marta                                           %
%     Last modified :  2 Jul 2015                                      %
%                                                                      %
%%%%%%%%%%%%%%%%%%%%%%%%%%%%%%%%%%%%%%%%%%%%%%%%%%%%%%%%%%%%%%%%%%%%%%%%

\section*{Resumo}

% Add entry in the table of contents as section
\addcontentsline{toc}{section}{Resumo}

Esta tese apresenta um novo ambiente de simulação para a arquitectura RV32-Versat baseado 
na ferramenta de simulação Verilator. A arquitectura RV32-Versat consiste no processador 
PicoRV32, com arquitectura RISC-V, ligado ao Versat. Este novo ambiente de 
simulação apresenta vantagens significativas quando comparado com os ambientes de 
simulação mais tradicionais que usam simuladores baseados em eventos. A primeira vantagem 
é a rapidez: o novo ambiente é significativamente mais rápido, poupando assim tempo no 
processo de desenvolvimento de novas aplicações para a arquitectura RV32-Versat. A 
segunda vantagem é o custo: o Verilator é disponibilizado com uma licença gratuita, 
enquanto os simuladores baseados em eventos típicos necessitam de licenças caras, 
difíceis de justificar para pequenas empresas e projectos. A terceira e última vantagem é 
o suporte directo para a co-simulação de \textit{software} e \textit{hardware}. Os 
simuladores baseados em eventos falham neste ponto, mas o novo ambiente baseado no 
Verilator resolve este problema, permitindo uma boa integração através do uso de C++ ou 
SystemC. Este novo ambiente de simulação é o resultado de um estudo detalhado, por um 
lado, dos diferentes tipos de simuladores existentes, e por outro lado, da arquitectura 
RV32-Versat, também apresentada nesta tese.

\vfill

\textbf{\Large Palavras-chave:} Matrizes Reconfiguráveis de Grão Grosso, Ambiente de 
Simulação, Verilator, Simulação de Matrizes Reconfiguráveis de Grão Grosso, Simulação de 
Alto Nível, Co-Simulação

