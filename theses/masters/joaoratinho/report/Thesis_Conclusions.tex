%%%%%%%%%%%%%%%%%%%%%%%%%%%%%%%%%%%%%%%%%%%%%%%%%%%%%%%%%%%%%%%%%%%%%%%%
%                                                                      %
%     File: Thesis_Conclusions.tex                                     %
%     Tex Master: Thesis.tex                                           %
%                                                                      %
%     Author: Andre C. Marta                                           %
%     Last modified :  2 Jul 2015                                      %
%                                                                      %
%%%%%%%%%%%%%%%%%%%%%%%%%%%%%%%%%%%%%%%%%%%%%%%%%%%%%%%%%%%%%%%%%%%%%%%%

\chapter{Conclusions}
\label{chapter:conclusions}

In this thesis the simulation environment developed for the RV32-Versat architecture was 
presented. This simulation environment presents a considerable improvement over the 
typical simulation environments using event-driven simulators: it is faster, it is 
inexpensive, not requiring the acquisition of licences, and allows a direct 
implementation of software and hardware co-simulation, avoiding the use of the \ac{VPI} 
or inefficient ad hoc solutions.

This ambient was developed after a detailed study of the state of the art of 
\ac{CPU} and \ac{CGRA} simulators, that did not only include the event-driven and 
cycle-accurate simulators, but also included custom simulators specifically developed for 
CGRAs. This was done to ensure that the simulation environment would correspond to the 
defined objectives.

To test the new simulation environment an application using a \ac{CNN} was developed and 
its results were compared with the ones obtained with the RV32-Versat architecture 
implemented in an \ac{FPGA}. This way it was possible to ensure that the results obtained 
with the new simulation environment were correct.

% ----------------------------------------------------------------------
\section{Achievements}
\label{section:achievements}

The first achievement of this thesis was the development of a faster simulation 
environment. As it was seen in the benchmark presented in 
Section~\ref{section:benchmark}, this simulation environment can be up to 3.81 times 
faster than a simulation environment using the traditional event-driven simulators. This 
can help to cut the time needed to develop and debug applications for the RV32-Versat.

Another important achievement is the reduction of the amount of money spent in licences. 
This happens because the new simulation environment uses Verilator, an open-source 
simulator, in contrast to the event-driven simulators that require expensive licences. 
This is particularly useful for small companies, allowing them to spend their limited 
resources in other areas.

The third achievement is the improved support for hardware and software 
co-simulation in this new simulation environment, dismissing the use of special 
interfaces (like the Verilog \ac{VPI}) or ad hoc solutions. With this new environment the 
testbench can be written in C++ or SystemC, therefore allowing a seamless integration of 
hardware and software co-simulation.

The last achievement was the development of a simulation environment that is independent 
from eventual changes in the RV32-Versat architecture. This means that if the 
architecture of RV32-Versat is changed, the simulation environment will keep working. 
This would not happen if a simulator developed for this specific architecture was used, 
instead of Verilator.

% ----------------------------------------------------------------------
\section{Future Work}
\label{section:future}

The work developed during this thesis can be developed in three different paths. The 
first is the development of a high-level simulator for this architecture that works at 
the functional unit level instead of the RTL. It could use a high-level simulator 
specially developed for the Versat architecture, similarly to~\cite{chen:CGRA}, extracting
the data from the memories and executing the desired high-level operations (no bit-level 
operations) by reading the configuration bits. This would allow for an even better 
performance, but it would also have a disadvantage: architecture changes in Versat would 
also require changes in the simulator architecture. This kind of high-level approach does 
not evaluate any kind of timing in the circuit, but this is also not needed once the 
circuit is silicone proven and its frequency of operation is known.

Another path would be to develop a software and hardware co-simulation environment for 
this architecture using the new simulation environment that, has explained before, makes 
this much easier. This would allow to run the software in a computer, adding a C++ class 
to simulate the RV32-Versat hardware.

The third path is a much more ambitious project: create an \ac{API} that would 
allow to generate and simulate Versat datapaths with an almost complete abstraction from 
the hardware. This could be done using Versat's C++ Driver, described in 
Section~\ref{subsection:c-driver}, as a basis, adding other classes to both generate the 
datapaths programmed in C++ and to simulate them. With this \ac{API} the whole system 
could be evaluated just in software, therefore not needing Verilog simulators. For 
further tests the Verilog code generated by the \ac{API} could be simulated using the 
Verilator co-simulation capabilities or it could be used directly in an \ac{FPGA}. This 
solution would allow to have an even faster way to simulate Versat without needing to 
significantly change the \ac{API}.
