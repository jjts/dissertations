%%%%%%%%%%%%%%%%%%%%%%%%%%%%%%%%%%%%%%%%%%%%%%%%%%%%%%%%%%%%%%%%%%%%%%%%
%                                                                      %
%     File: Thesis_Introduction.tex                                    %
%     Tex Master: Thesis.tex                                           %
%                                                                      %
%     Author: Andre C. Marta                                           %
%     Last modified :  2 Jul 2015                                      %
%                                                                      %
%%%%%%%%%%%%%%%%%%%%%%%%%%%%%%%%%%%%%%%%%%%%%%%%%%%%%%%%%%%%%%%%%%%%%%%%

\chapter{Introduction}
\label{chapter:introduction}


%%%%%%%%%%%%%%%%%%%%%%%%%%%%%%%%%%%%%%%%%%%%%%%%%%%%%%%%%%%%%%%%%%%%%%%%
\section{Motivation}
\label{section:motivation}

In any digital circuit simulation is fundamental to verify if the circuit is
working properly in the different stages of its development. The motivation for
this thesis became clear when I started working with Versat about a year ago, in
the context of an internship in which I had the opportunity to participate in
the development of an MP3 encoder. During that project I was able to understand
the main limitations of the simulation environments using traditional
simulators: they are slow with complex simulations, many times require costly
licences and they provide a difficult integration of software / hardware
co-simulation, many times leading to the use of ad hoc solutions.

These limitations gave me the motivation to look for a solution that would
address them, therefore making the simulation process during a project faster,
cheaper, more efficient and with an integrated support for software and hardware
co-simulation. That way, industrial projects like the one I participated in
could be done in a much more efficient way, which can be very important for
companies with little resources.

%%%%%%%%%%%%%%%%%%%%%%%%%%%%%%%%%%%%%%%%%%%%%%%%%%%%%%%%%%%%%%%%%%%%%%%%
\section{Objectives}
\label{section:objectives}

The main goal of this thesis is to develop a new simulation environment for the
RV32-Versat system that is able to overcome the typical problems inherent to
using simulation environments based on traditional simulators, as referred in
the previous section. The RV32-Versat system consists of a Versat \ac{CGRA},
controlled by a RISC-V PicoRV32 processor and has been developed in the context
of two masters theses, including this one.

Consequently, the new simulation environment has three main objectives. The
first one is being considerably faster than simulating the RV32-Versat
architecture with commercial simulators. This can be particularly useful when
developing new applications for this architecture, since it saves time during
the debug process. The second objective is being cheaper than traditional
simulators, saving resources. Finally, the third objective is to support
hardware and software co-simulation in an integrated evironment rather than
using special interfaces and/or ad hoc solutions.

Fulfilling these objectives required studying the current state of the art of
the simulators and the RV32-Versat architecture. This way, it was possible to
understand what types of simulators are more suitable for the Versat
architecture. To validate the simulation environment, a convolutional neural
network application for recognizing hand-written digits has been developed.
This application has also been used to benchmark different commercially
available simulators.


%%%%%%%%%%%%%%%%%%%%%%%%%%%%%%%%%%%%%%%%%%%%%%%%%%%%%%%%%%%%%%%%%%%%%%%%
\section{Author's Work}
\label{section:work}

This work was developed during an internship at IObundle, Lda, the
company that developed the Versat architecture. The internship started in
September 2018 and allowed the candidate to gain a deeper knowledge of the
Versat architecture, mainly by participating in a project with an
international client where the Versat architecture was used to accelerate
the front end of an MP3 encoder.

The work presented here is the result of the work of a few people: both the
PicoRV32 processor and the Versat \ac{CGRA} had been previously created, and the
RV32-Versat architecture was developed in collaboration with another student
that is developing a multi-layer architecture for RV32-Versat. Therefore, the main 
contributions made by the candidate for this architecture were the removal of the 
controller previously used by Versat, called PicoVersat~\cite{picoversat}, replacing it 
by the PicoRV32 processor, the development of the application presented in 
Section~\ref{section:application} and the creation of a new simulation environment.


%%%%%%%%%%%%%%%%%%%%%%%%%%%%%%%%%%%%%%%%%%%%%%%%%%%%%%%%%%%%%%%%%%%%%%%%
\section{Thesis Outline}
\label{section:outline}

This thesis is composed of 6 more chapters. In the second chapter the
RV32-Versat architecture is presented, in order to contextualize the other
chapters. The state of the art of \ac{CPU} and \ac{CGRA} simulators is detailed
in chapter 3. In the fourth chapter, a simulation environment using typical
commercial simulators is presented and their problems are reported. In the fifth
chapter the new simulation environment is presented, which overcomes the
problems of the previous one. In the sixth chapter the results with the
simulation environments are shown and, finally, in the seventh chapter the
conclusions are presented.

