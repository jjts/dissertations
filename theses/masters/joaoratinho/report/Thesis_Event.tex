\chapter{Simulating The RV32-Versat Architecture Using Event-Driven Simulators}
\label{chapter:sim_event}

In this chapter, the simulation environment for the RV32-Versat architecture
using commercially available event-driven simulators, more specifically the
Cadence NCsim and Synopsys VCS simulators, is detailed. As explained in the
previous chapter, event-driven simulators work by taking events sequentially,
propagating them through the circuit until the circuit reaches a steady
state~\cite{tan:vhstas,gunes:survey,palnitkar:verilog}. They produce accurate
simulation results with detailed timing information at the cost of the
simulation speed.

\section{Testbench}
\label{section:tb}

This type of simulators work with testbenches written in an \ac{HDL} such as
Verilog or VHDL. In Figure~\ref{fig:tb_verilog}, an example of a simple
testbench written in Verilog and used by NCSim and VCS is shown. This testbench
has RV32-Versat as the \ac{UUT} and uses a clock period of 10 ns, meaning that
the system is simulated at a frequency of 100 MHz. After 100 clock cycles with
the reset enabled ({\tt resetn=0}) the reset is disabled, and RV32-Versat starts
running. The data bus external ports are not used, since the memories are
initialized at compile time and no input hex file is loaded by the
testbench. The simulation finishes when the trap signal is activated (with {\tt
  resetn=0}), and if the flag VCD is passed when running the simulator a \ac{VCD}
file will be generated. The trap signal is activated when there is an invalid
memory access, which is useful for debug and for stopping the simulation, which
is caused by the user software that intentionally accesses the memory at a non
mapped location.

\lstset{language=verilog}
\begin{figure}[!htb]
	\begin{minipage}{\linewidth}
		\begin{spacing}{0.7}
			\begin{lstlisting}
			`timescale 1 ns / 1 ps
			
			module system_tb;
				reg clk = 1;
				always #5 clk = ~clk;
				
				reg resetn = 0;
				
				initial begin
					`ifdef VCD
						$dumpfile("system.vcd");
						$dumpvars();
					`endif
					repeat (100) @(posedge clk);
					resetn <= 1;
				end
				
				wire          led;
				wire          ser_tx;
				wire          trap;
				reg           databus_sel;
				reg           databus_rnw;
				reg [14:0]    databus_addr;
				reg [31:0]    databus_data_in;
				
				system uut (
					.clk             (clk        ),
					.resetn          (resetn     ),
					.led             (led        ),
					.databus_sel     (databus_sel),
					.databus_rnw     (databus_rnw),
					.databus_addr    (databus_addr),
					.databus_data_in (databus_data_in),
					.ser_tx          (ser_tx     ),
					.trap            (trap       )
				);
				
				initial begin
					databus_sel = 0;
					databus_rnw = 1;
					databus_addr = 0;
					databus_data_in = 0;
				end
				
				always @(posedge clk) begin
					if (resetn && trap) begin
						$finish;
					end
				end
			endmodule
			\end{lstlisting}
		\end{spacing}
	\end{minipage}
	\vspace*{-10mm}
	\caption{Example testbench in Verilog for RV32-Versat.}
	\label{fig:tb_verilog}
\end{figure}

\section{Verilog VPI}
\label{section:vpi}

The downside of using an HDL testbench with the event-driven simulators is their
limited support for software and hardware co-simulation. One way to overcome
this problem is to use the \ac{VPI}~\cite{vpi}, a C-programming interface for
Verilog that consists in a set of access and utility routines that are called
from standard C programming language functions. These routines interact with the
instantiated simulation objects contained in the Verilog design.

\lstset{language=C}
\begin{figure}[!htb]
	\begin{minipage}{\linewidth}
		\begin{spacing}{0.7}
			\begin{lstlisting}
			#include "vpi_user.h"
			
			void example() {
				vpi_printf("Example function\n");
				return;
			}
			\end{lstlisting}
		\end{spacing}
	\end{minipage}
	\vspace*{-10mm}
	\caption{Example C code for VPI.}
	\label{fig:vpi_c}
\end{figure}

To better explain how the~\ac{VPI} works there is a simple example of a C
function in Figure~\ref{fig:vpi_c}, called {\tt example.c} that prints text in
the terminal, using the {\tt printf} function from the \ac{VPI}. This function
now needs to be associated with a system task. For this, a special data
structure needs to be created, of the type {\tt vpi\_systf\_data}. The code for
the creation and registration of the system task can be seen in
Figure~\ref{fig:vpi_routine}. After this stage, the system task must be called
in the Verilog testbench of the circuit to simulate. This can be done either in
{\tt initial} blocks or in {\tt always} blocks. In Figure~\ref{fig:vpi_verilog}
an example using an {\tt initial} block is shown. The last thing to do is
linking the task with the simulator. This is usually made in the terminal when
calling the simulator.

\lstset{language=C}
\begin{figure}[!htb]
	\begin{minipage}{\linewidth}
		\begin{spacing}{0.7}
			\begin{lstlisting}
			#include "example.c"
			
			void hello_register()
			{
				s_vpi_systf_data tf_data; //data structure
			
				tf_data.type      = vpiSysTask; //does not return value
				tf_data.tfname    = "$hello"; //name of the task
				tf_data.calltf    = hello_calltf; //pointer for this routine
				tf_data.compiletf = hello_compiletf; //pointer for the compiled instance
																	  //of the task

				vpi_register_systf(&tf_data); //pointer to the structure
			}
			
			//Register the new task
			void (*vlog_startup_routines[ ]) () = {
				hello_register,
				0
			;}
			
			\end{lstlisting}
		\end{spacing}
	\end{minipage}
	\vspace*{-10mm}
	\caption{VPI task creation and registration.}
	\label{fig:vpi_routine}
\end{figure}

\lstset{language=Verilog}
\begin{figure}[!htb]
	\begin{minipage}{\linewidth}
		\begin{spacing}{0.7}
			\begin{lstlisting}
			module example ();
			
			initial begin
				$hello;
				#10 $finish;
			end
			   	  
			endmodule
			\end{lstlisting}
		\end{spacing}
	\end{minipage}
	\vspace*{-10mm}
	\caption{Verilog code to invoke the task.}
	\label{fig:vpi_verilog}
\end{figure}

As it can be seen, although it is possible to do software and hardware
co-simulation using Verilog testbenches, the process is not direct, making it a
disadvantage in simulation environments based on event-driven simulators. If the
low performance of this type of simulators is also taken into account (as it can
be seen in the benchmarks in Section~\ref{section:benchmark}), it can be
concluded that a simulation environment based on event-driven simulators does
not reach the objectives pretended: it is not fast, the simulators licences are
expensive and it does not provide a straightforward way of integrating hardware
and software co-simulation. Therefore, an alternative is necessary.  There is
another event-driven simulator (Icarus Verilog) that could solve the cost
problems, since it is free, but this simulator does not support the {\tt
  generate} loops used in the RV32-Versat Verilog code and has a very low
performance, as seen in Section~\ref{section:performance}.

