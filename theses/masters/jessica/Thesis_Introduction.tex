%%%%%%%%%%%%%%%%%%%%%%%%%%%%%%%%%%%%%%%%%%%%%%%%%%%%%%%%%%%%%%%%%%%%%%%%
%                                                                      %
%     File: Thesis_Introduction.tex                        %
%     Tex Master: Thesis.tex                                %
%                                                                      %
%     Author: Carlos A. Rodrigues                         %
%     Last modified : 15 Abril 2013                         %
%                                                                      %
%%%%%%%%%%%%%%%%%%%%%%%%%%%%%%%%%%%%%%%%%%%%%%%%%%%%%%%%%%%%%%%%%%%%%%%%

\chapter{Introdu\c{c}\~ao}
\label{chapter:introducao}

explicar o que é um sistema e o que se pretende com ele.


% --------------------------------------------------------------------- 
\section{Contexto}
\label{section:context}

n\~ao existe nenhum sistema desenvolvido com o que a startup pretende. e os que existem s\~ao demasiado caros, têm demasiadas coisas o que corresponde a um custo de energetico elevado.

% ----------------------------------------------------------------------
\section{Motiva\c{c}\~ao}
\label{section:motiva}

desenvolver um sistama necessario para uma startup, o sistema vai ser desenvolvido a medida com o pretendido com a startup. 

\section{Objectivos}
\label{section:objectivo}

dizer qual é o objectivo do projecto. explicar que é ter um sistema com um processador, spi, ROM, flash,i2c,fifos de comunica\c{c}\~ao,uart, GPIO

Imagem com um sistema que é pretendido,

explicar as interfaces de de entrada e saida de dados, e os modelos dentro do systema.


\section{Desafios}
\label{section:desafio}

desenvolver um sistema com todos os modelos a funcinar correctamente. 
