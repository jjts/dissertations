\chapter{Estado-da-arte}
\label{chapter:estadodaarte}

\section{OpenRisc}
\label{section:OpenRisc}

é uma comunidade opensource que desenvolve hardware e ferramentas que ajudam a na cria\c{c}\~ao de novos sistemas o OrpSoc

cores têm codigo escrito verilog e VDH sendo o verilog maioritário

\subsection{Toolchain}

é a toolchain da GNU onde foi adicionado o processador desenvolvida pela comunidade.

foi desenvolvida pela comunidade falar sobre ferramenteas 

tem a selac\c{c}\~ao do de onde o codigo vais correr.

por uma imagem a explicar a ideia de ser preciso compilar com o mboard

\subsection{OrpSoc}

explicar a ferramente que foi desenvolvida para que serve e as vantagens 

mudou de nome (FuseSoc)para se tornar uma ferramente independente da comunidade.

por uma imagem de como o OrpSoc está construido.

\section{Ferramentas de simula\c{c}\~ao}
\label{section:ferramentas}

vantagem de ter um sistema de simula\c{c}\~ao.

\subsection{Or1ksim}

dizer que é um simulador desenvolvido em C que predente simular o processador.

vantagens e desvantagens

\subsection{Icarus}

explicar como funciona o icarus que precisa de uma testbench em verilog, dá para ver as ondas e tem a vantagem de ter o estado de alta impedancia. 

Mas demora muito tempo.

Por uma imagem das ondas com alta impedancia com um tempo igual no verilog

\subsection{Verilator}

explica como funciona o verilator, converte os codigo em verilog em codigo C ou systemC conforme o que a pessoa pretender.

pode ser a testbench escrita em C ou systemC, que se pode ver as ondas e têm a desvantagem que n\~ao conseguir sumilar o estado de alta impetancia.

muito rapido comparado com o icarus

Por uma imagem das ondas com alta impedancia com um tempo igual no icarus

\subsection{Placa FPGA}

explicar o que é como funciona 

como funciona

\section{OpenOCD}
\label{section:OpenOCD}

explicar para que server e como é ligado. e as portas para cada ferramenta.

Por uma imagem sobre como é ligado o openocd

\section{Protocolos}
\label{section:protocolos}

\subsection{Uart}

falar da uart como funciona o protocolo

\subsection{GPIO}

falar um pouco do GPIO como funciona

\subsection{I2C}

Falar do I2C como funciona os pinos que tem para que servem

falar e imagem so start, stop bit e dos dados

\subsection{SPI}

falar do SPI. como funcionar quais s\~ao os pinos que tem. para que serve cada um
