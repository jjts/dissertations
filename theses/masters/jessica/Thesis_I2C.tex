%%%%%%%%%%%%%%%%%%%%%%%%%%%%%%%%%%%%%%%%%%%%%%%%%%%%%%%%%%%%%%%%%%%%%%%%
%                                                                      %
%     File: Thesis_I2C.tex                        %
%     Tex Master: Thesis.tex                                %
%                                                                      %
%     Author: Carlos A. Rodrigues                         %
%     Last modified : 15 Abril 2013                         %
%                                                                      %
%%%%%%%%%%%%%%%%%%%%%%%%%%%%%%%%%%%%%%%%%%%%%%%%%%%%%%%%%%%%%%%%%%%%%%%

\chapter{I2C}
\label{chapter:i2c}


Explicar o protocolo i2c. como funciona e as linhas que tem. explicar em que acasos é mais usado

por uma imagem os um master e slaves para mostrar como s\~ao as liga\c{c}ões

\section{Mestre I2C}

dizer como estava o master a I2C que n\~ao conseguia enviar ao receber dados.

por um diagrama de blocos de como circula a informa\c{c}\~ao dentro do meu modelo de SPI. fazer uma descri\c{c}\~ao do funcionamento com uma descri\c{c}ao do fluxo de dados.

por a tabela de sinais da interface de i2c (sinais de entrada e saida)

tabela de registos (endere\c{c}os do registos).

explicar os 2 modos de funcionamento de esrita e de leitura que tem 2 modos.

\section{escravo I2C}

processo de constru\c{c}\~ao do escravo 

por um diagrama de blocos de como circula a informa\c{c}\~ao dentro do meu modelo de SPI. fazer uma descri\c{c}\~ao do funcionamento com uma descri\c{c}ao do fluxo de dados.

por a tabela de sinais da interface de i2c (sinais de entrada e saida)

tabela de registos (endere\c{c}os do registos).

explicar os 2 modos de funcionamento de esrita e de leitura que tem 2 modos.
