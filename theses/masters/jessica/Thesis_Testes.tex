%%%%%%%%%%%%%%%%%%%%%%%%%%%%%%%%%%%%%%%%%%%%%%%%%%%%%%%%%%%%%%%%%%%%%%%%
%                                                                      %
%     File: Thesis_Bootrom.tex                        %
%     Tex Master: Thesis.tex                                %
%                                                                      %
%     Author: Carlos A. Rodrigues                         %
%     Last modified : 15 Abril 2013                         %
%                                                                      %
%%%%%%%%%%%%%%%%%%%%%%%%%%%%%%%%%%%%%%%%%%%%%%%%%%%%%%%%%%%%%%%%%%%%%%%

\chapter{Testes de funcionamento}
\label{chapter:teste}

explicar para que quero os testes.

como foi implementado no Orpsoc. 

adicionar uma imagem explicativa como foi implementado.

falar sobre as flags do teste

como posso ver os resultados dos teste

\section{adicionar novos testes}

fazer novos testes.

\section{Testes desenvolvidos}

para mostrar que o sistema funciona corremante foi desenvolvido estes testes 

\subsection{SPI}

explicar como é testado

\subsubsection{simulador verilator}

\subsubsection{Board VGA}

\subsection{I2C}

explicar como é testado

\subsubsection{simulador verilator}

\subsubsection{Board VGA}

\subsection{Interface FIFO}

explicar como é testado

\subsubsection{simulador verilator}

\subsubsection{Board VGA}

\subsection{Bootrom}

explicar como é testado

\subsubsection{simulador verilator}

\subsubsection{Board VGA}
