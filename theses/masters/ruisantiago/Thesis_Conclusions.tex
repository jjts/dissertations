%%%%%%%%%%%%%%%%%%%%%%%%%%%%%%%%%%%%%%%%%%%%%%%%%%%%%%%%%%%%%%%%%%%%%%%%
%                                                                      %
%     File: Thesis_Conclusions.tex                                     %
%     Tex Master: Thesis.tex                                           %
%                                                                      %
%     Author: Rui Santiago                         			   %
%     Last modified : 2 Junho 2015                         		   %
%                                                                      %
%%%%%%%%%%%%%%%%%%%%%%%%%%%%%%%%%%%%%%%%%%%%%%%%%%%%%%%%%%%%%%%%%%%%%%%%

\chapter{Conclusão}
\label{chapter:conclusao}

Os objectivos para esta introdução à tese foram cumpridos com
sucesso. Houve dificuldade tanto a achar a solução a usar para fazer o
compilador como a entender certos pormenores e limitações do {\it
  hardware}. Nesta fase não foi feito nenhum troço do compilador.


% ----------------------------------------------------------------------
\section{Trabalho feito}
\label{section:achievements}

%The major achievements of the present work...
Nesta fase conseguiu-se fazer o estudo da arquitectura do Versat, com
o objectivo de fazer o compilador e saber as limitações do {\it
  hardware}.  Conseguiu-se também achar maneiras diferentes para fazer o
compilador. Fez-se exemplos diferentes de código para se perceber as
vantagens e desvantagens de cada abordagem de compilação.

Devido à presença do controlador do Versat, que é uma máquina de
acumulador convencional, não se consegue excluir os aspectos comuns
das linguagens de programação tais como as expressões condicionais e
os ciclos. Por outro lado, observou-se que o Data Engine consegue ser
modelado por classes e a sua programação feita apenas chamando métodos
dessas classes.


% ----------------------------------------------------------------------
\section{Trabalho Futuro}
\label{section:futuro}

Na próxima fase será estuda a forma de unificar a programação do
controlador do Versat com a programação do Data Engine. Será feita a
construção e respectivo teste do compilador. Serão feitos exemplos de
código na linguagem construída com o objectivo de testar o
funcionamento e a eficiência do compilador.


\cleardoublepage

