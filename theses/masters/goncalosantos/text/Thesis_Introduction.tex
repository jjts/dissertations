%%%%%%%%%%%%%%%%%%%%%%%%%%%%%%%%%%%%%%%%%%%%%%%%%%%%%%%%%%%%%%%%%%%%%%%%
%                                                                      %
%     File: Thesis_Introduction.tex                                    %
%     Tex Master: Thesis.tex                                           %
%                                                                      %
%     Author: Gonçalo Santos                                           %
%     Last modified : 20 Oct 2018                                      %
%                                                                      %
%%%%%%%%%%%%%%%%%%%%%%%%%%%%%%%%%%%%%%%%%%%%%%%%%%%%%%%%%%%%%%%%%%%%%%%%

\chapter{Introduction}
\label{chapter:introducao}

In the past decades there has been a great focus in reconfigurable computing.
Due to the fact that these systems can change their architecture dynamically to
be best suit to the task being executed, they can greatly accelerate the
execution of applications mainly in the fields reigned by embedded systems, like
telecommunications, multimedia, etc~\cite{Carta06,Liu15,Lee18TACO}.  Many of
these embedded applications need to have low power consumption, and reduced
silicon area while still having a high number of computations done per second.

{\it Versat} is one of these architectures capable of changing component
functions and interconnections dynamically~\cite{Lopes16}.  This system uses a
Data Engine in order to execute code loops. {\it Versat}, like architectures of
this type, is not designed to run efficiently when the code has a great amount
of different instructions, but rather when the code revolves around loops and
repeated instructions in general.  To do the more eclectic code there is a
controller that also manages {\it Versat}, making it run and taking care of the
reconfigurations.

For an architecture like this there is not only the way the components operate
themselves but also the coding and execution management of the system.  For the
coding aspect of these systems the best solution would be to have a compiler for
the system that allowed the effective use of the components while being able to
be coded in a language easily understood by humans~\cite{Koeplinger18PLDI}.
Unfortunately there is not still an universal, or even close to universal,
compiler for these types of architectures. Even so, a compiler for this
architecture was made~\cite{Santiago2016} but it still has some flaws that need
to be resolved.

So, the point of this work is to replace the existing {\it Versat} compiler
solving the issues that the previous one still has. For this, some different options
will be discussed, keeping in mind that this is not a traditional CPU
architecture. This means that there is need to study compilers in general as
well as the architecture in question.

%Arquitectures often consist of tens to hundreds of FUs that are capable of executiong word, or even subword level operations instead of bit-level ones, which are usually found in FPGAs.
%Many embedded applications require high throughput, meaning that a large number of computations needs to be performed every second.
%At the same time, the power consumption of battery-operated devices needs to be minimized to increase their autonomy.


% ---------------------------------------------------------------------
\section{Context}
\label{section:context}
	
{\it Coarse Grain Reconfigurable Arrays} ({\sc CGRAs}) are reconfigurable
architectures that have been getting a lot of attention in the past decades.
These architectures are usually used in embedded systems because they are
designed to fit very specific needs. This means that even though they are less
flexible by nature they also have less delay, area, configuration time and
consume less power compared to Field Programmable Gate Arrays ({\sc FPGAs}), the
most commonly used reconfigurable architecture. As such CGRAs are meant to work
as accelerators.

These architectures can either be configured a single time per run of an
application or configured during the execution of the application. This means
that {\sc CGRAs} can either have static or dynamic configurations, respectively.
Static configurations are less flexible while dynamic configurations have to
account for the time spent doing the reconfiguration itself.

A compiler for an architecture like this has to take into account these
differences between a traditional architecture, and a reconfigurable one.
%to which most are designed for.
However, the basic structure and behavior of the compiler remains unchanged.
%Most of the basis a compiler needs to have need to still be present however.

%Less flexibility but also less delay, area, power and configuration time.
%The target applications for these architectures (telecomunications, multimedia eletronics, ...) often spend most of time executing a few {\it time-critical code segments} with well defined characteristics, which means that performance on these applications may be improved by adding a hardware accelerator to the system and mapping these {\it critical} sections to it. Not only that but these sections usually contain {\it loops} and are also proned to have inherent paralelism.
%Compute intensive kernels are perfect to be mapped to CGRAs, since these can take advantage of the high instruction level parallelism (because of the multiple FUs organized in a mesh).
%By neglecting non-loop code or outer-loop code that is assumed to be executed on other cores, CGRAs can exploit ILP in loops.
%CGRAs provide more FUs than typical VLIWs


% ----------------------------------------------------------------------
\section{Motivation}
\label{section:motiva}

High-level languages, like {\bf C}/{\bf C++} or Java, allow for program
development without the knowledge of the inner workings of a given
processor. Furthermore, the development is faster and programs are easier to
debug or test than using an assembly language for a specific processor. {\it
  Versat} is programmable in assembly using the {\it picoVersat} instruction
set. Using the assembler program available for program development is a
cumbersome form to port existing algorithms to the {\it Versat} platform, most
of them originally written in {\bf C} or {\bf C++}.  A compiler is essential for
the test and assessment of the {\it Versat} platform capabilities. A first
prototype compiler, developed by Rui Santiago~\cite{Santiago2016}, exists and
was used for the development of some examples. However, the compiler is not only
not very practical still, but also quite primitive. It supports a limited subset
of the {\bf C} language with {\bf C++} alike method invocation, with no
variables, functions or structures. All programming uses the {\it picoVersat}
register names to hold values. Furthermore, the only {\bf C++} syntax used is
the method invocation from object (object.method), but this being the only
difference from {\bf C}, and since there is not the option to create multiple
instances of these objects (they act as variables), this syntax could simply be
replaced with a {\bf C} compatible syntax. This would allow for the use of a
simpler compiler, since in reality none of the extra functionalities of {\bf
  C++} in relation to {\bf C} are even supposed to be used in this architecture.

\section{Objectives}
\label{section:objectivo}

The objective of this work is supersede the existing compiler Versat compiler
improving the following aspects:
\begin{itemize}
\item Partial reconfiguration: prepare the next {\it Versat} configuration by changing
  only the configuration fields that differ. Currently all configuration bits
  are updated during the reconfiguration, which costs time.
\item Support a variable number of functional units: the current implementation
  assumes a fixed number of functional units, and if the hardware is changed, then
  the compiler needs to change too. This improvement will lift this restriction.
\item Support variable declarations: currently all variables and objects are
  predefined and user defined variables are not supported. This improvement
  will greatly improve programmability.
\item Support function calls: currently all the code must be written in a single
  {\tt main()} function and no other functions can be called from this one. This
  improvement will allow not only for better programming practices, but also to
  make the code more extensible and easy to understand.
%\item Support constants definition:
\end{itemize}

\section{Challenges}
\label{section:desafio}

Building a full {\bf C} compiler on such a bare/simple instruction set requires
that most instructions need to be decomposed.  All {\bf C} language operations
that are not directly supported by the architecture have to be mapped into the
few existing ones.

Having such reduced register support in {\it picoVersat} requires constant
memory accesses, and not having a {\it stack pointer}, or a {\it frame pointer},
makes function calls (in particular recursion) slow.

Detecting, along with regular {\bf C} code, the portions that are meant to be
computed into instructions for {\it Versat} instead of {\it picoVersat} is
difficult, not only because there is a need to identify these "packs" of
instructions, as well as to analyze and to compute these instructions in a
different way.

\section{Document structure}
\label{section:parts}

This document is divided into six chapters.  The next chapter, State of the Art,
introduces the {\it Versat} architecture in its present form, and surveys the
existing {\bf B} and {\bf C} language compiler frameworks that can be adapted
into a full {\it Versat} compiler.  Chapter 3, Architecture, provides
the general architecture of the compiler for the {\it Versat}
platform.  The {\bf lcc} retargetable {\bf C} compiler was selected because
it can be easely extended by providing a new {\it Versat} {\it back-end}.
The following chapter, Development, explores the intrinsics of the
compiler {\it back-end} development and its integration into the {\bf lcc}
framework.  Chapter 5, Results, addresses the compiler functionality, efficiency
and limitations.  The last chapter, Conclusions, sums up the work undertaken.
