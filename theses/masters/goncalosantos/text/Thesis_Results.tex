%%%%%%%%%%%%%%%%%%%%%%%%%%%%%%%%%%%%%%%%%%%%%%%%%%%%%%%%%%%%%%%%%%%%%%%%
%                                                                      %
%     File: Thesis_Results.tex                                         %
%     Tex Master: Thesis.tex                                           %
%                                                                      %
%     Author: Gonçalo Santos                                           %
%     Last modified : 20 Oct 2018                                      %
%                                                                      %
%%%%%%%%%%%%%%%%%%%%%%%%%%%%%%%%%%%%%%%%%%%%%%%%%%%%%%%%%%%%%%%%%%%%%%%%

\chapter{Results}
\label{chapter:results}

The aim of this work is to produce a workable {\bf C} language
compiler for the {\it Versat} architecture using the {\it picoVersat}
instruction set.
The success can be measured by the number of {\bf C} language
constructs that are working properly.
Consequently, testing is of primordial importance, as are the range
of tests used to exercise the compiler.

\section{Functionality}

The compiler, as far the tests were comprehensive,
supports all {\bf C} language integer constructs.
Limitations are listed below.

Since the processor instruction set is reduced, as are the
number of {\bf lcc} terminals to be implemented by the compiler,
the testing of each operation, on its own, is strait
forward.
Testing sequences of such operations may prove more
difficult to test, since different {\bf C} programs
can produce different selection matches.

\section{Testing}

In the test of a compiler, where a small change can affect the
generation of multiple instructions, a good set of
regressive tests is very important.
In order to automate the process, a {\tt test/} directory
was setup.
This directory includes a set of {\bf .c} test files and
the expected output {\bf .out}.

The {\tt Makefile} compiles, executes and compares the new
result with the previously stored result.
All differences in output are printed and can then be analyzed.

Since the output from {\bf iverilog} includes the number
of clocks spent, it is easy to compare whether the changes in the
compiler result in improvements, or in performance degradation.

Some tests are very simple and its output can be easily predicted.
To make testing even simpler, the return value of the {\tt main}
routine is printed, unless the {\sc NORET} environment variables
is defined. Upon return from the {\tt main} routine, the lowest
nibble is printed as an {\sc ASCII} starting at $0$.
This means that values between $10$ and $15$ are printed as
the {\sc ASCII} character at the respective offset, namely the
sequence: \verb|:|, \verb|;|, \verb|<|, \verb|=|, \verb|>|, \verb|?|.

More complex tests can be compiled with {\bf gcc} and executed
to access the expected output.
This, however, can not be performed if the examples include
{\tt asm} calls, since the code can only be executed by the
{\it Versat}, or by the {\bf iverilog} simulator, and not by the
native testbench processor.
%vdb.c

A set of $86$ regression tests is currently being used, ranging
from specific operator testing to complex recursive and iterative
examples. % And the number of tests keeps growing ...

\section{Limitations}\label{limitations}

The {\bf C} language imposes that {\tt sizeof(char)==1}
as does the {\bf lcc} compiler (see~\cite[p.79]{hanson95}).
This works fine as far as {\tt sizeof(char)} can be 32-bits.
However, additionally, the {\bf lcc} assumes through out
the code that $8$ is the number of {\em bit-per-byte}.
If it was a variable, one could set it to $32$.
As it is hardcoded, all address literals
will be truncated to 8-bits ({\tt 8}${}^{\wedge}${\tt ty->size}).
\begin{verbatim}
int *addr = (int*)0x123456;
\end{verbatim}
This can be avoided by setting an integer to the
required value and then assigning it to a pointer.
This works since integer literals are 32-bit wide
and the conversion to pointer, controlled by the
{\it back-end}, does not truncate the value.
\begin{verbatim}
int *addr, value = 0x123456;
addr = (int*)value;
\end{verbatim}
Nevertheless, defining literal pointer is never
a good predictive in virtual memory machines.
In {\it Versat} it is useful to map variables
to specific addresses.

Due to the same reason, a warning message is issued
({\tt shifting an `int' by 12 bits is undefined})
but the code is correctly generated.

% #define LONG_MIN -2147483648
% warning: unsigned operand of unary -
% but 0x80000000 or bellow is OK!
% #  define LONG_MAX	2147483647
% #  define LONG_MIN	(-LONG_MAX - 1L)

In the initial version of {\it picoVersat}, all global
data must be added by {\tt MEM\_BASE=512}.
Since this is performed when addresses are fetched,
static assignments store the unadded value.
Therefore, all accesses must be added by $512$.
Must add {\tt 512} to global pointers in {\tt picoversat-0.0}
\begin{verbatim}
int mem[10], *base = mem;
int main() { base[6+512] = 9; return return mem[6]; }
\end{verbatim}
This can be avoided if assignment is performed during
execution (not at compile time), even if the variable
is global.
\begin{verbatim}
int mem[10];
int main() { int *base = mem; base[6] = 9; return return mem[6]; }
\end{verbatim}

Signed multiplication, division and modulus ({\tt \_mul},
{\tt \_div} and {\tt \_mod}) do not generate carry since
the flags register of {\it picoVersat} is read-only.

The {\bf C} programming languages relies on separate
compilation, where several files are independently
compiled and then linked together.
However, there is no linker in the {\it Versat} system
and the assembler {\bf va} does not support multiple
files.
The solution is to perform linking with {\bf cpp} %%%
include directives.
While in normal {\bf C} the {\tt .c} should be
included, rather compiled, the inclusion of {\bf .h}
as well as {\bf .c} accomplishes the desired result.
Since there is no linking, only multiple inclusion
of files must be avoided.

The {\it Versat} architecture is meant to be used offline
and no form of argument passing to the {\tt main} routine
is available.
Consequently, the stack is initialized at the top.
Therefor, even if the program declares arguments to
the {\tt main} routine ({\tt argc, argv, envp}) they should
never be accessed.
Also, since the system has no memory management unit,
all illegal accesses are silently ignored by the system.
Highly recursive routines that exhaust the stack will have
unpredictable behaviors, since they will begin to overwrite
the top of the code.
Even if it is not the compilers responsibility, it something
that the programmer should be aware, especially when
transitioning from a virtual managed memory system.

Finally, the compiler does not support floating point
data types, since every operation must be supported
by library routines.
This is the case for many android devices, namely
smartphones.
However, the {\it Versat} purpose is to perform
integer arithmetic operations fast and is not aimed at
scientific programming.
The error message {\tt compiler error in \_label--Bad terminal}
is issued by the compiler when it cannot handle a given operation,
namely floating point operations.

\section{Register assignment}

Register assignment in compiler design considers two types of registers:
global registers that hold variable values and scratch registers that hold
temporary values.
The {\bf lcc} compiler defines these registers by setting a mask for each
type of register.
It is up to each {\it back-end} to define the mask values according to processor
capabilities.
For instance, the {\bf sparc} processor defines $4$ sets of $8$ registers:
global, temporary, input and output; where the later two sets replace
the stack for argument passing.
In {\bf i386} all $7$ registers are temporary, while {\bf mips} uses half
for each purpose ($16+16$).

Since the {\it picoVersat} has no specific register assignment, a study
was carried out in order to assess the best balance between global and
scratch registers.
Registers {\tt R0} and {\tt R13} to {\tt R15} are used to communicate with {\it Versat}
and are invisible to the compiler.
The stack is controlled by a {\it stack pointer} ({\tt R12}) and a {\it frame pointer}
({\tt R11}).
The remaining registers ({\tt R1} to {\tt R10}) compose a mask {\tt 7FE}, where
the lowest bit ({\tt R0}) is omitted for register assignment, and the highest
used bit {\tt 400} is {\tt R10}.
The register {\tt R1} is used to return function values and all arguments are
passed on the stack.
At least two registers must be used as scratch for binary operations
temporaries.
The compiler allows the definition of a {\tt tmask} for temporaries and a
{\tt vmask} for variables.

Initially, in run $1$, the experiment uses all registers for temporaries.
Each run adds a variable register at the expense of a temporary, until only
two temporaries remain (run $9$).
Three examples where used: {\tt assign}, {\tt repeating locals} and
{\tt bubble sort}.
The first two represent opposite extremes of register usage, while the last is
a more balanced and realistic example.

The first example uses the {\bf C} language right associative {\it assign} operator
where each new assignment to the variable {\tt a} requires a new temporary register.
%(see Figure~\ref{fig:assign}).

%\begin{figure}
\begin{verbatim}
int f() { return 1; }
int main()
{
  int a;

  a = f() + (a =
      f() + (a =
      f() + (a =
      f() + (a =
      f() + (a =
      f() + (a =
      f() + (a =
      f() + (a =
      f() + (a =
      f() + (a =
      f() + (a =
      f() + (a =
      f() + (a = 1
                  )))))))))))));
  return a;
}
\end{verbatim}
%\caption{}
%\end{figure}

The register usage shows that each assign uses a register $4$ times at
the expense of the return register {\tt R1}.
The best solution, represented by the lowest clock count,
is to use only two temporaries, since more variables imply more stack ({\tt R12}),
saves and restores between each call to the function {\bf f}.

%\begin{table}
\begin{center}
{\small
\begin{tabular}{r|r|r|r|r|r|r|r|r|r|r|r|r|r|r|r|r}
run&vars&vmask&tmask&R1&R2&R3&R4&R5&R6&R7&R8&R9&R10&R11&R12&clks\\\hline
1&0&000&7FE&21&4&4&4&4&4&4&4&6&44&27&82&677\\
2&1&400&3FE&23&4&4&4&4&4&4&6&42&17&14&82&638\\
3&2&600&1FE&25&4&4&4&4&4&6&42&0&17&16&77&633\\
4&3&700&0FE&27&4&4&4&4&6&42&0&0&17&18&72&628\\
5&4&780&07E&29&4&4&4&6&42&0&0&0&17&20&67&623\\
6&5&7C0&03E&31&4&4&6&42&0&0&0&0&17&22&62&618\\
7&6&7E0&01E&33&4&6&42&0&0&0&0&0&17&24&57&613\\
8&7&7F0&00E&35&6&42&0&0&0&0&0&0&17&26&52&608\\
9&8&7F8&006&39&42&0&0&0&0&0&0&0&17&28&47&603\\
\end{tabular}
}
\end{center}
%\caption{}
%\end{table}
\vspace*{5mm}

The second example uses lots of repeating local variables so that each one is assigned
a register, for its uses from the first to last line, if one is available.
%(see Figure~\ref{fig:locals}).

%\begin{figure}
\begin{verbatim}
int func(int a, int b, int c, int d, int e, int f, int g, int h, int i, int j, int k) {
    a = a + b - c - d - e + f - g + h + i + j + k;
    b = a - b + c + d - e - f + g - h + i - j - k;
    c = a + b - c - d + e + f + g - h + i + j - k;
    d = a - b - c + d - e + f - g + h - i - j - k;
    e = a + b + c - d - e - f - g + h - i + j - k;
    f = a - b - c + d - e + f + g - h - i - j + k;
    g = a + b - c - d + e + f + g - h + i + j - k;
    h = a - b + c + d - e - f - g + h + i - j - k;
    i = a + b - c - d - e + f + g + h + i + j - k;
    j = a - b - c + d - e + f + g - h - i - j - k;
    k = a + b + c - d + e - f - g - h - i + j + k;
    return a + b + c + d + e + f - g + h - i + j - k;
}

int main() {
    return func(10, 9, 8, 7, 6, 5, 4, 3, 2, 1, 0);
}
\end{verbatim}
%\caption{}
%\end{figure}

As expected, the best solution is to use highest of temporaries in order
to reduce frame pointer ({\tt R11}) accesses to stack saved values.

%\begin{table}
\begin{center}
{\small
\begin{tabular}{r|r|r|r|r|r|r|r|r|r|r|r|r|r|r|r|r}
run&vars&vmask&tmask&R1&R2&R3&R4&R5&R6&R7&R8&R9&R10&R11&R12&clks\\\hline
1&0&000&7FE&45&27&12&13&48&46&40&45&54&62&68&56&956\\
2&1&400&3FE&53&31&13&54&46&40&47&54&62&0&78&51&1001\\
3&2&600&1FE&61&35&52&46&48&49&56&62&0&0&89&46&1051\\
4&3&700&0FE&89&39&59&63&51&56&62&0&0&0&101&41&1106\\
5&4&780&07E&109&43&75&49&74&79&0&0&0&0&113&36&1161\\
6&5&7C0&03E&146&45&84&58&105&0&0&0&0&0&124&31&1211\\
7&6&7E0&01E&156&88&112&94&0&0&0&0&0&0&138&26&1278\\
8&7&7F0&00E&171&117&176&0&0&0&0&0&0&0&154&21&1356\\
9&8&7F8&006&231&249&0&0&0&0&0&0&0&0&172&16&1447\\
\end{tabular}
}
\end{center}
%\caption{}
%\end{table}
\vspace*{5mm}

The last example, the {\it bubble sort}, uses a mixture temporaries and variable
reuses. %(see Figure~\ref{fig:bubble}).

%\begin{figure}
\begin{verbatim}
#include "printi.h"

int bubble(int list[], int n) {
    int c, d, t, swap, cnt = 0;

    for (c = 0; c < n - 1; c++) {
        for (swap = 0, d = n - 1; d > c; d--)
            if (list[d - 1] > list[d]) {    /* Swapping */
                swap++;
                t = list[d];
                list[d] = list[d - 1];
                list[d - 1] = t;
            }
        if (!swap)
            break;
        cnt++;
    }
    return cnt;
}

int v[] = { 7, 4, 9, 6, 2, 1, 3, 5, 8, 0 };

int main() {
    int i, size = sizeof(v) / sizeof(v[0]), cnt = bubble(v, size);
    for (i = 0; i < size; i++) {
        putchar(v[i] + '0');
        putchar(' ');
    }
    printi(cnt, 10);
    putchar('\n');
    return 0;
}
\end{verbatim}
%\caption{}
%\end{figure}

This example exploits the tradeoff between global and temporary register
usage.
In the first runs the compiler is unable to use all temporaries.
In the last runs some variable registers are left unassigned and the number
of required execution clocks rises again.

%\begin{table}
\begin{center}
{\small
\begin{tabular}{r|r|r|r|r|r|r|r|r|r|r|r|r|r|r|r|r}
run&vars&vmask&tmask&R1&R2&R3&R4&R5&R6&R7&R8&R9&R10&R11&R12&clks\\\hline
1&0&000&7FE&2&0&0&0&0&4&5&15&30&45&31&36&9855\\
2&1&400&3FE&2&0&0&0&0&4&7&29&41&11&24&36&8193\\
3&2&600&1FE&2&0&0&0&4&5&27&37&7&11&19&41&7714\\
4&3&700&0FE&2&0&0&0&5&25&37&4&7&11&17&41&7399\\
5&4&780&07E&2&0&0&5&19&37&6&4&7&11&13&46&7022\\
6&5&7C0&03E&2&0&5&19&33&6&6&4&7&11&9&49&6949\\
7&6&7E0&01E&2&5&19&33&0&6&6&4&7&11&9&49&6949\\
8&7&7F0&00E&5&19&33&0&0&6&6&4&7&11&9&44&6921\\
9&8&7F8&006&21&28&0&0&0&6&6&4&7&11&13&41&7492\\
\end{tabular}
}
\end{center}
%\caption{}
%\end{table}
\vspace*{5mm}

Based on experience with the examples above, a balanced approach should work best
in most cases.
Therefor, the first five registers, {\tt R1} to {\tt R5}, are used as temporaries
({\tt tmask=0x003E}) and the remaining five, {\tt R6} to {\tt R10}, are used as
variables ({\tt vmask=0x07C0}).

\section{Efficiency considerations}

Calls are very expensive operations for any processor.
{\it Intel Inc.} has made a significant effort over the year to address this
problem.
In the last years, its high end processors provide faster {\it calls} than
{\it jumps} at the expense of higher transistor count. %ref!
In a processor like {\it picoVersat}, the problem is magnified since
no stack specific registers or opcodes are available.

A function call in the {\bf C} programming language requires:
\begin{enumerate} \itemsep0em 
\item {\bf argument passing} by pushing values to the stack;
\item {\bf calling} the desired routine;
\item {\bf saving used registers} before the routine destroys its values;
\item {\bf frame pointer} saving to access arguments and locals;
\item {\bf allocate space} for local variables;
\item actually performing the routine operations;
\item {\bf restoring frame pointer} of the previous routine;
\item {\bf restoring used registers} previous values;
\item {\bf returning} to the calling routine;
\item {\bf removing arguments} from stack.
\end{enumerate}
The present compiler detects when a routine accesses no arguments or locals
and does not emit frame pointer code. So, if a routine only uses global
variables, the call becomes a bit more efficient.
Some of the tests used become upto 5\% faster by removing the frame
pointer in routines where it not needed.

As any routine can be called many times, even recursively, the compiler
must save, at the beginning, and restore, at the end,
all the registers the routine uses.
This means that, at the start of the program, the {\tt main} routine will spill
all registers it will use, although they have no defined value.
Such procedure is required since the routine may be recursively called.
However, in most cases, the {\tt main} routine is only invoked once, at
the start of the program.
The {\sc NOSAV} environment variable can be set if the {\tt main}
routine is not used recursively and no registers will saved by the compiler.
This special hack can be dangerous to use, but it makes {\tt main} based
programs more efficient.

The {\it picoVersat} controls {\it Versat} by setting specific values to
predefined memory positions.
The use of a routine to perform such a task is a very
expensive way to change memory positions, either through {\tt asm}
directives or standard {\bf C} code, as the tests {\tt set.c} and
{\tt setvar.c} show, respectively.
Memory values can be efficiently changed by assigning to a pointer
{\tt *addr=val} (see Limitations, above).

During this work, the {\it picoVersat} evolved. The use of a single
memory, for program code and data, removed the need for a {\tt addi MEM\_BASE}
instruction for each variable load and store, resulting in a 5\% improvement
over all the regression test in use, at the time. % 17022/16213 pico-0

Finally, the compiler some times generates a register read after the
same register was written by another instruction selection.
At least, the read can be suppressed, but {\bf lcc} provides no
peephole optimizer for final code cleanup.

\section{Compiler instalation}

The compiler itself, {\bf lcc}, can be invoqued directly with the
{\tt -target=versat} option, as long as the input file has already
been preprocessed ({\bf cpp}).
The compiler output is a {\it picoVersat} assembly, that can then
fed to the {\it versat} assembler ({\bf va}).

However, the complete compilation process, from {\bf C} language
source file to {\bf iverilog} simulation executable, can be
integrated as in a standard high-level compiler.

Section~\ref{app:integ} describes the requirements for such an integration.
The compiler {\tt Makefile}s, in the main and {\tt versat/} directories,
can be used to provide the instalation of all required files
for a complete development environment.
By default, without any changes to the {\tt Makefile}s, the
compiler development environment is placed under the
{\tt /usr/local/versat} directory.

The default directories for the compiler installation
({\tt make install}) are predefined as
{\tt /usr/local/versat/lcc} for the compiler files
({\tt lcc}, {\tt cpp}, {\tt rcc}, {\tt va}, and
{\tt xdict.json}), and can be redefined at compile
time or using the {\sc LCCDIR} environment variable
at runtime. The {\it picoVersat} {\tt rtl/} files
({\tt include/}, {\tt src/}, and {\tt testbench/})
should be copied to {\tt /usr/local/versat/pico}
(defined at compile time).
Also the {\bf iverilog} compiler is defined at
compile time as residing in {\tt /usr/local/bin/}.

The structure of the installed files,
in the current version is:
\begin{Verbatim}[baselinestretch=1.0]
/usr/local/versat/lcc/lcc
/usr/local/versat/lcc/cpp
/usr/local/versat/lcc/rcc
/usr/local/versat/lcc/va
/usr/local/versat/lcc/xdict.json
/usr/local/versat/lcc/include/strlen.h
/usr/local/versat/lcc/include/umod.h
/usr/local/versat/lcc/include/errno.h
/usr/local/versat/lcc/include/malloc.h
/usr/local/versat/lcc/include/umul.h
/usr/local/versat/lcc/include/itoa.h
/usr/local/versat/lcc/include/Makefile
/usr/local/versat/lcc/include/stdarg.h
/usr/local/versat/lcc/include/mem_ends.h
/usr/local/versat/lcc/include/xdict.h
/usr/local/versat/lcc/include/atoi.h
/usr/local/versat/lcc/include/puts.h
/usr/local/versat/lcc/include/dma.h
/usr/local/versat/lcc/include/versat.h
/usr/local/versat/lcc/include/ends.h
/usr/local/versat/lcc/include/mul.h
/usr/local/versat/lcc/include/xdictinc
/usr/local/versat/lcc/include/div.h
/usr/local/versat/lcc/include/ends.cbc
/usr/local/versat/lcc/include/udiv.h
/usr/local/versat/lcc/include/printf.h
/usr/local/versat/lcc/include/mod.h
/usr/local/versat/lcc/include/alloca.h
/usr/local/versat/lcc/include/printi.h
/usr/local/versat/lcc/include/gnuc.h
/usr/local/versat/lcc/include/putchar.h
/usr/local/versat/lcc/include/assign.h
/usr/local/versat/pico/testbench/sim_xtop.cpp
/usr/local/versat/pico/testbench/xtop_tb.v
/usr/local/versat/pico/include/xdefs.vh
/usr/local/versat/pico/src/xaddr_decoder.v
/usr/local/versat/pico/src/xctrl.v
/usr/local/versat/pico/src/xram.v
/usr/local/versat/pico/src/xregf.v
/usr/local/versat/pico/src/xcprint.v
/usr/local/versat/pico/src/xtop.v
\end{Verbatim}

After adding the {\tt /usr/local/versat/lcc} directory to
the {\sc PATH} environment variable, an executable example
can be produced with the command:\\
{\tt lcc example.c -o example}

The example can then be run with:\\
{\tt ./example}

Please note that the {\it versat} memory dump {\tt .hex}
file is stored in the {\tt /tmp} directory.

\cleardoublepage
