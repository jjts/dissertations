%%%%%%%%%%%%%%%%%%%%%%%%%%%%%%%%%%%%%%%%%%%%%%%%%%%%%%%%%%%%%%%%%%%%%%%%
%                                                                      %
%     File: Thesis_Resumo.tex                                          %
%     Tex Master: Thesis.tex                                           %
%                                                                      %
%     Author: Gonçalo Santos                                           %
%     Last modified : 20 Oct 2018                                      %
%                                                                      %
%%%%%%%%%%%%%%%%%%%%%%%%%%%%%%%%%%%%%%%%%%%%%%%%%%%%%%%%%%%%%%%%%%%%%%%%

\section*{Resumo}

% Add entry in the table of contents as section
\addcontentsline{toc}{section}{Resumo}

% Inserir o resumo em Portugu\^{e}s aqui com o máximo de 250 palavras e acompanhado de 4 a 6 palavras-chave...
Nos últimos anos, a computação reconfigurável tem recebido grande atenção,
pois permite mudar a arquitetura dinamicamente.
{\it Versat} é uma dessas arquiteturas reconfiguráveis.
O objetivo deste trabalho é fornecer um compilador da linguagem
{\bf C} para o {\it picoVersat}, o controlador do {\it Versat}.
O {\it picoVersat} possui um conjunto muito reduzido de instruções,
realizando cálculos simples e controlando os subsistemas a ele ligados. %61
Foi escolhido o compilador {\bf lcc} porque permite múltiplos processadores
alvo, está bem documentado e utiliza uma ferramenta de
seleção das instruções, facilitando o processo de geração de código. %88
Cada processador alvo do {\bf lcc} é configurado por uma estrutura
que parameteriza o gerador alvo.
A reserva de registos pode ajustar-se às caraterísticas de cada processador
e uma árvore gramatical permite descrever a maioria das operações,
associando-lhes um custo. %127
Consegue-se a otimização da seleção de instruções fornecendo diferentes
combinações de árvores gramaticais com custos distintos.
As instruções mais complexas podem ser codificadas manualmente, tal como o
registo de ativação das funções. %160
Adicionou-se ao {\bf lcc} a geração direta de instruções em {\it assembly},
pois não é uma função {\sc ANSI} {\bf C}. % 177
Como o {\it Versat} é configurado através da escrita em posições específicas
de memória, o seu controlo realiza-se com simples instruções de
atribuição em {\bf C}. %201
Foram realizados testes extensivos ao compilador num ambiente de
desenvolvimento integrado.
O compilador realizado permite controlar o {\it Versat} numa linguagem
de alto-nível.
O número reduzido de instruções disponíveis no {\it picoVersat} implica
que as operações na pilha, como manipulação de argumentos e funções,
resulta em longas sequências de instruções. %249

\vfill

\textbf{\Large Palavras-chave:} Versat, CGRA, picoVersat, compilador de C, geração de código com lcc.

\cleardoublepage

