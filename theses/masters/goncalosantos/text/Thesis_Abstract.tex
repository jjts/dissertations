%%%%%%%%%%%%%%%%%%%%%%%%%%%%%%%%%%%%%%%%%%%%%%%%%%%%%%%%%%%%%%%%%%%%%%%%
%                                                                      %
%     File: Thesis_Abstract.tex                                        %
%     Tex Master: Thesis.tex                                           %
%                                                                      %
%     Author: Gonçalo Santos                                           %
%     Last modified : 20 Oct 2018                                      %
%                                                                      %
%%%%%%%%%%%%%%%%%%%%%%%%%%%%%%%%%%%%%%%%%%%%%%%%%%%%%%%%%%%%%%%%%%%%%%%%

\section*{Abstract}

% Add entry in the table of contents as section
\addcontentsline{toc}{section}{Abstract}

%Insert your abstract here with a maximum of 250 words, followed by 4 to 6 keywords...

% Versat is CGRA so supports partial reconfiguration
% picoVersat is the reconfigurator and performs auxiliary tasks
% C-language compiler instead of specific picoVersat assembler
% lcc back-end: instr-selection, register assignment, code emiting, asm-support
% Versat FU control through regular C-language structures and constructs.

Reconfigurable computing has had a great focus in the past decades as it
promises to combine the performance of dedicated hardware with the flexibility
of software.  {\it Versat} is a Coarse Grained Reconfigurable Array architecture
capable of dynamic and partial reconfiguration. The purpose of this work is to
provide a full {\bf C} language compiler for {\it picoVersat}, the {\it Versat}
hardware controller. {\it PicoVersat} has a minimalist instruction
set, can be programmed in its assembly, and is used to perform simple
calculations and reconfiguring Versat. % 81

The {\bf lcc} framework was chosen because it is a retargetable compiler, well
documented and uses a code selection tool to help the code generation
process. Each {\bf lcc} {\it back-end} is configured through a structure that
parameterizes the {\it back-end} code generator.  Instruction selection
optimization is achieved by providing different tree grammar combination of
different costs. Complex instructions can be coded manually as well as function
activation records. Assembly in-lining through the {\tt asm} routine was added
to the compiler {\it front-end}, since it is not {\sc ANSI} {\bf C}. % 83
The {\it Versat} configuration is memory mapped, thus each functional unit
can be configured by writing to specific memory addresses with an ordinary
{\bf C} language assignment instruction. The compiler was integrated
into a compilation framework, and extensive testing was performed. % 39

The resulting compiler allows all benefits of the {\bf C} high-level language.
Low level assembly instructions are still possible through the {\tt asm} directive.
The minimalist {\it picoVersat} instruction set results in long code sequencies,
specially for stack operations like argument or function manipulation. % 41

\vfill

\textbf{\Large Keywords:} Versat, CGRA, picoVersat, C-compiler, lcc {\it back-end}.

\cleardoublepage

