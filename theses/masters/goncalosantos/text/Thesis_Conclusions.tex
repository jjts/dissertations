%%%%%%%%%%%%%%%%%%%%%%%%%%%%%%%%%%%%%%%%%%%%%%%%%%%%%%%%%%%%%%%%%%%%%%%%
%                                                                      %
%     File: Thesis_Conclusions.tex                                     %
%     Tex Master: Thesis.tex                                           %
%                                                                      %
%     Author: Gonçalo Santos                                           %
%     Last modified : 20 Oct 2018                                      %
%                                                                      %
%%%%%%%%%%%%%%%%%%%%%%%%%%%%%%%%%%%%%%%%%%%%%%%%%%%%%%%%%%%%%%%%%%%%%%%%

\chapter{Conclusion}
\label{chapter:conclusao}

% In this thesis we set out to develop a
The development of a full compiler for the {\it Versat} architecture is a
complex task. As any compiler, the main purpose is to provide code translation
from a high-level language, like the {\bf C} programming language, to the
assembly language accepted by the {\it Versat} architecture.  Although a simple
compiler exists, it only provides for very simple language constructs.  It does
not support variables or declarations, only the processor registers are
available.  Also, no functions or structures are supported, making the algorithm
development difficult.  This work aims at providing a complete {\bf C} language
compiler for the {\it Versat} architecture.

The first step was to decide whether to improve the existing compiler or to
develop a new compiler. Due to the limitations of the existing compiler, the
lack of documentation or support, it was decided to develop a new compiler
altogether. However, there are quite a few compiler frameworks, namely for the
{\bf C} language, that can be used to build a compiler for a new architecture.
Although they do not provide out of the box support a processor like
{\it Versat}, they are retargetable compilers, i.e., they offer support for
many different processors.
After an analysis of some of the more widespread retargetable {\bf
  C} compilers, the {\bf lcc} compiler framework was selected. It provided a
code generation selection tool to ease the code generation process, and select
the lowest cost instruction sequence. The compiler was supported by good
documentation and examples for the most widespread high-end processors.

The bulk of the work is developed in a {\it back-end} file that describes
the capabilities of the {\it picoVersat} assembler, from the point of view of
a {\bf C} language compiler.
The {\it picoVersat} {\it back-end} is configured through a structure that
parameterizes the code generator, and includes a set of routine pointers to
specific {\it back-end} procedures.
These procedures handle specific parts of the code generation, such as the
preamble or the activation record for each user defined routine.
However, most operations are described by a tree grammar.
The tree grammar describes most target processor operations with an associated
cost. Instruction selection optimization is achieved by providing different
tree grammar combination of different costs.

The {\tt asm} extension was added to {\bf lcc} to provide direct control
over {\it Versat} and {\it picoVersat}.
This extension required changes to the compiler {\it front-end}, since
{\bf lcc} is an {\sc ANSI} compiler.
The {\tt asm} extension was initially introduced by {\bf gcc}, but has
been added to other compilers, although it is not standard {\bf C}.

The register assignment can be tailored to each processor characteristics.
For {\it picoVersat}, different combinations of temporary ({\tt tmask})
and variable ({\tt vmask}) register assignment combinations were tested.
The register usage and total execution clock time was measured for a set
examples with different register requirements.
A balanced version, with almost the same number of temporary and variable
register was selected.


% ----------------------------------------------------------------------
\section{Achievements}
\label{section:achievements}
%The major achievements of the present work...

%Please fix
%1. cgra state-of-the art
%2. Versat review
%3. compiler state-of-art
%4. Sketch of the compiler to be developed (after chapter 3 is fixed)
The present work introduces the {\sc CGRA} concept and analyses the {\it Versat}
accelerator and its {\it picoVersat} controller.  It highlights the {\it Versat}
characteristics from a programming point of view, so that a compiler can be
developed.  It introduces the basic compiler development techniques and surveys
existing retargetable {\bf C} language compilers.  After a comparative analysis,
a {\it Versat} compiler is introduced as an evolution of an existing {\bf C}
language compiler.
% lcc Back-end
% instruction selection and specific coding of some instructions (requiring loops or labels)
% register assignment and stack/frame pointers
% asm in lcc

The development work first addressed the support for {\it picoVersat} as an
{\bf lcc} compiler {\it back-end}.
Integration of the compiler with {\bf lcc}'s own preprocessor ({\bf cpp}),
the {\it Versat} assembler ({\bf va}) and the {\it Verilog} simulator
({\bf iverilog}) was added for a smooth compilation of examples from source
to execution.
A large set of regression tests ensure that future changes do not compromise
existing functionality.
Then mechanisms were added for the control of the {\it Versat} {\sc CGRA},
using {\bf C} language structures and assembler macros.

The resulting compiler provides all the requirements initially set out for
this work as referred in section~\ref{section:objectivo}.

% ----------------------------------------------------------------------
\section{Future Work}
\label{section:futuro}

As previously concluded, the task to be undertaken required the development of
the compiler for the {\it Versat} architecture using the existing retargetable
{\bf lcc} compiler framework for the {\bf C} language.  Future work should
essentially address some of the limitations referred in
section~\ref{limitations}, namely {\it bits-per-byte}, wide integers and
floating point numbers.

Since {\it Versat} only manipulates 32-bit quantities and the {\bf C} language
imposes that {\tt sizeof(char)==1}, the usual 8 {\it bits-per-byte} of most
compilers, including {\bf lcc}, do not hold true in the {\it Versat}
architecture. To allow {\tt sizeof(char)} to be 32-bits wide, significant
changes had to be made to the core of the {\bf lcc} compiler. These changes were
not performed, resulting in the limitation, referred in
section~\ref{limitations}, that a literal must first be assigned to an integer
variable and only then this variable can be assigned to an integer pointer,
since the conversion does not truncate the original literal value.
Additionally, shift operations emit a warning message when shifted by a literal
for more than 8 bits.

Support for 64-bit wide integers, known as {\tt long long} in {\bf C}, can be
added. However all operations must be performed by software routines, since {\it
  picoVersat} only handles 32-bit integers. Nevertheless, this requires changes
to the compiler, since even basic operations, like addition or subtraction, must
emit function calls to the respective software routines, instead of the {\it
  picoVersat} assembly instructions.  As for 64-bit wide integers, floating
point support would require the same approach.  Until then, the compiler emits
an error message whenever those data types are used in the programs being
compiled.

\cleardoublepage

