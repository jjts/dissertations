\documentclass{IEEEconfA4}

\usepackage[latin1]{inputenc}
\usepackage[dvips]{graphicx}
\usepackage{epsfig}
\usepackage{amsfonts, color}
%\usepackage{subfigure}
\usepackage{eurosym}

%------------------------------------------------------------------------------
\begin{document}

\title{Building Reconfigurable Systems Using Open Source Components}

% For authors with the same affiliation
%\author{Author1, Author2, ...\\
%  \begin{affiliation}
%    Affiliation\\
%    \email{\{author1, author2, ...\} @email.org}
%  \end{affiliation}}

% For authors with different affiliations
\author{Jos\'e T. de Sousa, Carlos A. Rodrigues, Nuno Barreiro, Jo\~ao C. Fernandes\\
  \begin{affiliation}
    INESC-ID Lisboa\\
    \email{\{jose.desousa,carlos.rodrigues,nuno.barreiro,joao.fernandes\}@inesc-id.pt}
  \end{affiliation}}

\date{}

\maketitle

%------------------------------------------------------------------------------

\begin{abstract}
This paper presents a methodology for creating reconfigurable systems on chip (SoCs), using open source components. The complexity of reconfigurable systems is primarily determined by the complexity of the hardware-software environment where this reconfigurable system is integrated. Many different reconfigurable architectures and respective development tools have been proposed in the literature, but few reconfigurable computing startups have succeeded at creating sustainable business models. The extraordinary infrastructure needed to operate any computer in general, and a reconfigurable computer in particular, takes several years to develop and requires intensive capital investment. This problem is slowly but effectivelly been addressed by the open source community by making available, free of charge, high quality hardware and software components which innovative companies can use to assemble complex systems. This document describes the creation a simple SoC using the proposed methodology, while providing an overview and evaluation of the open source components used. 
\end{abstract}

%------------------------------------------------------------------------------
\section{Introduction}

The guidelines for paper submission are an important element to help achieving the professional appearance and quality of presentation. With today's computer-based word processors generally available, they are easy to follow. It suffices to take due note of the areas that are typically available for each part of the paper as well as the suggested type and size of the fonts. For better visibility, this sample document is fully written with the suggested font to be used through the paper. You may download it from the web.

%------------------------------------------------------------------------------
\section{Length, Margins, Font and Line Spacing}

The paper must have a maximum length of eight A4 pages for regular communications and four A4 pages for industrial communications.You have to follow these general rules:


\noindent a)	Paper size: DINA4 (21.1 cm x 29.7 cm)

\noindent b)	Margins:

\hspace{0.2cm} b1) 2.5 cm from the top to the first line

\hspace{0.2cm} b2) 2.5 cm as left and right margins

\hspace{0.2cm} b3) 3 cm from the bottom to the last line

\noindent c)	Format: The Title should be written in one column format at the top of the first page, whereas the remaining text should be laid out in a 2-columns format. The space between columns is 1 cm, which leaves 7.5 cm for the width of each column, exactly as formatted in this sample document.

\noindent d)	Font: The suggested font to be used throughout the paper is Times Roman, with varying font sizes according to the relevant portions of the paper to be written. For the title we suggest Times 18 bold, whereas for the author's names and affiliations a smaller Times 12 is suggested. Times 10 is suggested for the main body of the paper, with a simple exception for the section headers where Times 12 bold is advised for greater visibility.

\noindent e)	Line spacing: There are two simple rules take use of single-line spacing throughout the paper. Second, the separation between sections should be done with a single blank line.
Next, you will find a little more detail on how to layout the first page of the paper.


%------------------------------------------------------------------------------
\section{First Page}

Besides the Title, the first page contains the paper Abstract. This is a rather important element that gives a first quick impression of what your paper is about and which new contributions it is proposing. It should therefore be written with outmost care to convey all the necessary and relevant information in the most clear and concise form. Because of its importance in the overall structure of the paper, the Abstract appears at the top of the first column and should not occupy more than about 15 text lines.  We hope you will make a good effort to adhere to this rule.
After the Abstract, we see unfolding the various sections in which the paper is organised. First, the Introduction, where we normally learn about the main motivation and background work relating to your own work as well as the contributions that are being proposed and the form of presenting them. Furthermore, it is also normally here where we find the main body of references to the presented work. These are, of course, extremely important, and should be carefully selected among the world scientific publications relating to your work. The suggested way of calling a reference in the main body of the text is simply to insert between square brackets a sequential number each time a new reference is indicated. For example, \cite{Garcia2005,Franca2006}, and so forth. The full identity of such references, including authors, title, media where published and specific data of the publication (page numbers, date of publication) can then be found at the very end of the paper. It is very easy (and often occurs) to mix-up page numbers and even dates of publication, which can lead to significant frustration any time an interested reader attempts to track a reference. Please be as careful as you can in order to ensure that the information regarding all the references is both complete an reliable. This is obviously a very important and relevant part of your own work and therefore should not be overlooked.

%------------------------------------------------------------------------------
\section{Second Page and Beyond}

The subsequent pages of the paper are laid out in an uniform 2-columns format fully compliant with the margins indicate above: 2.5 cm on each side, 2.5 cm from the top of the paper to the text first line in the page and 3 cm from the bottom of the paper to the last line in the page. Of course, do no forget to keep on using Times Roman 10 for the text with single-line spacing.
One final recommendation. You do not need to number the pages as this will be done at the time of composing the Proceedings.

%------------------------------------------------------------------------------
\section{Some Words About Figures}


Figures are not always easily inserted in a formatted paper, as they widely vary in size, format and even detail of information. In order to cope with this difficulty there are two basic strategies normally followed. The simplest is to put them all together at the end of the paper (usually the last page suffices for this) whereas the hardest is to insert them in the text as they are mentioned and discussed. This latter form clearly gives a greater consistency between text and figures and makes it easier for the reader to follow both, whereas the former can make it a little more difficult for the reader to follow a figure in the last page while reading a portion of the text in a previous page.
Because of is inherent difficulty we would like you to feel free to place the figures as you feel most appropriate, given their nature and size of the figures as well as their relevance in connection with the written text.
Albeit the given freedom for the placement and edition of the figures, we would like to propose one very simple recommendation for uniformity, and which concerns the figure captions. They should be written just below the figure, preferably fully centered with reference to the figure format, and written in  a Times 9 font for differentiation from the main text. The figures are of course sequentially numbered and their number should appear in bold, exactly as indicated in the figure herein.


%------------------------------------------------------------------------------
\section{A Word About Tables}

The guidelines for editing and placing the tables are very simple. Just write them as text and place them as figures. Feel free to use bold face and font sizes (from Times 12 to Times 10) to differentiate the relevant information they contain, as indicated below.

\begin{table}[htbp]
  \centering
    \begin{tabular}{|c|c|c|c|}
    \hline
    {\bf Nov.  20} & {\bf Nov.  21} & {\bf Nov.  22} & {\bf Nov.  23} \\
    \hline
    Day 1 & Day 2 & Day 3 & Day 4 \\
    \hline
    Reception & Work  & Banquet & Work \\
    \hline
    \end{tabular}
  \caption{Illustrating the edition of a table....}
  \label{tab:addlabel}
\end{table}

%------------------------------------------------------------------------------
\section{Conclusions}

You have now all the basic guidelines we would like you to use when making the final preparation of your paper for submission to REC'2010. Please, note again the margins to be respected with reference to an A4-sized page, the areas dedicated to each one of the different portions of your paper as well as the font type and sizes you should use to differentiate them. As you have certainly guessed by now, this sample document has been written in full compliance with such guidelines. Just follow them carefully and you will be able to give your work the careful and high quality appearance that has always been the tradition of IEEE conferences.

\begin{figure}[!htbp]
    \centerline{\includegraphics[width=6.2cm]{twoarms.pdf}}
    \vspace{-3mm}\caption{Two hands and two arms...}
    \label{fig:twohands}
\end{figure}

%------------------------------------------------------------------------------
\bibliographystyle{unsrt}
\bibliography{BIBFile}
\nocite{*}

\end{document}
